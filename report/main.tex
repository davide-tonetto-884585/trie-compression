% --------------------------
% ---- DECLARE PACKAGES ----
% --------------------------
\documentclass[a4paper, oneside, openright]{book}
\usepackage[T1]{fontenc} % Font encoding, T1 = it
\usepackage{lmodern}
\usepackage{multicol} % Per il frontespizio
\usepackage[utf8]{inputenc} % Input encoding - per caratteri particolari
\usepackage[english]{babel} % Lingua principale italiano, con parti in inglese
\usepackage{blindtext} % Per la generazione di paragrafi lorem ipsum
\usepackage{graphicx} % Per includere immagini esterne
\usepackage[a4paper,top=2.5cm,bottom=2.5cm,left=3cm,right=3cm]{geometry} %impaginazione e margini documento
\usepackage[fontsize=12pt]{scrextend} %dimensione font
\usepackage{graphicx}
\usepackage[parfill]{parskip} % Disabilita l'indentazione dopo essere andati a capo
% \usepackage[hang,flushmargin]{footmisc} % Disabilita l'indentazione nelle footnotes
\usepackage{titlesec}
\usepackage{float}
\usepackage[font=scriptsize, skip=5pt]{caption} % Spazio tra la caption e l'immagine
\usepackage[backend=biber, style=numeric, backref=true,defernumbers=true]{biblatex}
\usepackage[immediate]{silence}
\WarningFilter[temp]{latex}{Command} % silence the warning
\usepackage{sectsty}
\DeactivateWarningFilters[temp] % So nothing unrelated gets silenced
\usepackage{hyperref} % Rende l'indice cliccabile
\usepackage[justification=centering]{caption} % Per centrare le captions
\usepackage[bottom]{footmisc} % Posiziona le footnotes alla fine della pagina
\usepackage{tikz}
\usepackage{forest}
\usepackage{tabularx}
\usepackage{multicol}
\usepackage{multirow}
\usepackage{lastpage}
\usetikzlibrary{
  positioning,
  quotes,
  arrows.meta,
  calc,
  automata,
  positioning,
  decorations.pathreplacing,
  tikzmark
}
\usepackage{enumerate, mdwlist}
\usepackage{fancyhdr}
\usepackage{amsthm}
\usepackage{amsmath,amssymb}
\usepackage{minted}
\usepackage{verbatim}
\usepackage{cleveref}
\usepackage{xspace}
\usepackage{enumitem}
\usepackage{subcaption}

% Algorithms
\usepackage{algorithm}
\usepackage{algpseudocode}

\usepackage{csquotes}
\usepackage[most]{tcolorbox}

\newtheorem{definition}{Definition}
\newtheorem{lemma}{Lemma}
\newtheorem{claim}{Claim}
\newtheorem{theorem}{Theorem}

\crefname{claim}{Claim}{Claims}
\crefname{lemma}{Lemma}{Lemmas}
\crefname{theorem}{Theorem}{Theorems}
\crefname{definition}{Definition}{Definitions}
\crefname{algorithm}{Algorithm}{Algorithms}
\crefname{table}{Table}{Tables}
\crefname{figure}{Figure}{Figures}
\crefname{section}{Section}{Sections}
\crefname{subsection}{Subsection}{Subsections}
\crefname{chapter}{Chapter}{Chapters}

\newcommand{\draft}[1]{{\color{red}{#1}}}
\newcommand{\rem}[1]{{\draft{\sout{#1}}}}
\newcommand{\alessio}[1]{{\color[HTML]{34aa15}{\bf{Alessio:}} #1}}
\newcommand{\davide}[1]{{\color{blue}{\bf{Davide T.:}} #1}}

\newcommand{\equivset}{\mathcal{C}}
\newcommand{\equivsetfunc}[1]{\equivset(#1)}

\newcommand{\sourceset}{\mathcal{S}}
\newcommand{\destset}{\mathcal{D}}
\newcommand{\treeset}[1]{\mathcal{T}_{#1}}

% example environment - REPLACE your current definition
\newcounter{example}[chapter]  % Reset per chapter
\renewcommand{\theexample}{\thechapter.\arabic{example}}
\def\exampletext{Example}

\NewDocumentEnvironment{example}{ O{} }
{
\refstepcounter{example}  % This makes labels work properly
\colorlet{colexam}{red!55!black}
\newtcolorbox{examplebox}{% Remove the counter from here
    empty,
    title={\exampletext~\theexample: #1},% Use \theexample instead of \thetcbcounter
    attach boxed title to top left,
    minipage boxed title,
    boxed title style={empty,size=minimal,toprule=0pt,top=4pt,left=3mm,overlay={}},
    coltitle=colexam,fonttitle=\bfseries,
    before=\par\medskip\noindent,parbox=false,boxsep=0pt,left=3mm,right=0mm,top=2pt,breakable,pad at break=0mm,
    before upper=\csname @totalleftmargin\endcsname0pt,
    overlay unbroken={\draw[colexam,line width=.5pt] ([xshift=-0pt]title.north west) -- ([xshift=-0pt]frame.south west); },
    overlay first={\draw[colexam,line width=.5pt] ([xshift=-0pt]title.north west) -- ([xshift=-0pt]frame.south west); },
    overlay middle={\draw[colexam,line width=.5pt] ([xshift=-0pt]frame.north west) -- ([xshift=-0pt]frame.south west); },
    overlay last={\draw[colexam,line width=.5pt] ([xshift=-0pt]frame.north west) -- ([xshift=-0pt]frame.south west); },
}
\begin{examplebox}}
{\end{examplebox}\endlist}

% Standard cleveref setup
\crefname{example}{Example}{Examples}
\Crefname{example}{Example}{Examples}

% ------------------------
% ---- DOCUMENT SETUP ----
% ------------------------
\pagestyle{plain}
\setlength{\headheight}{20pt}

\raggedbottom % Se la pagina non è completa, lascia lo spazio alla fine

\titleformat{\chapter}[display]
    {\normalfont\huge\bfseries}{\chaptertitlename\ \thechapter}{10pt}{\LARGE}
\titlespacing*{\chapter}{0pt}{0pt}{20pt}
\chaptertitlefont{\fontsize{22pt}{30pt}\selectfont}

\hypersetup{ % Setup dell'aspetto dei link
    colorlinks,
    citecolor=black,
    filecolor=black,
    linkcolor=black,
    urlcolor=black
}

% \renewcommand{\footnoterule}{ % Rende la linea delle footnotes larga tutta la pagina
%   \kern -3pt
%   \hrule width \textwidth height 1pt
%   \kern 2pt
% } 
\renewcommand{\footnotesize}{\fontsize{11pt}{13pt}\selectfont} % Imposta la dimensione del testo delle footnotes
\setlength{\footnotesep}{0.5cm} % Imposta lo spazio fra e singole footnotes
\setlength{\skip\footins}{1.5cm} % Imposta lo spazio fra il corpo e le footnotes

\usepackage[normalem]{ulem}


\DeclareUnicodeCharacter{02BC}{}

% ------------------------
% ---- DOCUMENT START ----
% ------------------------
\addbibresource{bibliography.bib} % Importiamo la bibliografia
\begin{document}
\pagenumbering{roman} 
\begin{titlepage}
\begin{figure}[!htb]
    % \centering
    \includegraphics[width=4cm]{Immagini/logo 2 unive.png}
\end{figure}

\begin{center}
    % \Large{\textbf{UNIVERSITÀ CA' FOSCARI DI VENEZIA}}
    % \vspace{3mm}
    % \\ \normalsize{DIPARTIMENTO DI SCIENZE AMBIENTALI, INFORMATICA E STATISTICA}
    % \vspace{6mm}
    \vspace{13mm}
    \normalsize{\textbf{Artificial Intelligence and Data Engineering}}
    \vspace{13mm}
    \\ \normalsize{Master Degree Thesis}
\end{center}

\vspace{10mm}
\begin{center}
    \LARGE{\textbf{A New Compression Technique for Repetitive Tries}}
\end{center}

\vspace*{\fill}

\begin{minipage}[t]{1\textwidth}
    {\normalsize{\textbf{Supervisor}}{\normalsize\vspace{1mm}
    \\ \normalsize{Nicola Prezza }}} \\ 

    {\normalsize{\textbf{Co-supervisor}}{\normalsize\vspace{1mm}
    \\ \normalsize{Alessio Campanelli }}} \\ 
        
    {\normalsize{\textbf{Author}}{\normalsize\vspace{1mm}
    \\ \normalsize{Davide Tonetto}\\
    \normalsize{Student ID 884585 }}} \\
\end{minipage}

\begin{flushleft}
    {\normalsize{\textbf{Academic year}}{\normalsize\vspace{1mm}
    \\ \normalsize{2024/2025}}}  
\end{flushleft}

\end{titlepage}
 % PAGINA FRONTESPIZIO
\newpage
\
\newpage
\chapter*{Abstract}
\addcontentsline{toc}{chapter}{Abstract}

This thesis introduces a novel compression technique for repetitive tries, which are fundamental data structures for representing large sets of strings in applications like bioinformatics and natural language processing. While tries are efficient for prefix-based queries, their inherent repetitiveness often leads to a large memory footprint. Traditional compression algorithms often fail to exploit this redundancy, and even specialized techniques like the Extended Burrows-Wheeler Transform (XBWT) may not be optimal in such cases.

The proposed method aims to find an effective trade-off between high compression and efficient indexing. The core idea is to identify and merge identical subtrees by reducing the trie to a minimal deterministic finite automaton (DFA). However, a fully minimized DFA is difficult to index efficiently. Therefore, this work proposes a partial minimization of the trie that ensures the resulting automaton is \textit{$p$--sortable}, a property that allows for efficient indexing.

The problem is framed as a \textbf{String Partitioning Problem}, where the sequence of nodes in the trie, ordered co--lexicographically, is partitioned into \textit{p} subsequences to maximize the merging of equivalent states. This optimization problem is then reduced to finding a \textbf{Minimum Weight Perfect Bipartite Matching (MWPBM)}, which can be solved efficiently.

The result is a compressed, $p$--sortable automaton that supports efficient queries. Experimental results demonstrate the effectiveness of this method, particularly for highly repetitive datasets, achieving a balance of compression and indexability that prior methods could not attain. Our approach opens new possibilities for managing and querying large-scale string datasets in a memory-efficient manner. % PAGINA ABSTRACT
\newpage
\

\tableofcontents  % Genera l'indice
% \addcontentsline{toc}{chapter}{\listfigurename}
% \listoffigures
\newpage % Nuova pagna

\fancypagestyle{IHA-fancy-style}{%
  \fancyhf{}% Clear header and footer
  %\fancyhead[LE,RO]{\slshape \rightmark}
  \fancyhead[R]{\slshape \leftmark}
  \fancyfoot[C]{\thepage\ of \pageref{LastPage}}% Custom footer
  \renewcommand{\headrulewidth}{0.4pt}% Line at the header visible
  %\renewcommand{\footrulewidth}{0.4pt}% Line at the footer visible
}

% Redefine the plain page style
\fancypagestyle{plain}{%
  \fancyhf{}%
  \fancyfoot[C]{\thepage\ of \pageref{LastPage}}%
  \renewcommand{\headrulewidth}{0pt}% Line at the header invisible
  \renewcommand{\footrulewidth}{0pt}% Line at the footer visible
}

\pagestyle{IHA-fancy-style}

\pagenumbering{arabic} % Riabilita la numerazione in modo che cominci dal primo capitolo
\setcounter{chapter}{0} % Fa risultare l'introduzione come capitolo 0
% ------------------
% ---- CHAPTERS ----
% ------------------
\chapter{Introduction} \label{chp:introduction}
\section{Project Overview}
The increasing availability of large, structured datasets, such as those found in XML documents, biological data, and hierarchical knowledge bases, has led to the need for efficient compression techniques for trees.
Trees are a natural choice for representing hierarchical data due to their ability to model parent-child relationships and nested structures. For instance, an XML document is inherently a tree, where tags are nested to create a structured hierarchy. Similarly, file systems are organized as trees of directories and files, and biological data, such as phylogenetic trees, use this structure to represent evolutionary relationships. Given their ubiquity in representing complex data, developing effective compression methods for trees is of paramount importance.
Traditional compression methods, such as general-purpose text compression algorithms, often fail to effectively exploit the hierarchical structure of trees. Consequently, specialized tree compression techniques have been developed to address this issue.

Among the most prominent techniques for tree compression, the \textit{Extended Burrows-Wheeler Transform} \cite{ferragina2009compressing} extends the classical Burrows-Wheeler Transform to labeled trees, leveraging their structural properties to achieve significant compression. Another notable approach includes \textit{Re-Pair-based compression} \cite{lohrey2011tree}, which applies grammar-based compression to the tree structure. 

Despite these advancements, existing techniques may not be optimal when dealing with trees characterized by a high degree of repetitiveness. Many real-world datasets, such as versioned documents or biological phylogenies, contain repeated substructures that can be exploited to achieve better compression. This thesis aims to study a novel compression technique designed to efficiently handle such highly repetitive trees. We implement and evaluate this method, comparing it with existing state of the art approaches to determine its effectiveness in different scenarios.

\section{Background and State of The Art} \label{sec:background}
Before the advent of the XBWT, Kosaraju \cite{kosaraju1989efficient} proposed a method to index labeled trees by extending the concept of prefix sorting, which is commonly applied to strings, to work with labeled trees by leveraging the structure of tries (prefix trees). To achieve this, he introduced the idea of constructing a suffix tree for a reversed trie allowing subpath queries in $O(|P|\log|\Sigma|+ occ)$ time, where $occ$ is the number of occurrences of $P$ in $T$ but still requiring $O(t \log t)$ space and so not being compressed.

The Extended Burrows-Wheeler Transform (XBWT) \cite{ferragina2009compressing} is a data structure designed for efficient compression and indexing of ordered node-labeled trees.
The XBWT works by linearizing a labeled tree into two arrays: one captures the structural properties of the tree, and the other stores its labels. This transformation allows for efficient representation, navigation, and querying of the tree. The key advantage of the XBWT lies in its ability to compress labeled trees while supporting a wide range of operations, such as parent-child navigation and sophisticated path-based searches, in (near-)optimal time and space.
The XBWT provides significant improvements in both compression ratio and query performance compared to traditional compression schemes, making it a valid resource for intensive applications.

Another notable approach is Tree Re-Pair \cite{lohrey2011tree}, a grammar-based compression technique adapted for tree structures. It extends the principles of the original Re-Pair algorithm \cite{larsson2000off} to handle the hierarchical nature of trees by identifying and compactly representing frequently occurring patterns.
The core idea of the tool is to identify frequently occurring patterns within the tree and represent them more compactly.
The process involves the linearization of the tree (e.g., using a specific traversal order) and then the application of the Re-Pair logic. In this way, it finds the most frequent pair of adjacent elements (which could represent nodes, labels, or structural components, depending on the linearization) in the sequence. The pair is then replaced by a new non-terminal symbol, and the corresponding production rule is added to a grammar. All this process is then repeated until no more pairs occur frequently enough or some other stopping criterion is met. The final output is a relatively small grammar (a set of production rules) and a sequence of symbols (including the newly introduced non-terminals) that can be used to reconstruct the original tree. An application of Tree Re-Pair to XML documents can be found in \cite{lohrey2013xml}.
While Tree Re-Pair is effective for general tree compression, this thesis focuses on developing a novel technique specifically tailored for highly repetitive trees. Therefore, we use the XBWT as our primary benchmark for comparison, as it represents a well-established and high-performance baseline in the field.

\section{Challenges and Contributions}
In order to develop an effective tree compression scheme that can exploit repetitive structures, we need to address several key challenges:
\begin{itemize}
    \item \textbf{Identification of repetitive structures:} The first step in compressing repetitive trees is to identify the repeated substructures efficiently. This requires the development of algorithms capable of detecting and representing these structures compactly.
    \item \textbf{Optimization of representation:} Once the repetitive structures have been identified, the challenge is to represent them in an optimized way that minimizes the overall size of the compressed tree. This involves finding the most efficient encoding for the repeated substructures.
\end{itemize}

We address these challenges by developing a novel tree compression scheme that first leverages the well-known automata minimization algorithm to identify repetitive structures. This algorithm efficiently groups together similar subtrees, enabling us to identify and compress them with high efficiency. The tree is treated as a deterministic finite automaton (DFA) where the root is the initial state and the leaves are the final states. Since trees are acyclic graphs, this structure is a specific type of DFA known as a Directed Acyclic Word Graph (DAWG). For this reason, we focus on an adaptation of Revuz's algorithm, which is specifically designed for minimizing acyclic DFAs, to make it more efficient for our purposes.

Then, we optimize the representation of these structures. Our approach is to partition the tree nodes into chains and apply Run-Length Encoding (RLE), a compression technique that stores sequences of identical data as a single value and a count. The key challenge is to create partitions that maximize the effectiveness of RLE. We prove that this optimization problem can be reduced to the Minimum Weight Perfect Bipartite Matching (MWPBM) problem. MWPBM is a classic graph theory problem focused on finding a pairing of all nodes in a bipartite graph such that the sum of the weights of the connecting edges is minimized. By modeling our partitioning problem as a bipartite graph, we can use efficient algorithms for MWPBM to find the optimal representation and achieve a higher compression ratio.

\section{Structure of The Thesis}
This thesis is structured to guide the reader from the foundational concepts of tree compression to the development and evaluation of our novel approach. The goal is to build a clear understanding of why each component of our proposed pipeline is necessary and how they fit together.

The logical flow is as follows:
\begin{itemize}
    \item We begin in \textbf{\cref{chp:thbg_labeled_tree}} by establishing the necessary theoretical background on labeled trees.
    \item In \textbf{\cref{chp:tree_compression}}, we examine the Extended Burrows-Wheeler Transform (XBWT), a state-of-the-art tree compression technique. This serves as a benchmark and highlights the opportunity for improvement, particularly in handling highly repetitive structures.
    \item To address this, we introduce a new approach based on automata. \textbf{\cref{chp:hopcroft}} describes how we use Deterministic Finite Automata (DFA) to model the tree's structure and apply Hopcroft's algorithm to minimize this automaton, effectively identifying all unique subtrees (i.e., the repetitions).
    \item Once repetitions are identified, we need an efficient way to store them. \textbf{\cref{chp:min_weight_perfect_bipartite_matching}} introduces the Minimum Weight Perfect Bipartite Matching problem, which we use to find an optimal way to chain the identified repetitive structures, minimizing the overall compressed size.
    \item \textbf{\cref{chp:project_overview}} unites these concepts, presenting the complete pipeline of our proposed compression scheme.
    \item Finally, \textbf{\cref{chp:implementation,chp:experimental_results,chp:conclusions}} describe the implementation, present the experimental results of our method against the benchmark, and discuss our conclusions and future work.
\end{itemize}

\newpage
\
\newpage
\chapter{Preliminary Concepts}
\section{Basic Notation and Concepts} \label{sec:notation}
% \alessio{As of now this is a stash of general stuff that is important to explain in a general section.}

%\par\noindent\rule{\textwidth}{0.4pt}

\begin{definition}[String]\label{def:string}
    A \textbf{string} is a sequence of characters $S = s_1s_2\ldots s_n$ drawn from an alphabet $\Sigma$.
\end{definition}
We use $|S| = n$ to denote the length of $S$.
The set of all strings is represented as $\Sigma^*$, which also contains the empty string $\varepsilon$.\newline
The notation $S[i] = s_i$ denotes the $i$-th character of $S$, and $S[i\dots j]$ the substring $s_is_{i+1}\ldots s_j$, for $1 \leq i,j \leq n$.
Therefore, a prefix of $S$ is a substring of the type $S[1\dots j]$, while a suffix is a substring $S[i \dots n]$.

\begin{definition}[Subsequence]
    $S' = s'_1\ldots s'_m$ is a \textbf{subsequence} of some string $S = s_1\ldots s_n$ if there exist a strictly increasing sequence of indices $1 \le i_1 < i_2 < \ldots < i_m \le n$ such that $s'_j = s_{i_j}$ for all $j = 1,\ldots, m$.    
\end{definition}
In other words, a subsequence is a sequence of characters from a string that are not necessarily contiguous, but are in the same order as they appear in the original string.

We recall the following important notation for strings from the introduction section: 
\vspace{-1em}
\setsubseqdef*

%\par\noindent\rule{\textwidth}{0.4pt}
A basic understanding of formal grammars is crucial for comprehending some of the state-of-the-art tree compression methods discussed in this thesis.
\begin{definition}[Grammar]
    A \textbf{formal grammar} is a set of rules for generating strings in a formal language. A grammar is typically defined as a 4-tuple $G = (N, \Sigma, P, S)$, where:
    \begin{itemize}
        \item $N$ is a finite set of non-terminal symbols.
        \item $\Sigma$ is a finite set of terminal symbols, disjoint from $N$.
        \item $P$ is a finite set of production rules, each of the form $(\alpha \rightarrow \beta)$, where $\alpha$ is a string of symbols from $(N \cup \Sigma)^*$ containing at least one non-terminal symbol, and $\beta$ is a string in $(N \cup \Sigma)^*$.
        \item $S \in N$ is the start symbol.
    \end{itemize}
\end{definition}

\begin{example}
    For example, consider the grammar $G = (\{S\}, \{a, b\}, P, S)$ with the following production rules in $P$:
    \begin{enumerate}
        \item $S \rightarrow aSb$
        \item $S \rightarrow \epsilon$
    \end{enumerate}
    This grammar generates the language $\{a^n b^n \mid n \geq 0\}$, which includes strings like $\epsilon$ (the empty string), $ab$, $aabb$, $aaabbb$, and so on.
\end{example}

Let us now move to the definition of rooted tree.
\begin{definition}[Tree] \label{def:rooted_tree}
    A \textbf{rooted tree} is a triple $T = (V, E, r)$ where:
    \begin{itemize}
        \item $V$ is a finite set of vertices (or nodes),
        \item $E\subseteq V \times V$ is a set of edges such that $|E| = |V|-1$ and the underlying graph is connected and acyclic,
        \item $r\in V$ is the root vertex.
    \end{itemize}
\end{definition}

We denote by $t = |V|$ the number of vertices in the tree.

\begin{definition}[Tree Terminology] \label{def:tree_terminology}
Given a rooted tree $T = (V, E, r)$, we define the following concepts:
\begin{itemize}
    \item \textbf{Parent:} For any two distinct nodes $u,v\in V$, with $v\neq r$, we say that $u$ is the parent of $v$ iff $(u, v) \in E$ and $u$ lies on the unique path from $r$ to $v$. It follows that each node $v$ (except from the root) has exactly one parent, which we denote as $\textup{parent}(v)$. 
    \item \textbf{Children:} For a vertex $u\in V$, we define the set $\textup{children}(u) = \{v \in V | (u,v)\in E \land u = \textup{parent}(v)\}$
    \item \textbf{Leaves:} A vertex $v \in V$ is a \textit{leaf} if it has no children, i.e., there is no edge $(v, w) \in E$ for any $w \in V$.
    \item \textbf{Degree:} The \textit{degree} of a node $u$, denoted $\text{deg}(u)$, is its number of children.
    \item \textbf{Depth:} The \textit{depth} (or level) of a vertex $v$, denoted $\text{depth}(v)$, is the length of the unique path from the root $r$ to $v$ minus 1. The root has depth 0.
    \item \textbf{Height:} The \textit{height} of the tree is the maximum depth among all its leaves.
    \item \textbf{Subtree:} For any vertex $v \in V$, the \textit{subtree rooted at $v$} is the tree $T_v = (V', E', v)$ where $V'$ contains $v$ and all its descendants, and $E'$ contains all edges between vertices in $V'$.
    \item \textbf{Linearization:} A \textit{linearization} of a tree is a total ordering of its vertices that respects some traversal order (e.g., pre-order, post-order, or in-order).
\end{itemize}
\end{definition}

% \par\noindent\rule{\textwidth}{0.4pt}

\section{Finite State Automata} \label{sed:fsa}
Finite automata are fundamental computational models that recognize regular languages through a finite set of states and transitions. They provide an elegant mathematical framework for representing and manipulating collections of strings, making them particularly suitable for applications in pattern matching, lexical analysis, and data compression. In the context of this thesis, finite automata serve as the underlying representation for tries and other string data structures, enabling efficient compression through state minimization.

Let $L \subseteq \Sigma^*$ be a finite set of strings. $L$ can be represented in many different ways, such as enumeration, context-free grammars, regular expressions, or automata.

\begin{definition}[Non-deterministic Finite Automaton] \label{def:nfa}
    A \textbf{non-deterministic finite automaton} (NFA) is a 5-tuple $\nfa = (Q, \Sigma, \delta, q_0, F)$ where:
    \begin{itemize}
        \item $Q$ is a finite set of states,
        \item $\Sigma$ is a finite alphabet,
        \item $\delta: Q \times \Sigma \to \mathcal{P}(Q)$ is the transition function, where $\mathcal{P}(Q) = \{A | A \subseteq Q \}$ is the powerset of $Q$,
        \item $q_0 \in Q$ is the initial state,
        \item $F \subseteq Q$ is the set of final (accepting) states.
    \end{itemize}
\end{definition}

An NFA processes an input string $S \in \Sigma^*$ one symbol at a time, starting from $q_0$ and following the transitions specified by $\delta$.
Let $\hat\delta: Q \times \Sigma^* \to \mathcal{P}(Q)$ be the extension of $\delta$ to strings defined as follows.
For all $u \in Q$, $a \in \Sigma$, and $\alpha \in \Sigma*$:
\[
\hat\delta(u,\varepsilon)=\{u\}, \qquad
\hat\delta(u,\alpha a)=\bigcup_{v \in \hat\delta(u,\alpha)} \delta(v,a).
\]
Therefore, $\delta(q_0,\alpha)$ denotes the set of states that can be reached from the start state $q_0$ by reading $\alpha$.
A string $S$ is \emph{accepted} if $\delta(q_0, S)\cap F \neq \emptyset$, or, in other words, if the automaton ends in any state $q \in F$ after processing the entire string.
The set of all strings accepted by $\nfa$ is called the \emph{language} of the automaton, and is denoted as
\[
\lang{\nfa}=\{S \in \Sigma^* \mid \delta(q_0, S)\cap F \neq \emptyset\}.
\]

Deterministic Finite Automata are a special case of NFAs, where each state has exactly one outgoing transition per character, i.e., $|\delta(q, a)| = 1$.

\begin{definition}[Deterministic Finite Automaton] \label{def:dfa}
    A \textbf{deterministic finite automaton} (DFA) is a 5-tuple $\dfa = (Q, \Sigma, \delta, q_0, F)$ where:
    \begin{itemize}
        \item $Q$ is a finite set of states,
        \item $\Sigma$ is a finite input alphabet,
        \item $\delta: Q \times \Sigma \rightarrow Q$ is the transition function,
        \item $q_0 \in Q$ is the initial state,
        \item $F \subseteq Q$ is the set of final (accepting) states.
    \end{itemize}
\end{definition}
Notice how the only difference between \cref{def:nfa} and \cref{def:dfa} is the return value of the transition function.
As a consequence, the expansion of $\delta$ to strings can be simplified to 
\[
    \hat\delta(u, \alpha a) = \delta(\hat\delta(\alpha), a)
\]
Similarly to NFAs, the language of a DFA is the set of strings that end at a state $q \in F$ after being processed:
\[
    \lang{\dfa} = \{S \in \Sigma^* \mid \hat\delta(q_0, a) \in F\}
\]

Also, we introduce the following notation for automata: For \( q \in Q \) we write \( I_q \) to denote the set of strings reaching \( q \) from the initial state:
\[
I_q = \{ \alpha \in \Sigma^* \mid q = \delta(q_0, \alpha) \}
\]

NFAs can be more compact in terms of the number of states required to represent a language: a well-known fact is that the smallest DFA for a language could be exponentially larger than the smallest equivalent NFA. In this thesis this will not be a concern, since we will assume that the input language is presented via a DFA (more specifically, a trie). 

%DFAs also present advantages in terms of compressibility \nicola{questa frase è in contrasto con il precedente paragrafo: spiega meglio}. For instance, as shown in \cref{thm:indexing}, when using compression schemes based on $p$-sortable automata (\cref{def:p-sorable-automaton}), DFAs offer a space advantage. A $p$-sortable DFA requires $\log p$ fewer bits per edge than its NFA counterpart, making it more space-efficient in this context. \nicola{riformula: DFA log p, mentre NFA 2log p (così non è chiaro quanto spazio occupano gli NFA)}

Finally, automata are naturally represented as labeled directed graphs, where the vertices are the states and the edges represent the transitions, labeled with characters from the alphabet.

\begin{example}
Figure \ref{fig:nfa_dfa_example} shows an example of an NFA and an equivalent DFA that both accept the language of strings over alphabet $\{a, b\}$ ending with the character `a'.

\begin{figure}[H]
    \centering
    \begin{subfigure}[b]{0.45\textwidth}
        \centering
        \begin{tikzpicture}[shorten >=1pt,node distance=2cm,on grid,auto] 
           \node[state,initial,initial text=] (q_0)   {$n_0$}; 
           \node[state,accepting](q_1) [right=of q_0] {$n_1$}; 
            \path[->] 
            (q_0) edge [loop above] node {$a,b$} ()
                  edge node {$a$} (q_1);
        \end{tikzpicture}
        \caption{}
        \label{fig:nfa_example}
    \end{subfigure}
    \hfill
    \begin{subfigure}[b]{0.45\textwidth}
        \centering
        \begin{tikzpicture}[shorten >=1pt,node distance=2cm,on grid,auto] 
           \node[state,initial, initial text=] (A) {$d_0$};
           \node[state,accepting](B) [right=of A] {$d_1$};
            \path[->] 
            (A) edge [loop above] node {b} ()
                edge node {a} (B)
            (B) edge [loop above] node {a} ()
                edge [bend right] node [above] {b} (A);
        \end{tikzpicture}
        \caption{}
        \label{fig:dfa_example}
    \end{subfigure}
    \caption{(a) An NFA and (b) an equivalent DFA for the language $(a|b)^*a$.}
    \label{fig:nfa_dfa_example}
\end{figure}
\end{example}

\subsection{Minimization} \label{sec:hopcroft}
The process of automata minimization consists of reducing the number of states in an automaton while preserving its accepted language. 
As extensively shown in the literature, determinism makes this problem much easier: while minimizing DFAs can be achieved in polynomial time, minimizing NFAs is a PSPACE--complete problem. 

Tries are inherently deterministic, since from any node, there is at most one outgoing edge for each symbol in the alphabet. 
This deterministic nature will be crucial in this work to apply powerful automata minimization techniques to compress the tree structure.

The minimization of DFAs is a well-studied problem in automata theory, and there are several algorithms available for this purpose. One of the most popular algorithms for DFA minimization is Hopcroft's algorithm, which was proposed by John Hopcroft in 1971~\cite{HOPCROFT1971189}. Hopcroft's algorithm is an efficient and simple algorithm that can minimize a DFA in $O(n \log n)$ time, where $n$ is the number of states in the DFA.

The algorithm enables computing equivalence classes of nodes, in particular, the Myhill--Nerode equivalence classes~\cite{nerode1958linear, myhill1957finite}. The Myhill--Nerode theorem states that a language is regular if and only if it has a finite number of Myhill--Nerode equivalence classes. This theorem provides a powerful tool for determining the regularity of languages and is a cornerstone of automata theory. Let us formalize the concept of equivalence classes and the Myhill--Nerode theorem.

\begin{definition}[Myhill--Nerode Equivalence Relation]
    For a language $L \subseteq \Sigma^*$ and any strings $x,y \in \Sigma^*$, we say $x$ is equivalent to $y$ with respect to $L$ (written as $x \approx_L y$) if and only if for all strings $z \in \Sigma^*$:
    \[ xz \in L \Leftrightarrow yz \in L \]
\end{definition}
That is, strings $x$ and $y$ are equivalent if they have the same behavior with respect to the language $L$: either they both lead to acceptance or both lead to rejection when any suffix $z$ is appended.

\begin{theorem}[Myhill--Nerode theorem~\cite{nerode1958linear,myhill1957finite}] \label{def:myhill-nerode}
    Let $L$ be a language over an alphabet $\Sigma$. Then $L$ is regular if and only if there exists a finite number of Myhill--Nerode equivalence classes for $L$. Specifically, the number of equivalence classes is equal to the number of states in the minimal DFA recognizing $L$.
\end{theorem}

Throughout this section let $M = (Q, \Sigma, \delta, q_0, F)$ be a DFA. For $q \in Q$ and $a \in \Sigma$, we adopt the shorthand $q.a := \delta(q,a)$. We extend $\delta$ to words by the usual recursion:
\[
\delta^{*}(q,\varepsilon) = q, \qquad
\delta^{*}(q,wa) = \delta\bigl(\delta^{*}(q,w),\, a\bigr) \quad \text{for } w \in \Sigma^{*},\ a \in \Sigma .
\]
For a word $w = w_1 w_2 \dots w_n \in \Sigma^{*}$, we then write $q.w := \delta^{*}(q,w)$ for the (unique) state reached from $q$ by reading $w$. A word $w$ is accepted by $M$ iff $\delta^{*}(q_0,w) \in F$.

Also, Let $M=(Q,\Sigma,\delta,q_0,F)$ be a DFA recognizing $L$. For states (nodes) $u,v\in Q$, we say that $u$ and $v$ are MN--equivalent iff
\[
\forall \alpha \in \Sigma^*:\ u.\alpha \in F \iff v.\alpha \in F.
\]

\subsection{Revuz' Minimization Algorithm} \label{sec:revuz}
For our purpose, we will focus on Acyclic Deterministic Finite Automata (ADFA). An ADFA is a DFA where the transition graph contains no cycles. The acyclic property is key, as it simplifies the minimization process significantly. 

In this section, we will discuss an efficient algorithm for minimizing ADFAs in linear time on the number of edges~\cite{revuz1992minimisation}.

Let us begin by providing some definitions needed to understand the algorithm.

\begin{definition}[Height function] \label{def:height}
    For a state $s$ in an automaton, the height $h(s)$ is defined as the length of the longest path starting at $s$ and going to a final state. 

    $$h(s) = \max\{|w|:s.w \text{ is final}\}$$
\end{definition}

This height function induces a partition $\Pi_i$ of $Q$, where $\Pi_i$ denotes the set of states of height $i$.

\begin{comment}
\begin{definition}[Distinguished set]
    We say that a set $\Pi_i$ is distinguished if no pair of states in $\Pi_i$ are MN--equivalent.
\end{definition}
\end{comment}

Now we define the canonical label of each state that will be necessary to identify MN--equivalent states. For $s\in Q$, let $l_1<\cdots<l_k$ be the symbols of the outgoing transitions defined at $s$ (listed in increasing order of $\Sigma$). With $b=\text{F}$ if $s\in F$ and $b=\text{NF}$ otherwise, we set
\[
\mathrm{l}(s) := \big(b,\, l_1,\, s.l_1,\, l_2,\, s.l_2,\, \dots,\, l_k,\, s.l_k\big).
\]

Also, the algorithm uses a function $R$ to map the labels of states to a new signature. This function is defined as follows:
\begin{definition}[Signature map $R$] \label{def:R}
Let $N[\,\cdot\,]$ be the current renaming array that assigns to each state its equivalence class identifier Myhill--Nerode.  
For a state $s$ labeled
\[
\mathrm{l}(s) \;=\; \big(b,\, l_1,\, s.l_1,\, l_2,\, s.l_2,\, \dots,\, l_k,\, s.l_k\big),
\]
where $b \in \{\text{F},\text{NF}\}$, $l_i \in \Sigma$ (listed in increasing order), and $nl_i \in Q$, define
\[
R\!\left(\mathrm{l}(s)\right) \;=\; \big(b,\, l_1,\, N[s.l_1],\, l_2,\, N[s.l_2],\, \dots,\, l_k,\, N[s.l_k]\big).
\]
\end{definition}

It is important to notice that, since the automaton is acyclic, every transition $s \xrightarrow{a} t$ strictly decreases the height: $h(t) < h(s)$. The main loop of \cref{alg:minimization-ADFA} processes levels in increasing order $i=0,1,\dots$, so by the time we handle a state $s\in\Pi_i$, all its targets $t$ lie in $\bigcup_{j<i}\Pi_j$ and have already been assigned a Myhill--Nerode equivalence class.

\begin{algorithm}[H]
    \caption{$\textsc{RevuzMinimization}(M)$}
    \label{alg:minimization-ADFA}
    \begin{algorithmic}[1]
    \Require $M=(Q,\Sigma,\delta,q_0,F)$ is an ADFA.
    \Ensure Minimal DFA $M'=(\{1,\dots,n\},\Sigma,\delta',N[q_0],F')$ with $F'=\{N[q]\mid q\in F\}$ and $\delta'(N[q],a)=N[\delta(q,a)]$.
    \State Calculate height $h(s)$ for every state $s$
    \State Create partitions $\Pi_i = \{s \in Q \mid h(s) = i\}$
    \State $N[1, |Q|] = \{1,2,\dots,|Q|\}$ \Comment{Renaming array}
    \State $n = 0$
    \For{$i := 0$ to $h(q_0)$} \Comment{$q_0$ is the initial state}
        \State Sort states in $\Pi_i$ based on $R(l(q)), q\in \Pi_i$
        \State $n = n + 1$
        \State $N[\Pi_i[1]] = n$;
        \For{$j := 2$ to $|\Pi_i|$}
            \If{$R(l(\Pi_i[j])) \ne R(l(\Pi_i[j-1]))$}
                \State $n = n + 1$
            \EndIf
            \State $N[\Pi_i[j]] = n$
        \EndFor
    \EndFor
    \end{algorithmic}
\end{algorithm}

The algorithm proceeds level by level, from $i=0$ up to the maximum height, ensuring that states at each level are correctly partitioned into Myhill--Nerode equivalence classes. For each level $i$, it groups the states in $\Pi_i$ based on their signatures computed by the function $R$ (see \cref{def:R}). As explained before, when processing level $i$, the equivalence classes for all states in lower levels ($j < i$) have already been finalized. The signature $R(l(s))$ for a state $s$ depends on its finality and the equivalence classes of its immediate successors. Therefore, two states $s, t \in \Pi_i$ have the same signature if and only if they are MN--equivalent. The algorithm assigns a unique class identifier to each group of states with the same signature.

The whole algorithm can be implemented to run in time $O(m)$ for an acyclic automaton with $m$ edges. Heights may be computed in linear time by
a bottom-up traversal. The lists of states of a given height are collected during this traversal. The signature of a state is easy to compute provided the edges starting in a state have
been sorted (by a bucket sort for instance to remain within the linear time constraint).
Sorting states by their signature again is done by a lexicographic sort~\cite{berstel2010minimization}. 

\begin{example} 
    \label{ex:ADFA_minimization}
    Now we are going to see an example of reduction for a given ADFA. The ADFA is represented in figure \cref{fig:example_ADFA} and, as we can notice, it is also a valid ordered rooted tree with $n = 11$ nodes, $e = 10$ edges, and the following alphabet: $\Sigma = \{0, 1\}$. The node $a$ is the root of the tree and the initial state of the automaton, while the leaf nodes $e,g,h,i,l,m$ are final states. It is important to note that while the algorithm applies to any ADFA, our focus is on those that are also trees, as this is the specific case relevant to our work.

    \begin{figure}[H]
        \centering
        \begin{tikzpicture}[
            level distance=1.5cm,
            sibling distance=3cm,
            state/.style={circle, draw, minimum size=7mm},
            accepting/.style={circle, draw, double, minimum size=7mm},
            edge from parent/.style={draw, -latex},
            level 1/.style={sibling distance=4cm},
            level 2/.style={sibling distance=2.5cm},
            level 3/.style={sibling distance=2cm}
            ]
        
        \node[state] (a) {a}
            child {node[state] (b) {b} 
            child {node[state] (d) {d}
                child {node[accepting] (h) {h}}
                child {node[accepting] (i) {i}}
            }
            child {node[accepting] (e) {e}}
            }
            child {node[state] (c) {c}
            child {node[state] (f) {f}
                child {node[accepting] (l) {l}}
                child {node[accepting] (m) {m}}
            }
            child {node[accepting] (g) {g}}
            };
        
        % Etichette degli archi
        \path (a) -- (b) node[midway, left] {0};
        \path (a) -- (c) node[midway, right] {1};
        \path (b) -- (d) node[midway, left] {0};
        \path (b) -- (e) node[midway, right] {1};
        \path (c) -- (f) node[midway, left] {0};
        \path (c) -- (g) node[midway, right] {1};
        \path (d) -- (h) node[midway, left] {0};
        \path (d) -- (i) node[midway, right] {1};
        \path (f) -- (l) node[midway, left] {0};
        \path (f) -- (m) node[midway, right] {1};
        \end{tikzpicture}
        \caption{Example ADFA to be minimized}
        \label{fig:example_ADFA}
    \end{figure}

    Now, let us apply the minimization algorithm step by step:
    \begin{enumerate}
        \item \textbf{Height Computation:} First, we compute the height of each state. The height is the length of the longest path to a final state. The final states ($e, g, h, i, l, m$) have a height of 0. For the other states, the height is calculated as follows:
        \begin{itemize}
            \item $h(d) = 1 + \max(h(h), h(i)) = 1 + 0 = 1$
            \item $h(f) = 1 + \max(h(l), h(m)) = 1 + 0 = 1$
            \item $h(b) = 1 + \max(h(d), h(e)) = 1 + \max(1, 0) = 2$
            \item $h(c) = 1 + \max(h(f), h(g)) = 1 + \max(1, 0) = 2$
            \item $h(a) = 1 + \max(h(b), h(c)) = 1 + \max(2, 2) = 3$
        \end{itemize}
        This gives us the following partitions based on height:
        \begin{itemize}
            \item $\Pi_0 = \{e, g, h, i, l, m\}$
            \item $\Pi_1 = \{d, f\}$
            \item $\Pi_2 = \{b, c\}$
            \item $\Pi_3 = \{a\}$
        \end{itemize}

        \item \textbf{Processing $\Pi_0$:} All states in $\Pi_0$ are final and have no outgoing transitions, so they are all equivalent. We merge them into a single class, let us call it $D = \{e, g, h, i, l, m\}$. After this step, we have a new state $D$ which is final.

        \item \textbf{Processing $\Pi_1$:} Now we examine the states in $\Pi_1$: $d$ and $f$. We check their transitions:
        \begin{itemize}
            \item State $d$: $\delta(d, 0) = h \in D$ and $\delta(d, 1) = i \in D$.
            \item State $f$: $\delta(f, 0) = l \in D$ and $\delta(f, 1) = m \in D$.
        \end{itemize}
        Since both states transition to the same equivalence class ($D$) for both symbols $0$ and $1$, they are equivalent. We merge them into a new class, $C = \{d, f\}$.

        \item \textbf{Processing $\Pi_2$:} Next, we process the states in $\Pi_2$: $b$ and $c$.
        \begin{itemize}
            \item State $b$: $\delta(b, 0) = d \in C$ and $\delta(b, 1) = e \in D$.
            \item State $c$: $\delta(c, 0) = f \in C$ and $\delta(c, 1) = g \in D$.
        \end{itemize}
        Both states have transitions to class $C$ on symbol $0$ and to class $D$ on symbol $1$. Therefore, $b$ and $c$ are equivalent. We merge them into a new class, $B = \{b, c\}$.

        \item \textbf{Processing $\Pi_3$:} Finally, we process $\Pi_3$, which contains only state $a$. There is nothing to compare it with, so it forms its class, $A = \{a\}$.
    \end{enumerate}

    After applying the algorithm, we obtain the minimized ADFA represented in \cref{fig:example_minimized_ADFA}. Each node of the original ADFA is represented by a node in the minimized ADFA (equivalence classes). The edges represent transitions between these nodes. The root node $A$ is the initial state of the minimized ADFA, while the node $D$ is the final state.
    
    \begin{figure}[H]
        \centering
        \begin{tikzpicture}[->, >=stealth, node distance=3cm, on grid, auto]
            \node[state, initial, initial text=] (A) {A};
            \node[state] (B) [right=of A] {B};
            \node[state] (C) [right=of B] {C};
            \node[state, accepting] (D) [right=of C] {D};
        
            \path (A) edge [bend left] node {0} (B)
                    edge [bend right] node[below] {1} (B)
                (B) edge [bend left] node {0} (C)
                    edge [bend right] node[below] {1} (D)
                (C) edge [bend left] node {0} (D)
                    edge [bend right] node[below] {1} (D);
        \end{tikzpicture}
        \caption{Minimized ADFA}
        \label{fig:example_minimized_ADFA}
    \end{figure}      

    The equivalence classes of the nodes are listed in \cref{tab:equivalence_classes}.
    
    \begin{table}[H]
        \centering
        \begin{tabular}{|c|l|}
            \hline
            \textbf{Class} & \textbf{States} \\
            \hline
            A & $a$ \\
            B & $b, c$ \\
            C & $d, f$ \\
            D & $e, g, h, i, l, m$ \\
            \hline
        \end{tabular}
        \caption{Equivalence classes of the nodes}
        \label{tab:equivalence_classes}
    \end{table}
\end{example}
\chapter{Labeled Trees} \label{chp:thbg_labeled_tree}
\section{Introduction and Motivation}
Before delving into specific compression techniques, it is essential to establish a solid theoretical foundation regarding labeled trees. These structures are fundamental for representing hierarchical data across diverse fields, from bioinformatics to document processing. This chapter provides the necessary background, defining labeled trees, exploring their common applications, and introducing the core concepts behind their compression and indexing. Understanding these principles, including the role of succinct data structures and the information-theoretic limits of compression, is crucial for appreciating the challenges and advancements in handling large-scale tree-structured data effectively, which forms the basis for the work presented in this thesis.

\begin{definition}
    A \textbf{labeled tree} is a rooted, ordered, hierarchical data structure in which every node is assigned a label from a predefined alphabet $\Sigma$. The structure consists of nodes connected by edges, forming a directed acyclic graph. Formally, a labeled tree $T$ with $t$ nodes can be defined as $T = (V, E, \ell)$, where:
    \begin{itemize}
        \item $V$ is the set of nodes.
        \item $E \subseteq V \times V$ is the set of directed edges.
        \item $\ell: V \to \Sigma$ is a labeling function that assigns a label $\ell(u) \in \Sigma$ to each node $u$.
    \end{itemize}
\end{definition}


In the case of ordered labeled trees, the children of a node are ordered, meaning their positions relative to each other matter. A labeled tree can have arbitrary degree and shape, and the alphabet $\Sigma$ used for labels can be of arbitrary size.

\section{Applications}
Labeled trees are widely used in computer science and data representation due to their hierarchical structure and flexibility in modeling relationships. Prominent applications include:
\begin{enumerate}
    \item \textbf{XML Data Representation:} XML documents are often modeled as labeled trees, where each element is a node labeled by its tag, and hierarchical nesting represents parent-child relationships.
    \item \textbf{JSON Data Representation:} JSON documents can be viewed as labeled trees, with keys as labels and values as children.
    \item \textbf{Bioinformatics:} Labeled trees are used to represent phylogenetic trees, genome annotations, and hierarchical clustering.
    \item \textbf{Compiler Design:} Abstract Syntax Trees (ASTs) for programming languages are labeled trees that capture the structure of code.
    \item \textbf{File Systems:} The directory structure of file systems can be viewed as a labeled tree.
\end{enumerate}

Efficient representation, navigation, and querying of labeled trees are essential for many applications, motivating the development of specialized data structures and algorithms. 

\section{Compression and Indexing} \label{compandindexinglabtree}
The goal of compressing and indexing labeled trees is to design a compressed storage scheme for a labeled tree $T$ with $t$ nodes that allows for efficient navigation operations in $T$, as well as fast search and retrieval of subtrees or paths within $T$. To be effective, the compressed representation should minimize the space required to store the tree while supporting a wide range of operations in (near-)optimal time.

Let $u$ be a node in the labeled tree $T$ and let $c \in \Sigma$. We define the following navigation operations on $T$:
\begin{itemize}
    \item \textbf{Navigational queries:} ask for the parent of $u$, the $i$-th child of $u$, or the label of $u$. The last two operations might be restricted to the children of $u$ with a specific label $c$.
    \item \textbf{Path queries:} retrieve the nodes in the subtree rooted at $u$ (any possible order should be implemented).
    \item \textbf{Subpath queries:} ask for the (number of occurrences of) nodes of $T$ that descend from a labeled subpath $P$, which may be anchored anywhere in the tree (i.e., not necessarily in its root). 
\end{itemize}

A naive solution to index labeled trees is to store the tree as a list of nodes with their labels and parent-child relationships using pointers in $O(t \log t)$. However, this representation is not space-efficient and does not support fast navigation or query operations. 

Many data structures have been proposed to compress and index labeled trees, each with its trade-offs in terms of space usage, query performance, and supported operations. One of the most successful approaches is the Extended Burrows-Wheeler Transform, which extends the classical Burrows-Wheeler Transform (BWT) to handle labeled trees efficiently (\cref{sec:background}).

\begin{comment}
    \section{Succinct Data Structures for Trees}
    \alessio{Questa sezione non è ben collegata al resto. Serve questa parentesi sulle SDS? Dato che anche prima parli di minimizzare lo spazio e time optimality, potresti parlare direttamente dei lower bound. Il filo logico sarebbe: "vogliamo fare le cose il meglio possibile, ma quanto vale il meglio?"}
    In order to compress the index of labeled trees, we need to avoid the use of pointers and store the tree in a space-efficient manner. Succinct data structures are a class of compressed data structures that support efficient navigation and query operations on the compressed data. These structures are designed to use close to the information-theoretic lower bound on space while providing fast access to the original data. They were first introduced by Jacobson \cite{jacobson1989space} and have been applied to various problems in string processing, graph theory, and data compression.
\end{comment}

\subsection{Information-Theoretic Lower Bound}
The information-theoretic lower bound for storing an unlabeled tree with $t$ nodes is given by:

\begin{itemize}
    \item The number of binary unlabeled trees with $t$ nodes is given by the Catalan number $C_t = \frac{1}{t+1} \binom{2t}{t}$ that can can be approximated as $C_t \approx \frac{4^t}{t^{3/2}\sqrt{\pi}}$ using Stirling's approximation.
    \item The entropy (or the information-theoretic minimum number of bits to encode the structure of the tree) is the logarithm (base 2) of the total number of trees, which is $-\log_2 C_t \approx 2t - \frac{1}{2} \log_2 \pi t^3$.
    \item The correction term $\frac{1}{2} \log_2 \pi t^3$ grows slower that the linear term $2t$, we can say that $-\frac{1}{2} \log_2 \pi t^3 = -\Theta(\log t)$.
    \item The information-theoretic lower bound for storing an unlabeled tree with $t$ nodes is $2t - \Theta(\log t)$ bits.
\end{itemize}

Then, for labeled trees, the labels assigned to each node must be stored, which requires an additional space:

\begin{itemize}
    \item Let $\Sigma$ denote the alphabet of labels, and let $|\Sigma|$ be the size of the alphabet.
    \item Each node in the tree requires $\log_2 |\Sigma|$ bits to store its label.
    \item Therefore, for $t$ nodes, the total space required to store the labels is $t \log_2 |\Sigma|$ bits.
\end{itemize}

Combining the structural representation and the labeling, the information-theoretic lower bound for storing a labeled tree is:

$$
    2t - \Theta(\log t) + t \log_2 |\Sigma| \text{ bits}
$$
\chapter{Extended Burrows-Wheeler Transform}
\section{Introduction} 

The Extended Burrows-Wheeler Transform (XBWT) is a data structure designed to efficiently compress and index labeled trees. A labeled tree is a data structure where each node is assigned a label from a given alphabet, and the tree can have an arbitrary shape and degree.

XBWT works by linearizing a labeled tree into two coordinated arrays: one capturing the structural properties of the tree and the other storing its labels. This transformation allows for efficient representation, navigation, and querying of the tree. The key advantage of XBWT lies in its ability to compress labeled trees while supporting a wide range of operations, such as parent-child navigation and sophisticated path-based searches, in (near-)optimal time and space.

One of the primary applications of XBWT is in the compression and indexing of hierarchical data formats, such as XML documents. It provides significant improvements in both compression ratio and query performance compared to traditional tools, making it an invaluable resource for data-intensive applications in fields like bioinformatics, information retrieval, and big data analytics.

This project aims to implement the XBWT data structure and explore its applications in the context of labeled trees. We will start by providing an overview of the theoretical foundations of the XBWT. Finally, we will describe and compare the algorithms for constructing the XBWT and demonstrate its use in compressing and indexing labeled trees.

Let's start with a quick overview of the XBWT and its theoretical foundations.


\subsubsection{How XBWT Works}
The transformation process of XBWT is as follows:
\begin{enumerate}
    \item \textbf{Path Sorting:} The labeled tree is linearized by sorting its nodes based on the \emph{paths} from each node's parent to the root. The resulting order groups nodes with similar upward paths together, clustering related labels.
    \item \textbf{Array Construction:} Two arrays, \( S_{\text{last}} \) and \( S_{\alpha} \), are generated:
    \begin{itemize}
        \item \( S_{\text{last}} \) stores structural information, such as whether a node is the last child of its parent. This encodes the tree’s structure without the need for explicit pointers.
        \item \( S_{\alpha} \) stores the labels of the nodes in the sorted order determined by their upward-path sorting.
    \end{itemize}
    \item \textbf{Compression:} Both \( S_{\text{last}} \) and \( S_{\alpha} \) are highly compressible due to the clustering of similar labels and structural redundancy.
\end{enumerate}

\subsubsection{Key Properties of XBWT}
The XBWT has several key properties that make it an effective tool for labeled tree compression and indexing:
\begin{itemize}
    \item \textbf{Succinctness:} The XBWT representation of a labeled tree uses space close to the \emph{information-theoretic lower bound}, which is \( 2t - \Theta(\log t) + t \log |\Sigma| \) bits for a tree with $t$ nodes and an alphabet of size $|\Sigma|$.
    \item \textbf{Efficient Querying:} XBWT supports a range of navigational operations, such as finding the parent, child, or subtree of a node in near-optimal time.
    \item \textbf{Scalability:} XBWT is particularly useful for large-scale hierarchical data, such as XML documents or phylogenetic trees, where both compression and fast querying are critical.
\end{itemize}

\subsection{Project implementation}

The XBWT data structure will be implemented in C++ using the Succinct Data Structure Library 2.0 (SDSL) for efficient representation and manipulation of compressed data structures. We will develop two algorithms for constructing the XBWT: one efficient linear-time recursive algorithm and one more straightforward iterative algorithm. Also, we will implement the necessary data structures and algorithms for navigating and querying the XBWT, such as parent-child navigation and path-based searches. 

The code is available on GitHub at \url{https://github.com/davide-tonetto-884585/XBWT}. The project will be structured as follows:

\begin{itemize}
    \item \textbf{XBWT.hpp}: File containing the class definition and implementation for the generic XBWT data structure.
    \item \textbf{LabeledTree.hpp}: File containing the class definition and implementation for the generic labeled tree data structure used to feed and test the XBWT.
    \item \textbf{main.cpp}: Main file containing the test cases and examples for the XBWT implementation.
    \item \textbf{experiments.cpp}: File containing the experiments and performance evaluation of the XBWT construction algorithms and compression efficiency.
    \item \textbf{CMakeLists.txt}: CMake configuration file for building the project.
\end{itemize}

The project will be developed and tested on a Linux environment using the GCC compiler and the CMake build system. 

\subsection{Definition}
The \textbf{Extended Burrows-Wheeler Transform} is a data structure designed to efficiently compress and index \emph{ordered node-labeled trees}. Inspired by the classical Burrows-Wheeler Transform (BWT)~\cite{burrows1994block} for strings, the XBWT extends its principles to hierarchical structures, enabling efficient storage, navigation, and querying of trees. %It is particularly effective for trees where each node has a label drawn from an alphabet $\Sigma$ and the tree structure has an arbitrary shape and degree.

\begin{definition}[XBWT basic notation]
    \label{def:node_informations}
    Let $T$ be a totally ordered node-labeled tree of arbitrary fan-out, depth, and shape, with $n$ internal nodes and $l$ leaves ($n + l = t$ nodes in total) and alphabet $\Sigma$. Given a node $u \in T$, we define the following information:
    \begin{itemize}
        \item $last(u)$: a binary value that is 1 if $u$ is the last (rightmost) child of its parent in the total order, and 0 otherwise.
        \item $\alpha(u)$: the label of node $u$ concatenated with one bit that is 1 if $u$ is a leaf and 0 otherwise.
        \item $\pi(u)$: the string obtained by concatenating the labels of the nodes on the upward path from $u$ parent to the root of $T$. Formally, given the parent node $u'$ of $u$, we can define $\pi(u)$ as follows:
        \begin{align*}
            \pi(u) = \begin{cases}
                \varepsilon & \text{\emph{if $u$ is the root}} \\
                label(u') \circ \pi(u') & \text{\emph{otherwise}}
            \end{cases}
        \end{align*}
        where $\circ$ is the concatenation operator.
    \end{itemize}
\end{definition}
Let $\Sigma_N$ be the alphabet of the internal nodes, and $\Sigma_L$ be the alphabet of the leaves of $T$.
If $\Sigma_L \cap \Sigma_N = \emptyset$, the additional indicator bit in $\alpha(u)$ becomes redundant and may be omitted.
Without loss of generality, we will only consider this case; therefore, we identify $\alpha(u)$ with the node label, i.e., $\alpha(u) := label(u)$.

The definition of the XBWT relies on a sequence $S$, which contains a triplet \\ 
$(last(u), \alpha(u), \pi(u))$ for each node $u$ in the tree $T$.

The construction of $S$ is a two-step process. First, an intermediate vector of triplets is created by traversing the tree $T$ in pre-order and generating a triplet $(last(u), \alpha(u), \pi(u))$ for each node. Then, $S$ is stably sorted according to the lexicographical order of the $\pi$ component of each triplet (see \cref{tab:xbwt_example} for an example).
From now on, we will use $S_{last}$ (resp. $S_{\alpha}$, $S_{\pi}$) to refer to the sequences made up of the $last$ (resp. $\alpha$, $\pi$) component of all triplets in $S$.

\begin{theorem}
    The XBWT of a labeled tree $T$ consists of two arrays, $S_{\text{last}}$ and $S_{\alpha}$, after sorting.
    %These are constructed from the sequence $S$ of triplets. Specifically, for each $i$ from 1 to $t$, $S_{\text{last}}[i]$ is the `$last$` component of the $i$-th triplet in $S$, and $S_{\alpha}[i]$ is the '$\alpha$' component.
    The total space required is $2t + t \log |\Sigma|$ bits.
\end{theorem}

Therefore, $S_{\pi}$ is not needed after the construction of the XBWT. However, in the following discussion, we will still refer to it as it possesses some important properties. See \cref{fig:example_tree} for an example of the tree $T$ and \cref{tab:xbwt_example} for its corresponding sequence $S$.

\begin{figure}[ht]
    \centering
    \begin{tikzpicture}[
        % Stili globali per l'albero
        level distance=1.5cm,      % Distanza verticale tra i livelli
        every node/.style={font=\sffamily}, % Usa un font sans-serif come nell'immagine
        % Stili specifici per ogni livello
        level 1/.style={sibling distance=4cm}, % Distanza tra i figli di A
        level 2/.style={sibling distance=1.5cm}, % Distanza tra i figli di B e C
        level 3/.style={level distance=1cm},     % Distanza verticale per i nodi foglia
        level 4/.style={level distance=1cm}      % Distanza verticale per i nodi foglia più bassi
    ]

    \node {A}
        child {
            node (B1) {B}
            child { node {D}
                child { node {a} }
            }
            child { node {a} }
            child { node {E}
                child { node {b} }
            }
        }
        child {
            node {C}
            child { node {D}
                child { node {c} }
            }
            child { node {b} }
            child { node {D}
                child { node {c} }
            }
        }
        child {
            node (B2) {B}
            child { node {D}
                child { node {b} }
            }
        };

    \node (label_u) [above left=0.2cm and 0.4cm of B1, font=\rmfamily] {Node $u$};
    \draw[->] (label_u) -- (B1);

    \node (label_v) [above right=0.2cm and 0.4cm of B2, font=\rmfamily] {Node $v$};
    \draw[->] (label_v) -- (B2);
\end{tikzpicture}
    
    \caption{A labeled tree with $\Sigma_N = \{A, B, C, D, E\}$ and $\Sigma_L = \{a, b, c\}$. Notice that $\alpha(u) = \alpha(v) = B$ and $\pi(u) = \pi(v) = A$.}
    \label{fig:example_tree}
\end{figure}

\begin{figure}[ht]
    \centering
    \begin{tabular}{r c c l}
    \hline\hline
    \textbf{} & \textbf{$S_{\text{last}}$} & \textbf{$S_{\alpha}$} & \textbf{$S_{\pi}$} \\
    \hline
    \textbf{1} & 0 & A & $\epsilon$ \\
    \textbf{2} & 0 & B & A \\
    \textbf{3} & 0 & C & A \\
    \textbf{4} & 1 & B & A \\
    \textbf{5} & 0 & D & BA \\
    \textbf{6} & 0 & a & BA \\
    \textbf{7} & 1 & E & BA \\
    \textbf{8} & 1 & D & BA \\
    \textbf{9} & 0 & D & CA \\
    \textbf{10} & 0 & b & CA \\
    \textbf{11} & 1 & D & CA \\
    \textbf{12} & 1 & a & DBA \\
    \textbf{13} & 1 & b & DBA \\
    \textbf{14} & 1 & c & DCA \\
    \textbf{15} & 1 & c & DCA \\
    \textbf{16} & 1 & b & EBA \\
    \hline\hline
    \end{tabular}
    \caption{Decomposition of $S$ for the tree shown in \cref{fig:example_tree}, obtained by stably sorting triplets according to $S_\pi$. In this representation, nodes $u$ and $v$ from the original tree $T$ appear at indices $2$ and $4$, respectively. The children's block of node $u$ occupies positions $5$ through $7$, while node $v$'s single child is located at index $8$.}
    \label{tab:xbwt_example}
\end{figure}

\subsection{Properties}
The XBWT's effectiveness as an indexing structure stems from a key property of the sequence $S$, which we call ``Grouping by Parent''. This property, along with its consequences, arises directly from the transform's definition and the sorting process.

\subsubsection{Key Property: Grouping by Parent}

The fundamental property of the XBWT is that the children of any node $u$ in the tree $T$ form a contiguous block in the sequence $S$. Let $u'_1, \dots, u'_z$ be the children of node $u$ in their original order. Their corresponding triplets will appear consecutively in $S$ in that same order.
\begin{example}
    Consider the node $u$ in \cref{fig:example_tree}. Looking at \cref{tab:xbwt_example}, we can see that its children form a contiguous block in positions $[5, 6, 7]$ of the sequence $S$.
\end{example}

This grouping provides several important consequent properties:

\textbf{Unary Degree Encoding:} The subarray $S_{\text{last}}$ for the block of children $[u_1, \dots, u_z]$ encodes the degree of $u$ in unary coding. Specifically, $S_{\text{last}}[u'_z] = 1$ and $S_{\text{last}}[u'_i] = 0$ for $1 \leq i < z$. 
\begin{example}
    Consider the node $u$ in \cref{fig:example_tree}. Looking at \cref{tab:xbwt_example}, we can observe that $S_{\text{last}}[5 \dots 7] = [0, 0, 1]$, which is equal to 3 in unary encoding, matching the number of children of node $u$.
\end{example}

\textbf{Preservation of Sibling Order:} If two nodes $u$ and $v$ have the same label, and the triplet for $u$ precedes the triplet for $v$ in $S$, then the entire block of children of $u$ will also precede the block of children of $v$.
\begin{example}
    Consider nodes $u$ and $v$ in \cref{fig:example_tree}. In \cref{tab:xbwt_example}, node $u$ appears at index 2 while node $v$ appears at index 4 in the sequence $S$. Following the preservation of sibling order property, all children of $u$ (occupying positions $[5, 6, 7]$) appear before the child of $v$ (at position $8$).
\end{example}

\textbf{Path-based Indexing:} This property extends to entire paths. For any label $c \in \Sigma$, all triplets whose $\pi$-components are prefixed by $c$ form a contiguous block in $S$. If $u$ is the $i$-th node with label $c$ in $S_{\alpha}$, its children's block is located within the larger block of all nodes with paths prefixed by $c$. This block is delimited by the $(i-1)$-th and $i$-th 1s in the corresponding section of $S_{\text{last}}$. \label{prop3}
\begin{example}
    Let's examine nodes $u$ and $v$ in \cref{fig:example_tree}, both labeled `B'. In the sequence $S$ shown in \cref{tab:xbwt_example}, $u$ is the first node with label 'B' (at index 2), and $v$ is the second (at index 4).

    The children of all nodes labeled `B' form a contiguous block in $S$. In this case, the children of both $u$ and $v$ are located in the range $[5 \dots 8]$. We can distinguish between the children of $u$ and the children of $v$ using the $S_{\text{last}}$ array:
    \begin{itemize}
        \item The block of children for $u$ (the first `B' node) starts at the beginning of the range (index 5) and ends at the position of the first 1 (index 7) in $S_{\text{last}}[5 \dots 8]$.
        \item The block of children for $v$ (the second `B' node) starts after the first 1 (index 8) and ends at the position of the second 1 (index 8) in $S_{\text{last}}[5 \dots 8]$.
    \end{itemize}
\end{example}

\subsubsection{Other Properties}
Additional properties of the XBWT components include:

\begin{itemize}
    \item The first triplet in $S$ always corresponds to the root of the tree $T$.
    \item $S_{\text{last}}$ contains exactly $n$ 1s (one for each internal node) and exactly $l$ 0s (one for each leaf).
    \item $S_{\alpha}$ is a permutation of the node labels in $T$.
\end{itemize}

\subsection{Construction}\label{sec:pathSort}
A naive approach to build the XBWT would be to explicitly construct $S$ through the concretization of $\pi$-strings and then sort it using a stable sorting algorithm. However, this approach would require $\Theta(t^2)$ space in the worst case, which is not feasible for deep trees. To overcome this issue, Ferragina et al.~\cite{ferragina2009compressing} proposed a more efficient algorithm that builds $S$ in linear time $O(t)$ and uses $O(t \log t)$ bits of working space.
This algorithm is called {\pathsort}, and is based on a generalization of the Skew algorithm, designed for the construction of suffix arrays on strings~\cite{karkkainen2006linear}.

The Skew algorithm works by first building the suffix array recursively on a string of length two-thirds of the original one.
This is done by working on the suffixes starting at positions $i$ such that $i \bmod 3 \neq 0$. 
Then, the suffix array of the remaining suffixes is built using the result of the previous step, and in the end, these two are merged into the final suffix array.
Thanks to linear-time radix sorting of the suffix array, a single-pass merge algorithm, and given that the recursion operates on a string of length $2n/3$, the overall time satisfies
\[
T(n) \;=\; T(2n/3) + O(n) \;=\; O(n).
\]

\begin{comment}
\begin{figure}
    \centering
    \includegraphics[width=1\textwidth]{Immagini/XBWT_example.png}
    \caption[XBWT example]{(a) A labeled tree $T$ where $\Sigma_N = \{A, B, C, D, E\}$ and $\Sigma_L = \{a, b, c\}$. Notice that $\alpha[u] = \alpha[v] = B$ and $\pi[u] = \pi[v] = A$. (b) The multi-set $S$ is obtained after the pre-order visit of $T$. (c) The final multi-set $S$ after the stable sort based on the $\pi$'s component of its triplets.}
    \label{fig:XBWT_example}
\end{figure}
\end{comment}


\begin{comment}
\alessio{Maybe all of this detail is not required for this algorithm... This is just a precursor of the {\pathsort}, which you explain in detail just after this subsubsection. Here, I think it's enough to give an intuition of the mod 3 stuff, and explain directly the {\pathsort}. Otherwise it can be confusing as you were talking about compressing trees and you jump to strings. I won't delete what you wrote because there are some comments that are still interesting (old part between the horizontal lines).}

\par\noindent\rule{\textwidth}{0.4pt}
Let us see briefly how the Skew algorithm works first.

\subsubsection{Skew Algorithm}
The Skew algorithm is an efficient method for constructing the suffix array of a string in linear time. A suffix array is a data structure that lists the starting indices of all the suffixes of a string in lexicographical order, and it is widely used in various string processing algorithms.

\paragraph{Step 0: Construct a Sample.}
For $k \in \{0,1,2\}$, define the index sets
\[
B_k = \{ i \in [0, n] \mid i \bmod 3 = k \}. 
\]
\alessio{If 0 and n are included, then it should be [1, n], otherwise you are counting n+1 elements. Moreover, in the XBWT table, the index starts at 1. Just checked the paper, they also consider the end character but don't represent it. I think that starting at 1 is ok then.}
Let $C = B_1 \cup B_2$ be the set of sample positions, and let $S_C$ denote the set of sample suffixes. \alessio{What does sample mean? Suffixes of what? What is the formal definition of $S_C$?}

\paragraph{Step 1: Sort Sample Suffixes.}
For both $k = 1$ and $k = 2$, construct the string $R_k$ whose characters are the triples $[t_i, t_{i+1}, t_{i+2}]$ for $i \in B_k$, in increasing order. \alessio{What are the $t_i$s? Also, the original paper uses a different notation with $k$ as subscript that is slightly clearer. Show how $R$ is made $[ttt][ttt]...$}
The last character of $R_k$ is unique because we pad it with sentinels so that $t_{n+1} = t_{n+2} = 0$. Let $R = R_1 R_2$ be the concatenation of $R_1$ and $R_2$.
Then the non-empty suffixes of $R$ correspond to the sample suffixes in $S_C$ in an order-preserving way: sorting the suffixes of $R$ yields the order of $S_C$.
To sort the suffixes of $R$, first radix sort the characters of $R$ (the triples) and rename them by their ranks to obtain a string $R'$. If all characters are distinct, their order directly gives the order of suffixes. Otherwise, sort the suffixes of $R'$ recursively using the same DC3 procedure.
Once $S_C$ is sorted, assign ranks to sample suffixes: for $i \in C$, let $\mathrm{rank}(S_i)$ be the rank of $S_i$ within $S_C$. Additionally, define $\mathrm{rank}(S_{n+1}) = \mathrm{rank}(S_{n+2}) = 0$. For $i \in B_0$, $\mathrm{rank}(S_i)$ is undefined.
\alessio{This paragraph is very complicated. The previous version (which Nicola didn't comment much on) was much clearer, despite being less formal. In particular: you don't define what sample means, what is $t_i$ in the triplets, what is $S_C$ and how it correlates to $R$, how the renaming works, what are the ranks of the triplets, what is the DC3 procedure. I can see that you copied the explanation of the original paper to be consistent with their notation, but you should be consistent with \textbf{your} notation. For example: you defined strings with $S = ...s_i...$ but here you use $t$. But $T$ in your thesis is a tree and you want to build a tree, so it seems like you are creating triplets of nodes of the tree.}
\alessio{Also, an example would help a lot (you can copy the one of the original paper if you want).}

\paragraph{Step 2: Sort Non-sample Suffixes.}
Represent each non-sample suffix $S_i$ with $i \in B_0$ using the pair
\[
\bigl(t_i,\ \mathrm{rank}(S_{i+1})\bigr).
\]
Note that $\mathrm{rank}(S_{i+1})$ is always defined for $i \in B_0$ by the previous step. Then, radix sort these pairs to obtain the order of $S_{B_0}$.

\paragraph{Step 3: Merge.}
Merge the two sorted sets $S_C$ and $S_{B_0}$ using a standard comparison-based merging. To compare a suffix $S_i \in S_C$ against a suffix $S_j \in S_{B_0}$, distinguish two cases:
\[
\begin{aligned}
&\text{if } i \in B_1: && S_i \le S_j \iff \bigl(t_i,\ \mathrm{rank}(S_{i+1})\bigr) \le_{\mathrm{lex}} \bigl(t_j,\ \mathrm{rank}(S_{j+1})\bigr),\\
&\text{if } i \in B_2: && S_i \le S_j \iff \bigl(t_i,\ t_{i+1},\ \mathrm{rank}(S_{i+2})\bigr) \le_{\mathrm{lex}} \bigl(t_j,\ t_{j+1},\ \mathrm{rank}(S_{j+2})\bigr).
\end{aligned}
\]
The ranks used above are defined in all cases by Step~1. Each comparison inspects $O(1)$ characters/ranks, and the two-way merge advances each pointer at most once, so the merge runs in linear time overall.

\paragraph{Time Complexity.}
Excluding the recursive call, all steps are linear-time via radix sorting and a single-pass merge. The recursion operates on a string of length $2n/3$, so the overall time satisfies
\[
T(n) \;=\; T(2n/3) + O(n) \;=\; O(n).
\]

\par\noindent\rule{\textwidth}{0.4pt}
\end{comment}

% \subsubsection{PathSort Algorithm} \label{sec:pathSort}

% The pseudocode of the pathSort algorithm is shown in \cref{alg:pathSort}. 
The {\pathsort} algorithm works similarly, adapted to work on labeled trees.
% This algorithm is based on the Skew algorithm, but it is adapted to work on labeled trees.
\alessio{What is an upward subpath? Is it the $\pi$ component of the node or is it something with the triplets? If it is the first case, use $\pi(u)$, otherwise define upward subpath. Reading the Ferragina et al paper it seems $\pi$ (or at least, some prefix), so i will edit it. The original version is in the old stuff folder.}
Given a value $j\in\{0,1,2\}$, the main idea is to recursively sort the $\pi$ component of the nodes in levels $\not\equiv j \pmod{3}$, then sort the $\pi$ component of the nodes in levels $\equiv j \pmod{3}$ using the result of the previous step, and finally merge the resulting two sets. %by exploiting their lexicographic names.
At each step $i$ of the recursion, the algorithm works on a labeled tree $T_i$, which is a \emph{contracted} (or shrunk) version of the original tree $T$. 
We set $T_0 = T$, and note that the structure of $T_{i+1}$ is derived from that one of $T_i$.
The inner workings of the recursive step will be explained in detail later.

The parameter $j$ is chosen in such a way that the number of nodes of the shrunk tree whose level is $\equiv j \pmod{3}$ is at least $t/3$, so that a constant fraction of upward paths $\pi$ is ensured to be dropped at each recursive step.
It is important to note that:
\begin{enumerate}
    \item the height of the new (contracted) tree shrinks by a factor of three, hence the node naming requires the radix sort over triples of names; 
    \item given the choice of $j$, the number of nodes of the new (contracted) tree will be at most $2t/3$, thus ensuring that the running time of the algorithm satisfies the recurrence $R(t) = R(2t/3) + \Theta(t) = \Theta(t)$; 
    \item following an argument similar to~\cite{karkkainen2006linear}, the names of the dropped subpaths can be computed in $O(t)$ time from the names of the non-dropped subpaths, by radix sorting.
\end{enumerate}
The pseudocode of the {\pathsort} algorithm is shown in \cref{alg:pathSort}.

\begin{algorithm}
    \caption{{\pathsort}($T$)}
    \label{alg:pathSort}
    \begin{algorithmic}[1]
    \State Initialize the array of triplets \texttt{IntNodes}[1 \dots $t$].
    \State Visit the internal nodes of $T$ in pre-order. For the $i$-th visited internal node $u$, set $\texttt{IntNodes}[i] = \bigl(\alpha(u),\, \text{level}(u),\, \text{parent}(u)\bigr)$.
    \State Let $j \in \{0, 1, 2\}$ be such that the number of nodes in \texttt{IntNodes} whose level is $\equiv j \pmod{3}$ is at least $t/3$. Sort recursively the upward subpaths starting at nodes in levels $\not\equiv j \pmod{3}$.
    \State Sort the upward subpaths starting at nodes in levels $\equiv j \pmod{3}$ using the result of Step 3.
    \State Merge the two sets of sorted subpaths by exploiting their lexicographic names.
    \end{algorithmic}
\end{algorithm}

\subsubsection{Recursive Step of PathSort}
At each recursive step, the algorithm constructs the array \texttt{IntNodes}, which stores the triplets $(\alpha(u), \text{level}(u), \text{parent}(u))$ for every internal node $u$ in the given tree $T$.  

Next, the algorithm selects a value $j$ such that the number of nodes in \texttt{IntNodes} with depth $\equiv j \pmod{3}$ is at least $t/3$. Based on this choice, two separate arrays are created:  
\begin{itemize}
    \item \texttt{IntNodesAtPosJ}, containing nodes at levels $\equiv j \pmod{3}$,
    \item \texttt{IntNodesNotAtPosJ}, containing nodes at levels $\not\equiv j \pmod{3}$
\end{itemize}

For each node $u$ in \texttt{IntNodesNotAtPosJ}, the algorithm extracts the upward path consisting of the first three ancestors of $u$, $\pi(u)[1 \dots 3]$. These paths are then sorted using radix sort. If all upward paths are unique, the nodes in \texttt{IntNodesAtPosJ} are sorted and subsequently merged with \texttt{IntNodesNotAtPosJ} using lexicographic ordering.
%, following the same merging strategy as in the Skew algorithm. 
Otherwise, the algorithm recursively calls the {\pathsort} function on a new contracted tree, where nodes are renamed according to their sorted paths.

\subsection{Inversion}
The ability to invert the XBWT is fundamental to its utility as a compression technique. Invertibility guarantees that the original tree can be perfectly reconstructed from its transformed representation ($S_{\text{last}}$ and $S_{\alpha}$). This ensures that the compression is lossless, meaning that no information is lost during the process, which is a critical requirement for most applications.

The property `Path-based Indexing' (\cref{prop3}) ensures that the two arrays $S_{\text{last}}$ and $S_{\alpha}$ of the XBWT can be used to reconstruct the original tree $T$. The algorithm for inverting the XBWT is linear in time and requires $O(t \log t)$ bits of space.

\cref{alg:rebuildTree} operates in three main steps. First, it constructs two auxiliary arrays, $F$ and $J$, which are crucial for navigating the tree structure within the compressed format.

\begin{itemize}
    \item \textbf{The $F$ array:} This array maps each character $c \in \Sigma$ to the index of the first occurrence in $S$ of a triplet whose $\pi$-component is prefixed by $c$. It essentially marks the starting points of blocks of nodes that share the same initial path label.
    \item \textbf{The $J$ array:} For each entry $i$ in $S$, $J[i]$ stores the index in $S$ corresponding to the first child of the node represented by $S[i]$. If $S[i]$ represents a leaf, $J[i]$ is set to a sentinel value (e.g., -1).
\end{itemize}

\begin{example}[$F$ and $J$ arrays]
    Considering the XBWT in \cref{tab:xbwt_example}, the $F$ array would map 'A' to index 2 (for node $r$), 'B' to index 5 (for the children of nodes with label 'B'), and so on. For the $J$ array, let's take the node $u$ at index 2 in $S$. Its first child is at index 5. Therefore, $J[2]$ would be 5.
\end{example}

Finally, the algorithm employs the array $J$ to
simulate a depth-first visit of $T$, creates its labeled nodes, and properly connects them to their parents. 

% \alessio{Why BuildF and BuildJ take in input XBWT[T] but do not use it? The function should use the parameters that you pass it, pass more specific variables (as below). Also, I removed XBWT[T] notation since I really don't like it :).}
\begin{algorithm}[H]
    \caption{\textsc{RebuildTree}($S_\alpha$, $S_\textup{last}$)}
    \label{alg:rebuildTree}
    \begin{algorithmic}[1]
    \State $F = $ \textsc{BuildF}($S_\alpha$, $\Slast$)
    \State $J = $ \textsc{BuildJ}($S_\alpha$, $\Slast$, $F$)
    \State Create node $r$ and set $Q = \{\langle1, r\rangle\}$; \Comment{$Q$ is a stack}
    \While{$Q \neq \emptyset$} \Comment{We still have nodes to create in $T$}
        \State $\langle i, u \rangle = $ pop($Q$);
        \State $j = J[i]$; \Comment{Take the block of $u$'s children in $S$}
        \If{$j = -1$} \Comment{$u$ is a leaf of $T$}
            \State \textbf{continue};
        \EndIf
        \State Find first $j' \geq j$ such that $S_{\text{last}}[j'] = 1$; \Comment{Range $[j, j']$ are the children of $u$ in $T$}
        \For{$h = j'$ downto $j$} 
            \State Create the node $v$ labeled $S_\alpha[h]$;
            \State Attach $v$ as first child of $u$;
            \State push($\langle h, v \rangle$, $Q$);
        \EndFor
    \EndWhile
    \State \Return node $r$.
    \end{algorithmic}
\end{algorithm}

\begin{algorithm}[H]
    \caption{\textsc{BuildF}($S_\alpha$, $\Slast$)}
    \label{alg:buildF}
    \begin{algorithmic}[1]
    \State $C[1,|\Sigma_n|] = \{0, 0, \dots, 0\}$;
    \State $F[1,t] = \{0, 0, \dots, 0\}$;
    \For{$i = 1, \ldots, t$}
        \State $C[S_\alpha[i]] = C[S_\alpha[i]] + 1$; \Comment{Count the occurrences of node labels}
    \EndFor
    \State $F[1] = 2$; \Comment{$S_\pi[1]$ is the empty string}
    \For{$i \in \{1, \ldots, |\Sigma_N|-1\}$} \Comment{Consider just the internal-node labels}
        \State $s = 0$; $j = F[i]$;
        \While{$s \neq C[i]$} \Comment{Not all blocks of children have been passed}
            \State $j = j + 1$;
            \If{$S_{\text{last}}[j] = 1$} \Comment{One further block of children has passed}
                \State $s = s + 1$;
            \EndIf
        \EndWhile
        \State $F[i+1] = j$;
    \EndFor
    \State \Return $F$.
    \end{algorithmic}
\end{algorithm}
    
\begin{algorithm}[H]
    \caption{\textsc{BuildJ}($S_\alpha$, $\Slast$, $F$)}
    \begin{algorithmic}[1]
    \State $J[1,t]=\{0, 0, \dots, 0\}$;
    \For{$i = 1, \ldots, t$}
        \If{$S_\alpha[i] \in \Sigma_L$}
            \State $J[i] = -1$; \Comment{$S_\alpha[i]$ is a leaf label}
        \Else
            \State $J[i] = F[S_\alpha[i]]$;
            \State $z = J[i]$;
            \While{$S_{\text{last}}[z] \neq 1$} \Comment{Reach the last child of $S_\alpha[i]$}
                \State $z = z + 1$;
            \EndWhile
            \State $F[S_\alpha[i]] = z + 1$;
        \EndIf
    \EndFor
    \State \Return $J$.
    \end{algorithmic}
\end{algorithm}

\subsection{Compressing Labeled Trees}
\begin{comment}
Let the $k$-context of a node $u \in T$ be the first $k$ symbols of $\pi(u)$. We denote this $k$-long prefix as $\pi_k[u]$. Thus, $\pi_k[u]$ represents the subpath of length $k$ leading to $u$ in $T$, or equivalently, the node $u$ descends from a subpath labeled as $\pi_k[u]$, where the nodes in $\pi_k[u]$ are encountered in an upward direction.
\end{comment}

The XBWT of a tree $T$ exhibits a local homogeneity property on $S_{\alpha}$: the labels ($\alpha$ components) of nodes whose upward paths ($\pi$ components) share long common prefixes appear in $S_{\alpha}$ in contiguous (or tightly bounded) clusters. 
This phenomenon can be formalized via the notion of $k$-contexts on trees.
This property mirrors the strong local homogeneity exhibited by strings under the Burrows-Wheeler Transform~\cite{burrows1994block} when applied to labeled trees.

To illustrate this, let us consider two arbitrary nodes $u$ and $v$ in $T$, and examine their contexts $\pi(u)$ and $\pi(v)$. Given the sorting of $S$, the greater the length of the shared prefix between $\pi(u)$ and $\pi(v)$, the closer the corresponding labels $\alpha(u)$ and $\alpha(v)$ will be in the string $S_{\alpha}$. These closely spaced labels are expected to be few in number, resulting in $S_{\alpha}$ exhibiting local homogeneity. As a consequence, we can leverage the advanced algorithmic techniques developed for BWT-based compression methods to achieve efficient compression.

At the end, the XBWT is used for turning the labeled tree compression problem into a string compression problem. To this aim, two string compressors
$C_{\alpha}$ and $C_{\text{last}}$ are used to compress the two strings $S_{\alpha}$ and $S_{\text{last}}$. respectively, by exploiting their fine specialties. Of course, many choices are possible for $C_{\alpha}$ and $C_{\text{last}}$, each having implications on the algorithmic time and compression bounds.

In general, the following theorem holds:

\begin{theorem}[\cite{ferragina2009compressing}, Theorem 4]
    let $C_{\alpha}$ be a $k$-th order string compressor that compresses any string $w$ into $|w|H_k(w) + |w| + o(|w|)$ bits, taking $O(|w|)$ time; and let $C_{\text{last}}$ be an algorithm that stores $S_{\text{last}}$ without compression. With this simple instantiation, the labeled tree $T$ can be compressed within $t H_k(S_{\alpha}) + 2t + o(t)$ bits and takes $O(t)$ optimal time.
\end{theorem}

Since $H_k(S_\alpha) \leq (\log |\Sigma|) + 1$
(where the additional $+1$ cost comes from the definition of $\alpha$ in \cref{def:node_informations}, since we consider the same alphabet $\Sigma$ for nodes and leaves), the above bound is at most $t(\log |\Sigma| + 3) + o(t)$ bits, and can be significantly better than the information-theoretic lower bound and the plain storage of the XBWT (both taking $2t + t \log|\Sigma|$ bits), depending on the distribution of the labels among its nodes.

\subsection{Indexing a Compressed Labeled Tree} \label{sec:xbwt_operations}
In order to implement the efficient operations listed in \cref{compandindexinglabtree} using the compressed arrays $S_{\text{last}}$ and $S_{\alpha}$ of XBWT, we need the chosen compressors $C_{\alpha}$ and $C_{\text{last}}$ to support the following operations:

Given a string $S[1, t]$ over alphabet $\Sigma$
\begin{itemize}
    \item \textbf{$rank_c(S, q)$}: gives the number of times the symbol $c \in \Sigma$ appears in $S[1, q]$.
    \item \textbf{$select_c(S, i)$}: gives the position of the $i$-th occurrence of the symbol $c \in \Sigma$ in $S$.
\end{itemize}

The compressed indexing of the XBWT will be based on three compressed data structures that support rank and select queries over the two strings $S_{\alpha}$ and $S_{\text{last}}$, and over an auxiliary binary array $A[1, t]$ defined as: $A[1] = 0$, $A[j] = 1$ if and only if the first symbol of $S_{\pi}[j]$ differs from the first symbol of $S_{\pi}[j - 1]$. 
Hence, $A$ contains at most $|\Sigma| + 1$ bits set to 1 out of $t$ positions. It is also easy to see that, through rank and select operations over $A$, we can succinctly implement the array $F$ employed in \cref{alg:rebuildTree,alg:buildF}.

%\alessio{Is $S$ a pair of strings or a string of pairs? $S$ such that $S[i] = (\Slast[i], S_\alpha[i])$, for $i = 1,...,t$?}
In this section, let $S \coloneq (S_{\text{last}}, S_{\alpha})$ such that $S[i] = (\Slast[i], S_\alpha[i])$ for $i = 1,\dots,t$, denote the XBWT obtained after the construction phase.
The compressed index supports the following methods:

\textbf{GetRankedChild($i$, $k$)}: Returns the position in $S$ of the $k$-th child of the node at index $i$. If the child does not exist, it returns -1. 
\begin{example}
    In \cref{tab:xbwt_example_2}, \texttt{GetRankedChild(2, 2)} returns 6.
\end{example}

\textbf{GetCharRankedChild($i$, $c$, $k$)}: Returns the position in $S$ of the $k$-th child labeled $c$ of the node at index $i$. If the child does not exist, it returns -1.
\begin{example}
    In \cref{tab:xbwt_example_2}, \texttt{GetCharRankedChild(1, B, 2)} returns 4.
\end{example}

\textbf{GetDegree($i$)}: Returns the total number of children of the node at index $i$ in $S$.

\textbf{GetCharDegree($i$, $c$)}: Returns the number of children of the node at index $i$ in $S$ that have the label $c$.

\textbf{GetParent($i$)}: Returns the position in $S$ of the parent of the node at index $i$. If the node is the root (at index 1), it returns -1.
\begin{example}
    In \cref{tab:xbwt_example_2}, \texttt{GetParent(8)} returns 4.
\end{example}

\textbf{GetSubtree($i$)}: Retrieves the labels of all nodes in the subtree rooted at the node at index $i$ in $S$. The labels can be returned in any standard traversal order (e.g., pre-order, in-order, or post-order).

\textbf{SubPathSearch($P$)}: For a given labeled path $P = c_1c_2 \cdots c_k$, this function finds the range $[\text{First}\dots\text{Last}]$ such that all strings in $S_{\pi}[\text{First}\dots\text{Last}]$ are prefixed by the reversed path $P^R = c_k \cdots c_2c_1$.
\begin{example}
    In \cref{tab:xbwt_example_2}, \texttt{SubPathSearch(BD)} results in the range [12, 13], and \\ \texttt{SubPathSearch(AB)} gives the range [5, 8].
\end{example}

It is important to note that their time complexity is dependent on the specific implementation for rank and select over the compressed strings $S_{\alpha}$ and $S_{\text{last}}$. 

Let's now see how to implement some of the above methods (from which the others can be derived) using the rank and select operations over the compressed strings $S_{\alpha}$ and $S_{\text{last}}$.

\begin{figure}
    \centering
    \begin{tabular}{r c c c l}
    \hline\hline
    \textbf{} & $A$ & $S_{\text{last}}$ & \textbf{$S_{\alpha}$} & \textbf{$S_{\pi}$} \\
    \hline
    \textbf{1} & 0 & 0 & A & $\epsilon$ \\
    \textbf{2} & 1 & 0 & B & A \\
    \textbf{3} & 0 & 0 & C & A \\
    \textbf{4} & 0 & 1 & B & A \\
    \textbf{5} & 1 & 0 & D & BA \\
    \textbf{6} & 0 & 0 & a & BA \\
    \textbf{7} & 0 & 1 & E & BA \\
    \textbf{8} & 0 & 1 & D & BA \\
    \textbf{9} & 1 & 0 & D & CA \\
    \textbf{10} & 0 & 0 & b & CA \\
    \textbf{11} & 0 & 1 & D & CA \\
    \textbf{12} & 1 & 1 & a & DBA \\
    \textbf{13} & 0 & 1 & b & DBA \\
    \textbf{14} & 0 & 1 & c & DCA \\
    \textbf{15} & 0 & 1 & c & DCA \\
    \textbf{16} & 1 & 1 & b & EBA \\
    \hline\hline
    \end{tabular}
    \caption{The sequence $S$ for the tree shown in \cref{fig:example_tree}, obtained by stably sorting triplets according to their `$\pi$' components. In contrast to \cref{tab:xbwt_example}, the auxiliary binary array $A$ is shown in the second column.}
    \label{tab:xbwt_example_2}
\end{figure}

\subsubsection*{GetChildren($i$)}
\cref{alg:getchildren} exploits directly the properties described before, in particular Property `Path-based Indexing' (\cref{prop3}). The rank operation at line 5 is used to get the number $r$ of nodes labeled $c$ up to position $i$ in $S_{\alpha}$. Then, the position $F[c]$ is obtained through a select operation on $A$ (line 6). By Property `Path-based Indexing', the children of $S[i]$ are located at the $r$-th block of children following position $F[c]$. Lines $8 - 9$ identify this block. 

\begin{example}
    Let's walk through an example using \cref{tab:xbwt_example_2}. Consider the node $u$ at index 2 labeled with $B$. To find its children:

    \begin{enumerate}
        \item First, we compute $r = 1$ since this is the first occurrence of $B$ in $S_{\alpha}$ up to position 2.
        \item Next, we find $y = F[B] = 5$, which marks the start of the block containing children of all nodes labeled $B$.
        \item Then, we count $z = 1$ ones in $S_{\text{last}}$ up to position $y-1$.
        \item Finally, the children block is delimited by the $z+r-1 = 1$st and $z+r = 2$nd ones in $S_{\text{last}}$, giving us the range $[5,7]$.
    \end{enumerate}

    This range $[5,7]$ indeed contains the three children of the node at index 2, as we can verify from the tree structure in \cref{fig:example_tree}.
\end{example}

\begin{algorithm}[H] 
    \caption{GetChildren($S_\alpha$, $\Slast$, $i$)}
    \label{alg:getchildren}
    \begin{algorithmic}[1]
    \If{$S_\alpha[i] \in \Sigma_L$}
        \State \Return $-1$ \Comment{$S[i]$ is a leaf}
    \EndIf
    \State $c = S_\alpha[i]$ \Comment{$S[i]$ is labeled $c$}
    \State $r = \text{rank}_c(S_\alpha, i)$
    \State $y = \text{select}_1(A, c)$ \Comment{$y = F[c]$}
    \State $z = \text{rank}_1(S_{\text{last}}, y - 1)$
    \State $\text{First} = \text{select}_1(S_{\text{last}}, z + r - 1) + 1$
    \State $\text{Last} = \text{select}_1(S_{\text{last}}, z + r)$
    \State \Return $(\text{First}, \text{Last})$
    \end{algorithmic}
\end{algorithm}

\subsubsection*{GetParent($i$)}
\cref{alg:getparent} is based on Property `Path-based Indexing' (\cref{prop3}) and it is the inverse of the GetChildren method. In line 4, the algorithm computes the label $c$ of the parent of $S[i]$ that prefixes the upward path leading to $S[i]$. Then, the parent of $S[i]$ is searched among the nodes labeled $c$ in $S_{\alpha}$ by exploiting Property `Path-based Indexing' in a reverse manner. Namely, the number $k$ of children-blocks in the range $S[y, i]$ is computed; these are children of nodes labeled $c$ and preceding $i$ in the stable sort of $S$. Then, the $k$-th occurrence of $c$ in $S_{\alpha}$ is selected, which is indeed the parent of $S[i]$.

\begin{example}
    Let's illustrate how to find a node's parent using \cref{tab:xbwt_example_2}. Consider node $v$ located at index 4 with label $B$. The process to find its parent involves:
    \begin{enumerate}
        \item Computing $c = \text{rank}_1(A, 4) = 1$, which tells us the parent has label `A' (as $A$ contains exactly one 1 up to position 4).
        \item Locating $y = F[A] = 2$, which indicates where the block of children for nodes labeled `A' begins.
        \item Calculating $k = \text{rank}_1(S_{\text{last}}, 4-1) - \text{rank}_1(S_{\text{last}}, 2-1) = 0$, meaning no complete child blocks appear before position 4.
        \item Therefore, $v$'s parent is the first ($(k+1)$-th) occurrence of `A' in $S_{\alpha}$, corresponding to index 1 (the root of $\mathcal{T}$).
    \end{enumerate}
    This example demonstrates how the XBWT structure efficiently encodes parent-child relationships using just the $S_{\text{last}}$ and $S_{\alpha}$ arrays.
\end{example}

\begin{algorithm}[H]
    \caption{GetParent($S_\alpha$, $\Slast$, $i$)}
    \label{alg:getparent}
    \begin{algorithmic}[1]
    \If{$i = 1$}
        \State \Return $-1$ \Comment{$S[i]$ is the root of $\mathcal{T}$}
    \EndIf
    \State $c = \text{rank}_1(A, i)$
    \State $y = \text{select}_1(A, c)$
    \State $k = \text{rank}_1(S_{\text{last}}, i - 1) - \text{rank}_1(S_{\text{last}}, y - 1)$
    \State $p = \text{select}_c(S_\alpha, k + 1)$
    \State \Return $p$
    \end{algorithmic}
\end{algorithm}

\subsubsection*{SubPathSearch($P$)}
We assume that $P = c_1c_2 \cdots c_k$ algorithm SubPathSearch computes the range $[First, Last]$ in $|P| = l$ phases, each one preserving the following invariant:

\begin{itemize}
    \item Invariant of Phase $i$. At the end of the phase, $S_{\pi}[First]$ is the first entry prefixed by $P[1, i]^R$ , and $S_{\pi}[Last]$ is the last entry prefixed by $P[1, i]^R$ , where $s^R$ is the reversal of string $s$.
\end{itemize}

At the beginning (i.e., $i = 1$), First and Last are easily determined via the entries $F[c_1]$ and $F[c_1 + 1] - 1$, which point to the first and last entry of $S_{\pi}$ prefixed by $c_1$ (by definition of array $F$). Since we do not have the $F$ array, we implement these operations via rank and select queries over array $A$. Let us assume that the invariant holds for Phase $i - 1$, and prove that the $i$-th iteration of the for-loop in algorithm SubPathSearch preserves the invariant. More precisely, let $S_{\pi}[First, Last]$ be all entries prefixed by $P[1, i - 1]^R$. So $S[First, Last]$ contains all nodes descending from $P[1, i - 1]$. SubPathSearch determines $S[z_1]$ (respectively $S[z_2]$) as the first (respectively last) node in $S[First, Last]$ that descends from $P[1, i - 1]$ and is labeled $c_i$, if any. Then it jumps to the first child of $S[z_1]$ and the last child of $S[z_2]$. From Property 2 (item 2) and the correctness of algorithms GetChildren and GetDegree, we infer that the positions of these two children are exactly the first (respectively last) entry in $S$ whose $\pi$-component is prefixed by $P[1, i]^R$. 

The time complexity of the SubPathSearch algorithm is $O(l)$, where $l$ is the length of the input path $P$.

\begin{example}
    Consider the tree in \cref{fig:example_tree}, and let $P = BD$. The algorithm \textsc{SubPathSearch}($P$) returns the range $[12, 13]$ through the following steps:

    \begin{enumerate}
        \item Initially, $First = F[B] = 5$ and $Last = F[C] - 1 = 8$. The range $S[5,8]$ contains all nodes descending from paths prefixed by $B$.
        
        \item For $c_2 = D$:
        \begin{itemize}
            \item Compute $k_1 = 0$ and $k_2 = 2$
            \item This yields $z_1 = 5$ and $z_2 = 8$
            \item The first child of $S[5]$ is at position $12$
            \item The last (and only) child of $S[8]$ is at position $13$
        \end{itemize}
        
        \item Therefore, the algorithm returns the range $[12,13]$
    \end{enumerate}

    Note that the number of occurrences of subpath $P$ is 2, as evidenced by the two occurrences of 1 in range $S_{\text{last}}[12,13]$.
\end{example}

\begin{algorithm}[H]
    \caption{\textsc{SubPathSearch}($S_\alpha$, $\Slast$, $P$)}
    \label{alg:subpathsearch}
    \begin{algorithmic}[1]
    \State $First = F(c_1)$; $Last = F(c_1 + 1) - 1$
    \If{$First > Last$}
        \State \textbf{return} ``$P$ is not a subpath of $T$''
    \EndIf
    \For{$i = 2, \dots, k$}
        \State $k_1 = \text{rank}_{c_i}(S_\alpha, First - 1)$; 
        \State $z_1 = \text{select}_{c_i}(S_\alpha, k_1 + 1)$
        \Comment{first entry in $S_\alpha[First, t]$ labeled $c_i$}
        \State $k_2 = \text{rank}_{c_i}(S_\alpha, Last)$; 
        \State $z_2 = \text{select}_{c_i}(S_\alpha, k_2)$
        \Comment{last entry in $S_\alpha[1, Last]$ labeled $c_i$}
        \If{$z_1 > z_2$}
            \State \textbf{return} ``$P$ is not a subpath of $T$''
        \EndIf
        \State $First = \text{GetRankedChild}(z_1, 1)$ \Comment{get the first child of $S[z_1]$}
        \State $Last = \text{GetRankedChild}(z_2, \text{GetDegree}(z_2))$ \Comment{get the last child of $S[z_2]$}
    \EndFor
    \State \textbf{return} $(First, Last)$
    \end{algorithmic}
\end{algorithm}

\subsection{Summary}
To sum up, the XBWT has several key properties that make it an effective tool for labeled tree compression and indexing:
\begin{itemize}
    \item \textbf{Succinctness:} The XBWT representation of a labeled tree uses space close to the worst-case entropy (\cref{lem:info_theoretic_lower_bound}), which is \(H_{wc} = 2t + t \log |\Sigma| - \Theta(\log t) \) bits for a tree with $t$ nodes and an alphabet of size $|\Sigma|$.
    Notice that the second component of $H_{wc}$ is $t\log |\Sigma|$ and not $m\log |\Sigma|$ since the XBWT works on \emph{node} labeled trees.
    \item \textbf{Efficient Querying:} The XBWT supports navigational queries (\cref{def:tree_operations}) in optimal time $O(1)$ if $|\Sigma| = O(polylog(t))$, otherwise in $O(\log\log^{1+\epsilon} |\Sigma|)$ time. Whereas, given $s \in \Sigma^*$, subpath queries (\cref{def:tree_operations}) are supported in $O(|s|)$ if $|\Sigma| = O(polylog(t))$, otherwise in $O(|s| \log\log^{1+\epsilon} |\Sigma|)$ time.
    \item \textbf{Scalability:} The XBWT is particularly useful for large-scale hierarchical data, such as XML documents or phylogenetic trees, where both compression and fast querying are critical.
\end{itemize}
\subsection{Implementation} \label{sec:xbwt_impl}

% \nicola{Il lettore qui rimane spiazzato: stai descrivendo lo stato dell'arte e tutto ad un tratto spunta un'implementazione e degli esperimenti.  Sposta questa sezione in sezione 4 (in una sottosezione a parte) e motiva perché hai re-implementato la XBWT (cioè per compararla con il tuo compressore). }


The XBWT data structure has been implemented in C++ using the Succinct Data Structure Library 2.0 (SDSL) for efficient representation and manipulation of compressed data structures. Also, we will implement the necessary data structures and algorithms for navigating and querying the XBWT, such as parent-child navigation and path-based searches. 

The implementation of the XBWT is based on the descriptions provided in \cref{sec:XBWT} and it is available on GitHub at the following link: \url{https://github.com/davide-tonetto-884585/XBWT}.

\subsubsection{Implementation Choices}
Follows a list of the main choices made during the implementation of the XBWT:
\begin{itemize}
    \item The implementation is not focused on a specific kind of data, such as XML documents or JSON files, but it is designed to work with any kind of labeled tree. 
    \item The construction method takes as input a labeled tree. It constructs directly a compressed indexing scheme based on the Extended Burrows-Wheeler Transform of the tree as described in the previous sections.
    \item The implementation is based on the Succinct Data Structure Library (SDSL) to handle the compressed data structures generated by the XBWT. The SDSL library provides efficient implementations of various compressed data structures and algorithms, which are essential for representing and querying the XBWT efficiently.
    \item The labels of the alphabet are encoded as integers, starting from 0 to $|\Sigma| - 1$, where $|\Sigma|$ is the cardinality of the alphabet. This encoding respects the order of the labels in the alphabet and allows simplifying and reducing the space needed to store the labels in the compressed data structure. For this reason, the constructor of the XBWT class takes as input a generic labeled tree.
\end{itemize}

\subsubsection{Succinct Data Structures}
The implementation of the XBWT relies heavily on succinct data structures to achieve space efficiency while maintaining fast query operations. In particular, we use succinct data structures to compress the two main arrays of the XBWT: $S_\alpha$ and $S_{\text{last}}$. These arrays, which can be quite large for substantial trees, benefit significantly from compression.

The compression is achieved through the Succinct Data Structure Library (SDSL), which provides efficient implementations of various compressed data structures. For $S_{\text{last}}$, which is a binary sequence, we utilize a compressed bit vector that supports fast rank and select operations. For $S_\alpha$, which contains labels from a potentially large alphabet, we employ a wavelet tree structure that provides both compression and efficient query capabilities.

The SDSL is a C++ library that provides efficient implementations of various compressed data structures and algorithms. It is used in this project to handle the compressed data structures composing the XBWT. The SDSL library provides a wide range of succinct data structures, such as bit vectors, wavelet trees, and compressed suffix arrays, which are essential for representing and querying the XBWT efficiently. The library is available at \url{https://github.com/simongog/sdsl-lite}~\cite{gbmp2014sea}. Let's see the implementation details of the SDSL data structures used in the XBWT implementation.

\subsubsection{RRR Bit Vector}
The RRR bit vector is designed to provide space-efficient representations of bit vectors while supporting efficient rank and select operations. This data structure implements the RRR (Raman, Raman, and Rao) encoding method, which compresses bit vectors by partitioning them into fixed-size blocks and encoding each block based on its population count (the number of 1s) and specific configuration~\cite{raman2002succinct}. 

The space needed for an RRR bit vector of length $n$ with $m$ set bits is $nH_0 + o(n)$ ($\approx \lceil \log \binom{n}{m} \rceil$). 
The rank support is provided by \texttt{sdsl::rank\_support\_rrr}, adding $80$ bits and requiring $O(\log k)$ time for rank queries, where $k$ is the number of set bits. The select support is provided by \texttt{sdsl::select\_support\_rrr}, adding $64$ bits and requiring $O(\log n)$ time for select queries.

This data structure is used to represent $S_{\text{last}}$, a dedicated binary array $B_{\alpha}$ that stores the additional information associated with each entry of $S_{\alpha}$ (i.e., $B_{\alpha}[i]=1$ if the $i$-th symbol in $S_{\alpha}$ corresponds to a leaf and $0$ otherwise), and the $A$ array of the XBWT.

\subsubsection{Wavelet Tree}
The Wavelet tree is designed to efficiently handle sequences over large alphabets, such as integer sequences. It provides a space-efficient representation while supporting fast access, rank, and select operations. The wavelet tree is a balanced binary tree that recursively partitions the alphabet into two equal-sized subsets and encodes the sequence based on the partitioning~\cite{grossi2003high}. The \texttt{sdsl::wt\_int} uses the RRR bit vectors or other succinct representations for storing the bit vectors in each node of the wavelet tree. This makes the structure space-efficient.

If RRR-compressed bitvectors are used for the internal bitmaps, a wavelet tree over a sequence $S \in \Sigma^n$ (with $|\Sigma|=\sigma$) occupies $n H_0(S) + o(n \log \sigma) + \Theta(\sigma \log n)$ bits of space, where $H_0(S)$ is the zero-order empirical entropy of $S$, and it supports access, rank, and select queries in $O(\log \sigma)$ time.

This data structure is used to represent the $S_\alpha$ array of the XBWT.

\begin{comment}
\subsection{Details of the XBWT Class Elements}
\alessio{Questa sezione non mi convince. Non descrivere troppo il codice, la documentazione dovrebbe essere nella repo e nel codice, non nella tesi.}
The XBWT class utilizes several data structures from the SDSL library to efficiently represent and query the compressed data. Below are the details of the main elements used in the class:

\begin{itemize}
    \item \texttt{sdsl::rrr\_vector<> SLastCompressed}: This is a compressed bit vector that stores the $S_{\text{last}}$ array of the XBWT. 
    \item \texttt{sdsl::wt\_int<sdsl::rrr\_vector<>> SAlphaCompressed}: This is a wavelet tree built on top of a compressed bit vector. The wavelet tree is used to compress and index the $S_{\alpha}$ array of the XBWT.
    \item \texttt{sdsl::rrr\_vector<> SAlphaBitCompressed}: Another compressed bit vector used to store the additional bit of $S_{\alpha}$ needed to distinguish between internal and leaf nodes.
    \item \texttt{sdsl::rrr\_vector<> ACompressed}: A compressed bit vector representing the $A$ array of the XBWT used to in the $F$ array of the XBWT.
    \item \texttt{sdsl::rrr\_vector<>::rank\_1\_type SLastCompressedRank}: A rank support structure for the \texttt{SLastCompressed} bit vector, allowing efficient rank queries.
    \item \texttt{sdsl::rrr\_vector<>::select\_1\_type SLastCompressedSelect}: A select support structure for the \texttt{SLastCompressed} bit vector, allowing efficient select queries.
    \item \texttt{sdsl::rrr\_vector<>::rank\_1\_type ACompressedRank}: A rank support structure for the \texttt{ACompressed} bit vector.
    \item \texttt{sdsl::rrr\_vector<>::select\_1\_type ACompressedSelect}: A select support structure for the \texttt{ACompressed} bit vector.
    \item \texttt{std::unordered\_map<T, unsigned int> alphabetMap}: A hash map that maps each label in the alphabet to a unique integer.
    \item \texttt{unsigned int cardSigma}: The cardinality of the alphabet $\Sigma$.
    \item \texttt{unsigned int cardSigmaN}: The cardinality of the $\Sigma_N$ alphabet. Where $\Sigma_N$ is the set of labels that appear in the internal nodes of the labeled tree.
    \item \texttt{unsigned int maxNumDigits}: The maximum number of digits that has the integer code associated with the greater label in the alphabet (needed to sort the labels in the alphabet).
\end{itemize}

The overall space complexity of the XBWT class can be derived from the space complexity of the compressed data structures used in the class. 

\subsection{Construction implementation}
\alessio{Perché non fare pseudocodice come per Alg 1?}
The construction of the XBWT is done by the constructor of the XBWT class. The constructor takes as input a generic labeled tree and constructs the compressed indexing scheme using the linear pathSort (also, the naive construction method can be used by passing the boolean flag \texttt{usePathSort = false}). The construction process is divided into the following steps:

\begin{enumerate}
    \item \textbf{Alphabet Encoding}: The first step is to encode the labels of the alphabet as integers. The labels are sorted in lexicographical order and assigned a unique integer code starting from 1 to $|\Sigma|$. Two hash maps are used to map each label to a unique integer and vice versa. 
    \item \textbf{Construct \texttt{intNodes} array}: The next step is to construct the \texttt{intNodes} array as described in the previous chapters. \texttt{intNodes} is an array of triplets of length $t$ in which a node is represented as a triplet containing the node's label, its level, and the index of its parent node in the array (from $1 to t$, root has parent $0$). The nodes are inserted in a preorder traversal of the labeled tree.
    \item \textbf{Sort \texttt{intNodes} array:} Call the \texttt{pathSort} or \texttt{upwardStableSortConstruction} (naive method) method to get the sorted array of nodes \texttt{intNodes}.
    \item \textbf{Construct $S_{\text{last}}$ array}: Construct the $S_{\text{last}}$ array by iterating over the sorted \texttt{intNodes} array.
    \item \textbf{Construct $S_{\alpha}$ array}: Construct the $S_{\alpha}$ array by iterating over the sorted \texttt{intNodes} array, along with the additional bit array to distinguish between internal and leaf nodes.
    \item \textbf{Construct $A$ array}: Construct the $A$ array by iterating over the sorted \texttt{intNodes} array.
    \item \textbf{Construct rank and select support structures}: Construct the rank and select support structures for the compressed bit vectors.
\end{enumerate}

\subsection{Navigational Operations}
\alessio{Non basta dire che implementi tutte le funzioni descritte nella sezione 3.7?}

The XBWT class provides several navigational operations to traverse the labeled tree and retrieve information about the nodes. The navigational operations implemented are:

\begin{itemize}
    \item \texttt{getChildren(unsigned int i)}: This method returns a pair of integers representing the indices of the leftmost and rightmost children of the node at index \texttt{i}.
    \item \texttt{getRankedChild(unsigned int i, unsigned int k)}: This method returns the index of the \texttt{k}-th child of the node at index \texttt{i}.
    \item \texttt{getCharRankedChild(unsigned int i, T label, unsigned int k) const}: This method returns the index of the \texttt{k}-th child of the node at index \texttt{i} with the specified label.
    \item \texttt{getDegree(unsigned int i)}: This method returns the degree (number of children) of the node at index \texttt{i}.
    \item \texttt{getCharDegree(unsigned int i, T label)}: This method returns the number of children of the node at index \texttt{i} with the specified label.
    \item \texttt{getParent(unsigned int i)}: This method returns the index of the parent of the node at index \texttt{i}.
    \item \texttt{getSubtree(unsigned int i, unsigned int order = 0)}: This method returns a vector containing the labels of the nodes in the subtree rooted at index \texttt{i}. The \texttt{order} parameter specifies the traversal order (e.g., preorder, post-order).
\end{itemize}

All the methods refer to the index of the nodes in $S_{\text{last}}$ and $S_{\alpha}$ arrays. 

\subsection{Search Operations}
The XBWT class provides search operation \texttt{subPathSearch(const std::vector<T> \&path)} that searches for a subpath in the XBWT structure. It uses the compressed vectors to determine the range of positions corresponding to the nodes whose upward path is prefixed by a given vector reversed.

\end{comment}

\subsection{Construction Time Experiments} 

To evaluate the performance of the implemented algorithms, we conducted a series of experiments on randomly generated trees, with sizes ranging from 100 to 900,000 nodes. For each tree, we executed the construction algorithms 10 times, recording the average execution time for both the linear \textsc{PathSort} algorithm and the Naive Sort algorithm used for constructing the XBWT.
By ``Naive Sort'', we refer to a straightforward approach that first precomputes all the upward paths in the tree and then sorts them using a standard sorting algorithm. This method contrasts with \textsc{PathSort}, which is specifically designed to achieve linear time complexity in path sorting.
This approach enabled us to compare their performance across various tree sizes and evaluate their scalability.

From the results shown in Table \cref{tab:experiments}, we can draw several conclusions about the performance of the \textsc{PathSort} algorithm compared to the Naive Sort algorithm.

The \textsc{PathSort} algorithm consistently outperforms the Naive Sort algorithm in terms of execution time, especially as the number of nodes increases. For smaller trees, the difference in execution time between the two algorithms is minimal. However, as the number of nodes grows, the \textsc{PathSort} algorithm demonstrates significantly better scalability. For instance, with 900,000 nodes, the \textsc{PathSort} algorithm takes 8.51 seconds, whereas the Naive Sort algorithm takes 34.2 seconds, giving a speedup of more than $4\times$. A visual representation of the results can be seen in \cref{fig:xbwt_exp_plots} where both the time comparison and the speedup are shown.

\begin{table}[h]
    \centering
    \begin{tabular}{|r|r||r|r|}
        \hline
        \textbf{\# Nodes} & \textbf{Depth} & \textbf{Naive Sort (s)} & \textbf{\textsc{PathSort} (s)} \\
        \hline
            100 &    22 &  0.001 & 0.002 \\
            500 &    45 &  0.002 & 0.004 \\
          1,000 &    74 &  0.003 & 0.006 \\
          5,000 &   175 &  0.015 & 0.028 \\
         10,000 &   288 &  0.053 & 0.056 \\
         50,000 &   486 &  0.350 & 0.310 \\
        100,000 &   754 &  1.250 & 0.690 \\
        500,000 & 2,246 & 16.460 & 4.700 \\
        900,000 & 2,658 & 34.200 & 8.510 \\
        \hline
    \end{tabular}
    \caption{Performance comparison of the execution times for the Naive Sort and \textsc{PathSort} algorithms on randomly generated trees of varying sizes.}
    \label{tab:experiments}
\end{table}

\begin{figure}[H]
    \centering
    \begin{subfigure}[b]{0.48\textwidth}
        \centering
        \includegraphics[width=\textwidth]{"Immagini/execution_time_comparison.pdf"}
        \caption{Time Comparison}
        \label{fig:time_comparison}
    \end{subfigure}
    \hfill %
    \begin{subfigure}[b]{0.48\textwidth}
        \centering
        \includegraphics[width=\textwidth]{"Immagini/speedup_comparison.pdf"}
        \caption{Speedup}
        \label{fig:speedup}
    \end{subfigure}
    \caption{Time comparison and speedup plots for the experiments in \cref{tab:experiments}. Image (a) shows \textsc{PathSort} time in seconds (yellow dashed line) vs. Naive Sort time (blue line). Image (b) shows the speedup of \textsc{PathSort} over Naive Sort. The area below the red dashed line indicates points where the speedup is less than or equal to 1, representing no improvement of \textsc{PathSort} over Naive Sort.}
    \label{fig:xbwt_exp_plots}
\end{figure}

\begin{comment}
\subsubsection{Space Analysis}
To evaluate the space savings achieved through XBWT compression, we conducted experiments on the same set of randomly generated trees used for the construction performance tests. For each tree, we compared the memory usage (in bytes) of three representations: the plain tree, the uncompressed XBWT, and the compressed XBWT.

The plain tree representation consists of the simple balanced parentheses encoding of the tree structure combined with the edge labels. For example, for the tree in \cref{fig:example_tree,tab:xbwt_example}, the plain tree representation would be:

\texttt{(A(B(D(a))(a)(E(b)))(C(D(c))(b)(D(c)))(B(D(b))))}.

By \emph{uncompressed XBWT}, we refer to the XBWT arrays $S_{\text{last}}$ and $S_{\alpha}$ (including the additional bit) stored without any compression. Specifically, $S_{\text{last}}$ is represented as a plain bitvector (\texttt{sdsl::bit\_vector}), and $S_{\alpha}$ is stored as a wavelet tree (\texttt{sdsl::wt\_int}) with plain bitvectors (\texttt{sdsl::bit\_vector}). In contrast, the \emph{compressed XBWT} representation stores $S_{\text{last}}$ and $S_{A}$ as compressed RRR bitvectors (\texttt{sdsl::rrr\_vector}), and $S_{\alpha}$ as a wavelet tree with RRR bitvectors, as described in the previous chapter.

\cref{tab:experiments_2} reports the sizes (in bytes) for each representation of the trees across different sizes. The last column highlights the space savings achieved by the compressed XBWT compared to the plain tree representation, expressed as a percentage. These results illustrate the substantial space reductions achieved through compression, especially as the tree size increases.

%\alessio{Oltre ai punti di prima, metti la percentuale anche per UXBWT, magari non come un'altra colonna ma metti tra parentesi. Te lo faccio sulle prime righe per la C.XBWT. Se ti piace, ricorda di spiegare cosa sono i numeri tra parentesi nella descrizione.}
\begin{table}[ht]
    \centering
    \begin{tabular}{|r||r|r|r|}
        \hline
        \textbf{\# Nodes} & \textbf{Plain tree} & \textbf{U. XBWT} & \textbf{C. XBWT} \\
        \hline
            100 &       390 &       424 &       496 \color{gray}{(-27.18\%)}\\
            500 &     2,390 &     1,112 &     1,136 ~\color{gray}{(52.47\%)} \\
          1,000 &     4,890 &     2,242 &     2,056 ~\color{gray}{(57.96\%)} \\
          5,000 &    28,890 &    12,911 &    10,400 ~\color{gray}{(64.00\%)} \\
         10,000 &    58,890 &    45,625 &    21,848 ~\color{gray}{(62.90\%)} \\
         50,000 &   338,890 &   175,146 &   123,216 ~\color{gray}{(63.64\%)} \\
        100,000 &   688,890 &   349,478 &   259,376 ~\color{gray}{(62.35\%)} \\
        500,000 & 3,888,890 & 1,850,850 & 1,451,570 ~\color{gray}{(62.67\%)} \\
        900,000 & 7,088,890 & 3,480,190 & 2,718,570 ~\color{gray}{(61.65\%)} \\
        \hline
    \end{tabular}
    \caption{Space analysis of the XBWT. The first column represents the number of nodes, the others the bytes used by each representation.
    ``Plain tree'' is the size of the tree in the simple balanced parenthesis representation plus the edge labels, ``U. XBWT'' is the size of the uncompressed XBWT, and ``C. XBWT'' is the size of the compressed XBWT.
    Gray numbers between parentheses represent the improvement relative to the plain tree representation.}
    \label{tab:experiments_2}
\end{table}

For small trees, the compressed XBWT does not always provide immediate savings due to the overhead of succinct data structures. For instance, for 100 nodes, the compressed representation is larger than the plain tree, showing a \(-27.18\%\) increase in space. However, as the number of nodes increases, the compression becomes more effective, achieving savings of over 60\% for large trees.

The space reduction becomes particularly evident for trees with more than 500 nodes. These results confirm that the compressed XBWT provides a scalable and space-efficient alternative for storing and indexing labeled trees. The efficiency gains are particularly beneficial for applications requiring large-scale tree processing, such as bioinformatics and text indexing.
\end{comment}

\chapter{Hopcroft Algorithm for Minimization of DFA}

\section{Introduction}
The process of automata minimization is the process of reducing the number of states in a DFA while preserving the language accepted by the DFA. The minimization of DFA is crucial for a variety of applications, such model checking, hardware
design, and compilers, as it produces a more effective and compact representation of the automata.

The minimization of DFA is a well-studied problem in automata theory, and there are several algorithms available for this purpose. One of the most popular algorithms for DFA minimization is the Hopcroft algorithm, which was proposed by John Hopcroft in 1971 \cite{HOPCROFT1971189}. The Hopcroft algorithm is an efficient and simple algorithm that can minimize a DFA in $O(n \log n)$ time, where $n$ is the number of states in the DFA.

Before we delve into the details of the Hopcroft algorithm, let us first introduce the concept of DFA.
\subsection{Deterministic Finite Automata (DFA)}
A deterministic finite automaton (DFA) is a 5-tuple $M = (Q, \Sigma, \delta, q_0, F)$ where:
\begin{itemize}
    \item $Q$ is a finite set of states
    \item $\Sigma$ is a finite set of input symbols (alphabet)
    \item $\delta: Q \times \Sigma \rightarrow Q$ is the transition function
    \item $q_0 \in Q$ is the initial state
    \item $F \subseteq Q$ is the set of final (accepting) states
\end{itemize}

The DFA processes an input string by starting from the initial state $q_0$ and following transitions based on the input symbols. The string is accepted if the DFA ends in an accepting state after processing all input symbols.

\section{Hopcroft's Minimization Algorithm}
Minimization of deterministic finite automata (DFA) is a classical and widely studied problem in Theory of Automata and Formal Languages. It consists in finding the unique (up to isomorphism) finite automaton with the minimal number of states, recognizing the same regular language of a given DFA.

The Hopcroft algorithm works by iteratively refining partitions of states until no further refinement is possible. The algorithm is the following:

\begin{algorithm}
    \caption{Hopcroft's Algorithm: DFA Minimization ($\mathcal{A} = (Q, \Sigma, \delta, q_0, F)$)}
    \begin{algorithmic}[1]
        \State $\Pi \gets \{F, Q \setminus F\}$
        \ForAll{$a \in \Sigma$}
            \State $\mathcal{W} \gets \{(\min(F, Q \setminus F), a)\}$
        \EndFor
        \While{$\mathcal{W} \neq \emptyset$}
            \State choose and delete any $(C, a)$ from $\mathcal{W}$
            \ForAll{$B \in \Pi$}
                \If{$B$ is split from $(C, a)$}
                    \State $B' \gets \delta_a^{-1}(C) \cap B$
                    \State $B'' \gets B \setminus \delta_a^{-1}(C)$
                    \State $\Pi \gets \Pi \setminus \{B\} \cup \{B', B''\}$
                    \ForAll{$b \in \Sigma$}
                        \If{$(B, b) \in \mathcal{W}$}
                            \State $\mathcal{W} \gets \mathcal{W} \setminus \{(B, b)\} \cup \{(B', b), (B'', b)\}$
                        \Else
                            \State $\mathcal{W} \gets \mathcal{W} \cup \{(\min(B', B''), b)\}$
                        \EndIf
                    \EndFor
                \EndIf
            \EndFor
        \EndWhile
    \end{algorithmic}
\end{algorithm}

The algorithm enables to compute equivalence classes of nodes in $O(n\log n)$, in particular, the Myhill-Nerode equivalence classes. The Myhill-Nerode theorem states that a language is regular if and only if it has a finite number of Myhill-Nerode equivalence classes. This theorem provides a powerful tool for determining the regularity of languages and is a cornerstone of automata theory. Let's formalize the concept of equivalence classes and the Myhill-Nerode theorem.

\begin{definition}[Equivalence Relation]
    For a language $L \subseteq \Sigma^*$ and any strings $x,y \in \Sigma^*$, we say $x$ is equivalent to $y$ with respect to $L$ (written as $x \approx_L y$) if and only if for all strings $z \in \Sigma^*$:
    \[ xz \in L \Leftrightarrow yz \in L \]
    That is, strings $x$ and $y$ are equivalent if they have the same behavior with respect to the language $L$ - either they both lead to acceptance or both lead to rejection when any suffix $z$ is appended.
\end{definition}

\begin{theorem}[Myhill-Nerode theorem] \label{def:myhill-nerode}
    Let $L$ be a language over an alphabet $\Sigma$. Then $L$ is regular if and only if there exists a finite number of Myhill-Nerode equivalence classes for $L$. Specifically, the number of equivalence classes is equal to the number of states in the minimal DFA recognizing $L$.
\end{theorem}

\section{Minimization of acyclic DFA in linear time}
For our purpose, we will focus on a specific type of finite automaton: an acyclic deterministic finite automaton (or DAWG). An acyclic DFA is one where there are no cycles within the transitions. This property simplifies the minimization process since it ensures that every state can be reached from the start state through a unique path.

Let's start by giving the notion of directed acyclic word graph (DAWG):
\begin{definition}[DAWG]
    A \textbf{DAWG} (directed acyclic word graph) or automaton $\mathcal{A}$ is defined by the following 5-uple:
    \[
    \mathcal{A} = (Q, \Sigma, F, T, q_0),
    \]
    where
    \begin{itemize}
        \item $Q$ is a set of states;
        \item $\Sigma$ is an alphabet of finite cardinal denoted by $|\Sigma|$;
        \item $q_0$ is the initial state;
        \item $T$ is the subset of terminal states of $Q$;
        \item $F$ is a function of $Q \times \Sigma$ into $Q$ defining the transitions (arcs) of the automaton.
    \end{itemize}
\end{definition}

In this section, we will discuss an efficient algorithm for minimizing acyclic deterministic finite automata in linear time on the number of states \cite{revuz1992minimisation}. The minimization process involves identifying and merging equivalent states. Two states are considered equivalent if they have the same set of reachable final states, meaning that from any state $q$, there is a path to a final state in both states. This equivalence relation partitions the DAWG into disjoint sets of states, each representing an equivalence class. The purpose of using this approach is then to apply it to the input tree for our pipeline since the problem can directly be applied to trees where the leaf nodes are considered as final states, the root as the initial state and the edges of the tree are considered directed from node to its children. 

\subsection{The minimization algorithm}
The minimization algorithm introduced in \cite{revuz1992minimisation} operates by labeling each state with a unique identifier that represents the structure of the automaton from that state onward. It proceeds in the following steps:

\begin{enumerate}
    \item \textbf{Height Computation:} The height of each state is determined, where the height of a state is the length of the longest path from that state to a final state.
    \item \textbf{State Labeling:} Each state is labeled based on the structure of its transitions. The label consists of:
    \begin{itemize}
        \item Whether the state is final or not.
        \item The transitions, recorded as ordered pairs of symbols and target state identifiers.
    \end{itemize}
    \item \textbf{Lexicographic Sorting:} States at each height level are sorted lexicographically based on their labels using a bucket sort technique.
    \item \textbf{Merging Equivalent States:} After sorting, states with identical labels are merged, ensuring that equivalent states are unified.
\end{enumerate}

\section{Wheeler and \texorpdfstring{$p$}{p}-sortable Graphs}
\label{sec:wheeler_and_psortable_graphs}

\subsection{Introduction and Motivation}
As established in the introduction, the primary goal of this thesis is to find an effective balance between compressing a finite language and preserving its indexability. The two extremes—full DFA minimization and the raw input trie—are inadequate, as one sacrifices indexing for compression and the other sacrifices compression for indexing. The solution to this problem lies in a specific class of graphs that are structured enough to be indexed efficiently yet flexible enough to allow for significant compression. This chapter introduces the theoretical framework that underpins our approach: Wheeler graphs and their generalization, $p$-sortable graphs.

A crucial observation is that the input trie representing our language is already a highly structured object. It is a Wheeler graph, a type of graph that admits a special ordering on its nodes and edges, making it exceptionally well-suited for indexing. In formal terms, a trie is a 1-sortable graph. This property explains both its powerful indexing capabilities and its inherent lack of compression.

The concept of $p$-sortability offers a way to navigate the trade-off. By controllably increasing the sortability parameter $p$, we can begin to merge MN-equivalent states (see \cref{def:myhill-nerode}), thereby compressing the graph. The resulting automaton is no longer a simple trie but a more general $p$-sortable graph that retains strong indexing properties. This chapter will formally define these concepts, which are the foundation of our algorithm for achieving a practical compromise between compression and indexability.

\subsection{Orders}
The core property that makes Wheeler and $p$-sortable graphs efficiently indexable is the existence of a specific ordering on their states. This ordering provides the necessary structure to navigate the automaton and answer queries quickly, a task that is computationally hard on general graphs. The fundamental ordering used in this context is the \textit{co-lexicographic order} (co-lex), which compares states based on the labels of the paths that reach them. This section formally defines this order and the related concepts that are essential for understanding the structure of indexable automata.

\begin{definition} [Co-lexicographic Order on $\Sigma^*$]
    The co-lex order $\preceq$ is defined as follows. Given two strings $\alpha, \beta \in \Sigma^*$, we say that $\alpha \preceq \beta$ if and only if either:
    \begin{itemize}
        \item $\alpha$ is a suffix of $\beta$, or
        \item there exist strings $\alpha', \beta', \gamma \in \Sigma^*$ and symbols $a, b \in \Sigma$, such that $\alpha = \alpha'a\gamma$, $\beta = \beta'b\gamma$, and $a \prec b$.
    \end{itemize}
\end{definition}

Now, let us define the formal concept of partial order and the width of a partial order. 
\begin{definition}[Partial Order]
    A partial order is a binary relation $\leq$ over a set $S$ that is reflexive, antisymmetric, and transitive. That is, for all $a, b, c \in S$:
    \begin{itemize}
        \item $a \leq a$ (reflexivity)
        \item if $a \leq b$ and $b \leq a$, then $a = b$ (antisymmetry)
        \item if $a \leq b$ and $b \leq c$, then $a \leq c$ (transitivity)
    \end{itemize}
\end{definition}

A partial order $(S, \leq)$ can be visualized using a \textit{Hasse diagram}. In a Hasse diagram, each element of $S$ is represented by a node, and given $a,b \in S$ there is a line segment or curve going upward from $a$ to $b$ if $a \leq b$ and there is no element $c\in S$ such that $a \leq c \leq b$. The direction of the relation is implicitly understood to be upwards, so arrows are not needed. Also, if two elements of $S$ are incomparable under $\leq$ they are displayed at the same level in the diagram. An example is given in \cref{ex:hasse}.

\begin{example} \label{ex:hasse}
    In the example shown in \cref{fig:hasse_diagram_example}, the set is composed of the divisors of 12, and the relation is divisibility. An edge is drawn from $a$ to $b$ if $a$ divides $b$ and there is no other element $c$ in the set such that $a/c$ and $c/b$. For instance, there is an edge from 2 to 4 because 2 divides 4, and no other element in the set is a multiple of 2 and a divisor of 4. There is no direct edge from 2 to 12 because the relationship is captured transitively through other elements, such as $2/4/12$ or $2/6/12$.
    \begin{figure}[H]
        \centering
        \begin{tikzpicture}[node distance=1.5cm]
            \node (1) at (0,0) {1};
            \node (2) at (-1,1) {2};
            \node (3) at (1,1) {3};
            \node (4) at (-1,2) {4};
            \node (6) at (1,2) {6};
            \node (12) at (0,3) {12};

            \draw (1) -- (2);
            \draw (1) -- (3);
            \draw (2) -- (4);
            \draw (2) -- (6);
            \draw (3) -- (6);
            \draw (4) -- (12);
            \draw (6) -- (12);
        \end{tikzpicture}
        \caption{Hasse diagram for the set $\{1, 2, 3, 4, 6, 12\}$ with the "divides" relation.}
        \label{fig:hasse_diagram_example}
    \end{figure}
\end{example}

Now we can define the concept of width of a partial order.

\begin{definition}[Antichain]
    An antichain of a partial order $(S, \leq)$ is a subset of $S$ where any two distinct elements are incomparable. That is, for any two distinct elements $a, b$ in the antichain, neither $a \leq b$ nor $b \leq a$ holds.
\end{definition}

\begin{definition}[Width of a partial order, \cite{dilworth1990decomposition}]
    The width of a partial order $\leq$, denoted by $width(\leq)$, is the size of the largest possible antichain.
\end{definition}

By Dilworth's Theorem \cite{dilworth1990decomposition}, the width of a partially ordered set $(S, \leq)$ is equal to the cardinality of its largest antichain; this can be equivalently defined as the minimum number of chains needed to partition $S$, where each chain is a totally ordered subset of $S$ under the relation $\leq$. 

\subsection{Wheeler Graphs}
With the concept of co-lex order established, we can now define the class of graphs that form the starting point of our work. A Wheeler automaton is an automaton where the states can be arranged in a strict, total order.

\begin{definition}[Wheeler automaton, \cite{gagie2017wheeler}]
    \label{def:wheeler_automaton}
    A finite state automaton \\
    $\mathcal{A} = (Q, \Sigma, \delta, q_0, F)$ is a \textit{Wheeler automaton} if there exists a total order $\leq$ on its set of states $Q$ that satisfies the following axioms:
    \begin{enumerate}
        \item The initial state precede all other states in the order.
    \suspend{enumerate}
    For any two transitions $u \in \delta(u', a)$ and $v \in \delta(v', b)$:
    \resume{enumerate}[{}]
        \item $a<b \implies u \leq v$,
        \item $a=b \wedge u' < v' \implies u \leq v$.
    \end{enumerate}
    The order $\leq$ is called a \textit{Wheeler order}.
\end{definition}

Consequently, we define the concept of Wheeler language.
\begin{definition}[Wheeler language]
    A Wheeler language $L$ is a language accepted by a deterministic Wheeler automaton.
\end{definition}

The most important example for Wheeler automaton in this thesis is the trie. Any trie representing a finite language is a Wheeler automaton. The co-lexicographic order of the strings spelling the paths from the root to each node provides the required total ordering of the states. This is why tries are inherently indexable. However, this rigid structure also means they are uncompressed, as every unique path must be stored explicitly, even if it corresponds to a substring that appears many times in the language. Our work begins with this observation: we start with a Wheeler automaton (the trie) and seek to compress it while preserving efficient indexability.

\subsection{The Co-lex Width of an Automaton}
Now that we have the concepts of co-lex order and width, we can combine them to formally define the class of indexable automata that are central to this thesis. The width of the co-lex partial order on an automaton's states is the critical measure of its structural complexity from an indexing perspective.

The co-lex order can be extended to the set of states of an automaton. The idea of co-lex order on the states of an automaton was first introduced with the notion of Wheeler graphs by Gagie et al.~\cite{gagie2017wheeler} and was later generalized to arbitrary finite automata by Cotumaccio and Prezza~\cite{cotumaccio2021indexing}, where a partial order replaces the total order. 

Let $\lambda(q)$ denotes the set of labels of transitions entering in state $q$, and $\min\lambda(q)$ and $\max\lambda(q)$ represent the minimum and maximum element of the set, respectively. The definition of co-lex order on an automaton is as follows:
\begin{definition}[\cite{cotumaccio2023co}]
    \label{def:colex_order_on_automaton}
    Let $N = (Q, \Sigma, \delta, q_0, F)$ be an NFA. A co-lex order on $N$ is a partial order $\leq$ on $Q$ that satisfies the following two axioms:
    \begin{enumerate}
        \item For every $u, v \in Q$, if $u < v$, then $\max\lambda(u) < \min\lambda(v)$
        \item For every $a \in \Sigma$ and $u, v, u', v' \in Q$, if $u \in \delta(u', a)$, $v \in \delta(v', a)$ and $u < v$, then $u' \leq v'$
    \end{enumerate}
\end{definition}

The two axioms in \cref{def:colex_order_on_automaton} allow for states of a finite automaton to be compared. When $\leq$ is total, we say that the co-lex order is a Wheeler order (introduced in \cite{gagie2017wheeler} and \cref{def:wheeler_automaton}). 

Consequently, we can introduce the concept of co-lex width of an automaton.
\begin{definition}[\cite{cotumaccio2023co}]
    The co-lex width of an NFA N is the minimum width of a co-lex order on N, i.e.,
    $$
        width(N) = \min \{width(\leq)|\leq \text{ is a co-lex order on } N\}
    $$
\end{definition}

The requirement of a Wheeler order is powerful but restrictive. Many automata, especially those resulting from DAG compression, may not satisfy it (see \cref{ex:incomparability}). This introduces a fundamental trade-off: while DAG compression minimizes an automaton's size, it can destroy the very structure that enables efficient indexing. In fact, it has been shown that indexing general graphs—and thus, highly compressed automata—to support fast string matching is computationally expensive, as showed in \cite{equiGraphsCannotBe2023}. The second axiom of \cref{def:colex_order_on_automaton} does not always enforce an ordering between any two states, leading to a partial order instead of a total one. This gives rise to the more general notion of a \textit{$p$-sortable automaton}: 

\begin{definition}[$p$-sortable automaton] \label{def:p-sorable-automaton}
    Let $\mathcal{A} = (Q, \Sigma, \delta, q_0, F)$ be a finite-state automaton. We call $\mathcal{A}$ \emph{$p$-sortable} if there exists a co-lexicographic order $\leq$ on $Q$ such that $Q$ can be partitioned into $p$ chains $\{Q_i\}_{i=1}^p$, where each $(Q_i, \leq)$ is totally ordered.
\end{definition}

Under these definitions, a Wheeler automaton is a \textbf{1-sortable automaton}, as a total order has a width of 1 (the largest antichain is a single element).

\begin{example} \label{ex:incomparability}
    State incomparability can arise in several situations. For example, consider two states $u$ and $u'$.
    \begin{itemize}
        \item As illustrated in Figure~\ref{fig:incomparability-scenarios-combined}-(a), if there are two same-labeled transitions to the target states $u$ and $v$ from two incomparable states $u’$ and $v’$, then $u$ and $v$ are also incomparable.
        \item Conflicting constraints from different labels can force incomparability, as shown in Figure~\ref{fig:incomparability-scenarios-combined}-(b). An existing order on the sources of $a$-transitions (e.g., $u' < v'$) may require $u < v$ to satisfy the Wheeler axioms, while an order on the sources of $b$-transitions (e.g., $v'' < u''$) may require the opposite, $v < u$. Since both cannot be true, the targets $u$ and $v$ must be incomparable.
    \end{itemize}

    \begin{figure}[H]
    \centering
    \begin{subfigure}[b]{0.4\textwidth}
        \centering
        \begin{tikzpicture}[
            node distance=2.5cm, 
            on grid, 
            auto,
            state/.style={circle, draw, minimum size=1cm}
        ]
        % Scenario 1
        \node[state, initial, initial text=] (u') at (0,0) {$u'$};
        \node[state, accepting] (u) at (2.5,0) {$u$};
        \node[state, initial, initial text=] (v') at (0,-2.5) {$v'$};
        \node[state, accepting] (v) at (2.5,-2.5) {$v$};

        \path[->] (u') edge node {$a$} (u);
        \path[->] (v') edge node {$a$} (v);

        \node at (0, -3.5) {$u' \parallel v'$};
        \node at (2.5, -3.5) {$u \parallel v$};
        \end{tikzpicture}
        \caption{}
        \label{fig:incomparability-scenario1}
    \end{subfigure}
    \hfill
    \begin{subfigure}[b]{0.55\textwidth}
        \centering
        \begin{tikzpicture}[
            node distance=2.5cm, 
            on grid, 
            auto,
            state/.style={circle, draw, minimum size=1cm}
        ]
        % Scenario 2
        \node[state, initial, initial text=] (u'2) at (0,0) {$u'$};
        \node[state, accepting] (u2) at (2.5,0) {$u$};
        \node[state, initial, initial text=, initial where=right] (u''2) at (5,0) {$u''$};
        
        \node[state, initial, initial text=] (v'2) at (0,-2.5) {$v'$};
        \node[state, accepting] (v2) at (2.5,-2.5) {$v$};
        \node[state, initial, initial text=, initial where=right] (v''2) at (5,-2.5) {$v''$};
    
        \path[->] (u'2) edge node {$a$} (u2);
        \path[->] (u''2) edge node[above] {$b$} (u2);
        \path[->] (v'2) edge node {$a$} (v2);
        \path[->] (v''2) edge node[below] {$b$} (v2);

        \node at (0, -3.5) {$u' < v'$};
        \node at (2.5, -3.5) {$u \parallel v$};
        \node at (5, -3.5) {$v'' < u''$};
        \end{tikzpicture}
        \caption{}
        \label{fig:incomparability-scenario2}
    \end{subfigure}
    \caption{Examples of state incomparability in automata.}
    \label{fig:incomparability-scenarios-combined}
    \end{figure}
\end{example}

\begin{example} \label{ex:p-sortable}
    Consider the following DFA $D$ of \cref{fig:non_wheeler_example}-(a) and its partial co-lex order \cref{fig:non_wheeler_example}-(b). The DFA is not Wheeler because states $v_3$ and $v_5$ and states $v_4$ and $v_7$ are incomparable (as shown in the Hasse diagram). $D$ admits a partition into two chains of totally ordered states, for example one possible chain partition is given by:
    \begin{itemize}
        \item Chain 1: $v_1 \leq v_2 \leq v_3 \leq v_4$
        \item Chain 2: $v_5 \leq v_6 \leq v_7$
    \end{itemize}

    \begin{figure}[H]
        \centering
        \begin{subfigure}[b]{0.6\textwidth}
            \centering
            \begin{tikzpicture}[->, >=stealth, node distance=1.5cm, auto]
                \tikzset{
                    state/.style={circle, draw, minimum size=0.8cm},
                    accepting/.style={state, double}
                }

                \node[state, initial, initial text=] (v1) at (0,1) {$v_1$};
                \node[state] (v2) at (2,1) {$v_2$};
                \node[state] (v3) at (4,1) {$v_3$};
                \node[accepting] (v5) at (4,-1) {$v_5$};
                \node[accepting] (v6) at (6,1) {$v_6$};
                \node[state] (v7) at (8,1) {$v_7$};
                \node[accepting] (v4) at (8,-1) {$v_4$};

                \path (v1) edge node {a} (v2)
                      (v2) edge [bend left=30] node[above] {b} (v6)
                      (v6) edge [bend left=0] node[above] {a} (v3)
                      (v3) edge [bend left=20] node[right] {a} (v5)
                      (v5) edge [bend left=20] node[left] {a} (v3)
                      (v5) edge node[below] {b} (v7)
                      (v6) edge [bend right=0] node[above] {b} (v7)
                      (v7) edge [bend left=20] node[right] {b,c} (v4)
                      (v4) edge [bend left=20] node[left] {b} (v7);
            \end{tikzpicture}
            \caption{}
            \label{fig:non_wheeler_graph_example}
        \end{subfigure}%
        \hfill
        \begin{subfigure}[b]{0.35\textwidth}
            \centering
            \begin{tikzpicture}
                \node (v1) at (0,0) {$v_1$};
                \node (v2) at (0,1) {$v_2$};
                \node (v3) at (-0.75,2) {$v_3$};
                \node (v5) at (0.75,2) {$v_5$};
                \node (v6) at (0,3) {$v_6$};
                \node (v4) at (0.75,4) {$v_4$};
                \node (v7) at (-0.75,4) {$v_7$};

                \draw (v1) -- (v2);
                \draw (v2) -- (v3);
                \draw (v2) -- (v5);
                \draw (v3) -- (v6);
                \draw (v5) -- (v6);
                \draw (v6) -- (v4);
                \draw (v6) -- (v7);
            \end{tikzpicture}
            \caption{}
            \label{fig:non_wheeler_graph_poset}
        \end{subfigure}
        \caption{An example of a $2$-sortable DFA and its corresponding Hasse diagram of the partial order.}
        \label{fig:non_wheeler_example}
    \end{figure}
\end{example}

We now introduce another important result by \cite{manziniRationalConstructionWheeler2024}. Let $D = (Q, \Sigma, \delta, s, F)$ denote the minimal DFA accepting a Wheeler language $L$, and let $D^w = (Q^w, \Sigma, \delta^w, s^w, F^w)$ denote the minimal Wheeler DFA (WDFA) accepting $L$, i.e.\ the DFA with the minimum number of states among all Wheeler DFAs accepting $L$. Since $D^w$ is Wheeler for any two distinct states $q, q' \in Q^w$, the associated intervals $I_q, I_{q'}$ are disjoint. This property does not generally hold for $D$, where states may correspond to overlapping sets of prefixes. As a consequence, when transforming $D$ into $D^w$, certain states of $D$ may need to be \emph{split} into several states in $D^w$, potentially leading to an exponential blow-up in the number of states.

\begin{example}[\cite{manziniRationalConstructionWheeler2024}]
    We now provide a simple example of an automaton $D$ with width $n$ that accepts a Wheeler language, yet its minimum equivalent Wheeler DFA, $D^w$, is exponentially larger. Let $D$ be the automaton depicted in Figure~\ref{fig:exponential-gap-dfa}.

    \begin{figure}[H]
    \centering
    \begin{tikzpicture}[
        node distance=1.5cm and 2cm,
        auto,
        state/.style={circle, draw, minimum size=0.8cm, inner sep=0pt}
    ]
        \node[state, initial, initial text=] (start) at (0,0) {$V_0$};
        
        \node[state] (w1_1) at (1.5,1.3) {$V_1$};
        \node[state] (w1_2) at (1.5,-1.3) {$V_2$};
        \node[state] (q1) at (3,0) {$V_3$};
        %\node at (3, -0.7) {$q_1$};

        \path[->] (start) edge node[above] {a} (w1_1);
        \path[->] (start) edge node[below] {b} (w1_2);
        \path[->] (w1_1) edge node[above] {c} (q1);
        \path[->] (w1_2) edge node[below] {c} (q1);

        \node[state] (w2_1) at (4.5,1.3) {};
        \node[state] (w2_2) at (4.5,-1.3) {};
        \node[state] (q2) at (6,0) {};
        %\node at (6, -0.7) {$q_2$};

        \path[->] (q1) edge node[above] {a} (w2_1);
        \path[->] (q1) edge node[below] {b} (w2_2);
        \path[->] (w2_1) edge node[above] {c} (q2);
        \path[->] (w2_2) edge node[below] {c} (q2);

        \node at (7.5, 0) {$\cdots\cdots\cdots$};

        \node[state] (qn) at (9,0) {};
        \node[state] (wn_1) at (10.5,1.3) {};
        \node[state] (wn_2) at (10.5,-1.3) {};
        \node[state, accepting] (t) at (12,0) {$t$};
        %\node at (9, -0.7) {$q_{n-1}$};
        %\node at (12, -0.7) {$q_n$};

        \path[->] (qn) edge node[above] {a} (wn_1);
        \path[->] (qn) edge node[below] {b} (wn_2);
        \path[->] (wn_1) edge node[above] {c} (t);
        \path[->] (wn_2) edge node[below] {c} (t);

    \end{tikzpicture}
    \caption{A DFA accepting a finite (and thus Wheeler) language, for which the minimal equivalent Wheeler DFA is exponentially larger.}
    \label{fig:exponential-gap-dfa}
    \end{figure}

    The language $L = \mathcal{L}(D)$, being finite, is a Wheeler language \cite{alanko2021wheeler}. However, any Wheeler automaton accepting $L$ must have a number of states that is exponential in $n$. Since there are exponentially many such pairwise distinct strings leading to $t$, a Wheeler automaton must partition the set $I_t$ into an exponential number of sub-intervals. This forces the state $t$ to be ``split'' into exponentially many copies, leading to an exponential blow-up in the size of the minimal Wheeler DFA.
\end{example}

The previous example highlights a crucial trade-off: enforcing the strict ordering of a Wheeler DFA can lead to an exponential increase in the number of states compared to a minimal DFA. 

\begin{comment}
\cref{thm:exp_increase} formalizes this observation, showing that the size of a minimal Wheeler DFA cannot be bounded by a polynomial in the size of the minimal DFA.
\begin{theorem}[\cite{manziniRationalConstructionWheeler2024}, Theorem 29] \label{thm:exp_increase}
    Let $L = \mathcal{L}(D) = \mathcal{L}(D^w)$, where $L$ is Wheeler, $D$ is minimal, $D^w$ is minimal Wheeler, and let $f(\cdot,\cdot)$ be such that \\
    $|D^w| = \mathcal{O}(f(|D|,width(D)))$. Then, for any $k,p \in \mathbb{N}$, $f(n,p) \notin \mathcal{O}(n^k + 2^p)$.
\end{theorem}
\end{comment}

Since this work aims to transform a trie (a 1-sortable automaton) into a more general $p$-sortable graph (with width $p>1$) in a way that introduces DAG compression while maintaining efficient indexability, it is motivated by the powerful result of \cref{thm:exp_increase} leading to the fact that even a small increase in sortability (for example from $p=1$ to $p=2$) can yield exponential compression. This highlights the potential of exploring the trade-off between sortability and size, which is the central theme of this thesis.

\subsection{Indexing Finite State Automata} \label{sec:indexing}
Now we introduce the current state of the art in indexing finite state automata. In 2023, Cotumaccio et al. \cite{cotumaccio2021indexing} introduced a compressed data structure for automata whose performance and space complexity are directly tied to the automaton's co-lex width $p$. This structure generalize the famous Burros-Wheeler transform \cite{burrows1994block}. 

\begin{theorem}[Cotumaccio et\ al.\ \cite{cotumaccio2021indexing}]
    \label{thm:indexing}
    Let $A$ be a $p$-sortable automaton. There exists a compressed data structure for $A$ that supports subpath queries (\cref{def:tree_operations}) on a query word $\alpha$ of length $m$ in $O(mp^2\log(p|\Sigma|))$ time. The space required is:
    \begin{itemize}
        \item $\log(|\Sigma|) + \log p + 2$ bits per edge for DFAs
        \item $\log(|\Sigma|) + 2\log p + 2$ bits per edge for NFAs
    \end{itemize}
\end{theorem}
This highlights a direct trade-off: both query time and space per edge depend on the width parameter $p$, which governs the automaton's compressibility.

To highlight the importance of this data structure, we recall that the final output of our compression pipeline is a $p$-sortable DAG compressed automaton with a controlled co-lex width $p$. This allows us to leverage these advanced indexing capabilities on the compressed automata produced by our method.
\chapter{Min-Weight Perfect Bipartite Matching} \label{chp:min_weight_perfect_bipartite_matching}
\alessio{Ricorda sempre un cappello introduttivo ad ogni capitolo}

\section{Problem definition}
Given a weighted bipartite graph $G = (V, E)$ \draft{(remember that a bipartite graph is a graph whose vertices can be divided into two disjoint sets $V_1$ and $V_2$ such that every edge connects a vertex in $V_1$ to a vertex in $V_2$)}, let's define the concept of a matching. \alessio{Non hai mai spiegato cos'è un grafo bipartito, quindi non posso ricordarlo :)}
\begin{definition}[Matching] \label{def:matching}
    \draft{Given a generic graph $G = (V, E)$, }A matching $M \draft{\sout{\in}} \draft{\subseteq} E$ is a collection of edges such that every vertex of $V$ is incident to at most one edge of $M$. \draft{\sout{In other words, a matching is a set of edges such that no two edges share a common vertex}}
\end{definition}
\draft{In other words, a matching is a set of edges such that no two edges share a common vertex}
If a vertex $v$ has no edge of $M$ incident to it then $v$ is said to be exposed (or unmatched). A matching is perfect if no vertex is exposed; in other words, a matching is perfect if its cardinality is equal to $|V_1| = |V_2|$ \cite{goemans2009matching}.

\begin{figure}[H]
    \centering
    \includegraphics[width=0.6\textwidth]{Immagini/matching_example.png}
    \caption{Example of a \draft{perfect matching} in a bipartite graph. \alessio{Però ci sono dei nodi exposed, quindi non è perfect. Farei 3 subfigure: un non matching, un matching non perfect, un matching perfect. La legenda la puoi spiegare poi nella caption.}}
    \label{fig:matching_example}
\end{figure}

The problem of finding a minimum weight perfect matching in a bipartite graph is a well-known problem in combinatorial optimization. The problem can be formulated as follows: 

\begin{definition}[Minimum weight perfect matching in bipartite graphs (\textsc{MWPBM})] \label{def:mwpbm}
    Given a weighted bipartite graph $G = (V, E)$, \draft{where $V = V_1 \cup V_2$ and $V_1 \cap V_2 = \emptyset$}, find a perfect matching $M$ such that the sum of the weights of the edges in $M$ is minimized. \draft{The weight of a matching is the sum of the weights of the edges in the matching. The weight of an edge $e = (u, v)$ is denoted by $w(e)$. This problem is also called \textbf{the assignment problem}}.
\end{definition}
\alessio{La parte sui nodi $V_1$ e $V_2$ è nella definizione di grafo bipartito, per cui spostiamola nella definizione.}
\alessio{La parte sui pesi si può spiegare fuori dalla definizione, dato che non è relativa al problema ma al grafo. Occhio che qua usi $e$ per indicare un arco, ma prima usavi $e$ per indicare il numero di archi in un grafo. $e$ ci sta meglio qua a mio parere, prima scegli se usare $m$ (se è un simbolo libero) o direttamente $|E|$}

\section{The existence of perfect matchings in bipartite graphs}
In this section \draft{we introduce} two theorems that state\draft{\sout{s}} a condition for the existence of perfect matchings in bipartite graphs \draft{\sout{are introduced}}. These theorems will be useful in the following chapter to proof our reduction \cite{viswanath2004perfect}. \alessio{Specifica se la condizione è sufficiente e/o necessaria}

\subsection{The Tutte matrix and its determinant}
Let's start with the definition of the \textbf{Tutte matrix} of a bipartite graph.
\begin{definition}[Tutte matrix] \label{def:tutte_matrix}
    The Tutte matrix of bipartite graph $G = (U, V, E)$ is an $n \times n$ matrix $M$ with the entry at row $i$ and column $j$
    \begin{equation}
        M_{i,j} =
        \begin{cases}
            0 & \text{if } (u_i, u_j) \notin E \\
            x_{i,j} & \text{if } (u_i, u_j) \in E
        \end{cases}
    \end{equation}
    \alessio{Cos'é $U$ nella definizione del grafo? Prima hai usato $V_1$ e $V_2$ per le due componenti, mantieni la stessa notazione (o cambiala prima, ma forse è più bella con i pedici).}
    \alessio{Posta così, $u_i$ e $u_j$ sembrano appartenere solo a $U$. In ogni caso, la Tutte matrix si può calcolare per un qualsiasi grafo, quindi lascerei la definizione generica. Altrimenti se vuoi stare sul grafo bipartito, bisognerebbe usare la Edmonds matrix (e in quel caso usare $U$ e $V$)}
    \alessio{Dovresti specificare che i vari $x_{i,j}$ sono ``indeterminates''}

\end{definition}

The determinant of the Tutte matrix is useful in testing whether a graph has a perfect matching or not, as the following theorem introduced in \cite{lovasz1979matching} shows. 

\begin{theorem}[Existence of perfect matchings in bipartite graphs \cite{lovasz1979matching}] \label {thm:perfect_matching_existence}
    Given a bipartite graph $G$ and the Tutte matrix $M$ for $G$ then the following equivalence holds:
    $$
    Det(M) \neq 0 \iff \text{There exists a perfect matching in G}
    $$
\end{theorem}

\begin{proof}
    We have the following expression for the determinant, \draft{also called Leibniz formula}:

    $$
    \text{Det}(M) = \sum_{\pi \in S_n} (-1)^{sgn(\pi)} \prod_{i=1}^{n} M_{i,\pi(i)}
    $$
    \alessio{Cosa ritorna sgn? 0 o 1? In caso puoi fargli ritornare -1 e 1 e non ti serve l'esponente.}

    where $S_n$ is the set of all permutations on $[n]$, and $sgn(\pi)$ is the sign of the permutation $\pi$. \alessio{Definisci il segno di una permutazione. Prima indicavi le permutazioni con $\Pi$, ora con $\pi$. Immagino che sia per la moltiplicatoria, valuta se usare $\pi$ anche prima.}
    There is a one-to-one correspondence between a permutation $\pi \in S_n$ and a (possible) perfect matching 

    $$
    \{(u_1, v_{\pi(1)}), (u_2, v_{\pi(2)}), \cdots , (u_n, v_{\pi(n)})\} \text{ in } G.
    $$

    Note that if this perfect matching does not exist in $G$ (i.e., some edge $(u_i, v_{\pi(i)}) \notin E$), then the term corresponding to $\pi$ in the summation is $0$. So we have

    $$
    \text{Det}(M) = \sum_{\pi \in P} (-1)^{sgn(\pi)} \prod_{i=1}^{n} x_{i,\pi(i)}
    $$

    where $P$ is the set of perfect matchings in $G$. This is clearly zero if $P = \emptyset$, i.e., if $G$ has no perfect matching. If $G$ has a perfect matching, there is \draft{at least} a $\pi \in P$ and the term corresponding to $\pi$ is

    $$
    \prod_{i=1}^{n} x_{i,\pi(i)} \neq 0.
    $$

    Additionally, there is no other term in the summation that contains the same set of variables. Therefore, this term is not cancelled by any other term. So in this case, $\text{Det}(M) \neq 0$.
\end{proof}

\subsection{\draft{\sout{The}} Hall's Marriage Theorem}
Hall's Marriage Theorem \cite{hall1935representatives} provides a necessary and sufficient condition for the existence of a matching in a bipartite graph that saturates one side of the partition. It's often stated in the context of finding pairings (like marriages) between two sets of entities.
    
\begin{definition}[Neighborhood] \label{def:neighborhood}
For a subset of vertices $W \subseteq V_1$, the \textbf{neighborhood} of $W$, denoted by $N(W)$, is the \draft{sub}set of all vertices in $V_2$ that are adjacent to at least one vertex in $W$.
\[ N(W) = \{ v \in V_2 \mid \exists u \in W \text{ \draft{\sout{such that} $\land$} } \{u, v\} \in E \} \]
\end{definition}

\begin{theorem}[Hall's Marriage Theorem \cite{hall1935representatives}] \label{thm:halls_marriage_theorem}
Let \draft{$G = (V_1 \cup V_2, E)$ be a bipartite graph}. \alessio{Adatta la notazione a quella che hai scelto prima}.
There exists a perfect matching $M$ in $G$ if and only if for every subset $W \subseteq V_1$, the following condition holds:
\[ |N(W)| \geq |W| \]
\draft{\sout{This condition is known as \textbf{Hall's condition}.}}
\end{theorem}
\draft{This condition is known as \textbf{Hall's condition}.}

In simpler terms, a matching that covers all vertices in $V_1$ exists if and only if every group of vertices chosen from $V_1$ collectively has at least as many neighbors in $V_2$ as there are vertices in the chosen group.

\section{Problem formulation}
The problem of finding a minimum weight perfect matching in a bipartite graph can be formulated as an integer linear program (ILP), i.e. an optimization problem in which the variables are restricted to integer values, and the constraints and the objective function are linear as a function of these variables. Given a matching $M$, let $x$ be its incidence vector where $x_{ij} = 1$ if edge $(i, j)$ is in the matching, and $x_{ij} = 0$ otherwise. Then, the problem can be formulated as follows:

\begin{equation}
    \begin{aligned}
        \text{minimize} \quad & \sum_{(i, j) \in E} w_{ij} x_{ij} \\
        \text{subject to} \quad & \sum_{j \in V_2} x_{ij} = 1, \quad \forall i \in V_1 \\
        & \sum_{i \in V_1} x_{ij} = 1, \quad \forall j \in V_2 \\
        & x_{ij} \in \{0, 1\}, \quad \forall (i, j) \in E
    \end{aligned}
\end{equation}

Notice that any solution to this integer program corresponds to a matching and therefore this is a valid formulation of the minimum weight perfect matching problem in bipartite graphs.

The linear program relaxation of the above integer program is as follows:

\begin{equation}
    \begin{aligned}
        \text{minimize} \quad & \sum_{(i, j) \in E} w_{ij} x_{ij} \\
        \text{subject to} \quad & \sum_{j \in V_2} x_{ij} = 1, \quad \forall i \in V_1 \\
        & \sum_{i \in V_1} x_{ij} = 1, \quad \forall j \in V_2 \\
        & 0 \leq x_{ij} \leq 1, \quad \forall (i, j) \in E
    \end{aligned}
\end{equation}

The set of feasible solutions to the constraints in \draft{(P)} \alessio{Cos'é (P)?} forms a polytope. When optimizing a linear constraint over a polytope, the optimum will be achieved at one of the ``corner'' or extreme points of the polytope. An extreme point $x$ of a set $Q$ is an element $x \in Q$ that cannot be expressed as $\lambda y + (1 - \lambda) z$ with $0 < \lambda < 1$, $y, z \in Q$, and $y \neq z$. (This concept will be formalized and discussed in more detail when we cover polyhedral theory \alessio{Fai una ref a dove ne parli})

In general, even if all the coefficients of the constraint matrix in a linear program are either 0 or 1, the extreme points of a linear program are not guaranteed to all have integral coordinates. This is not surprising since the general integer programming problem is NP-hard, while linear programming is solvable in polynomial time. Consequently, there is no guarantee that the value $Z_{IP}$ of an integer program is equal to the value $Z_{LP}$ of its LP relaxation. However, since the integer program is more constrained than the relaxation, we always have $Z_{IP} \geq Z_{LP}$, implying that $Z_{LP}$ is a lower bound on $Z_{IP}$ for a minimization problem. Moreover, if an optimal solution to a linear programming relaxation is integral, then it must also be an optimal solution to the integer program.

In our problem, the constraint matrix has a special form that leads to the following result: 

\begin{theorem}
    Any extreme point of ($P$) is a $0-1$ vector, hence, it is the incidence vector of a perfect matching.
\end{theorem}

Consequently, the polytope

\begin{equation}
    \begin{aligned}
        P = \{ x: & \sum_{j \in V_2} x_{ij} = 1, \quad \forall i \in V_1, \\
        & \sum_{i \in V_1} x_{ij} = 1, \quad \forall j \in V_2, \\
        & 0 \leq x_{ij} \leq 1, \quad \forall (i, j) \in E \}
    \end{aligned}
\end{equation}

is called the bipartite perfect matching polytope. 

\section{Solutions to the problem}
There are several algorithms to solve the problem of finding a minimum weight perfect matching in a bipartite graph. The first algorithm to solve this problem was proposed by Kuhn in 1955 \cite{kuhn1955hungarian}. The algorithm is based on the Hungarian method, which is a combinatorial optimization algorithm that solves the assignment problem in polynomial time. In the original paper the complexity of the algorithm was $O(n^4)$, but later Dinic and Kronrod \cite{dinic1969algorithm} showed that the algorithm can be implemented in $O(n^3)$ time.

The Hungarian method is a powerful algorithm, however, \draft{\sout{the algorithm} it} is not very intuitive and can be difficult to implement. In recent years, several other algorithms have been proposed to solve \draft{\sout{the problem of finding a minimum weight perfect matching in a bipartite graph} this problem}. In 1970, Edmonds and Karp \cite{edmonds1972theoretical} proposed an algorithm that solves the problem in $O(nm + n^2 \log n)$ time. In 1989 Gabow and Tarjan \cite{gabow1989faster} proposed an algorithm that solves the problem in $O(\sqrt{n}m \log(nW))$ time,  where $n,m$ and $W$ denote the number of vertices, number of edges, and largest magnitude of a cost; costs are assumed to be integral. \draft{The algorithms work by scaling}. Lastly, in 2009, Sankowski and Piotr \cite{sankowski2009maximum} introduced a randomized algorithm that solves the problem in $O(Wn^w)$ time, where $w$ is the exponent of matrix multiplication, and $W$ is the highest edge weight in the graph.

In 2022, Chen, Li, et al. \cite{chen2022maximum} proposed a new solution to the Minimum-Cost Flow problem that works in almost-linear time, precisely in $O(m^{1+o(1)})$ time. The minimum-cost flow problem is a classic combinatorial graph problem that find\draft{s} numerous applications in engineering and scientific computing. This result is important also for our problem, since the maximum weight perfect matching problem can be reduced to the minimum-cost flow problem, allowing to solve the problem in almost-linear time.

\section{Implementation used in this work}
\davide{Da completare una volta deciso come procedere}

In this section, we will present an implementation of the Gabow and Tarjan algorithm to solve the problem of finding a minimum weight perfect matching in a bipartite graph. The algorithm is based on scaling and is a generalization of the Hungarian method. The algorithm works by scaling the edge weights and then finding a perfect matching in the scaled graph. 

\newpage
\
\newpage
\chapter{Tree Compression Scheme} \label{chp:project_overview}
As introduced in the first chapter, the primary goal of this thesis is to develop a novel tree compression scheme that effectively leverages repetitive structures within the input trie. The proposed algorithm is designed to identify and compactly represent these recurring patterns, thereby improving compression performance, particularly for highly repetitive tries. This chapter provides an overview of the proposed compression scheme.

\section{Compression Scheme Pipeline}
The overall pipeline of our proposed method is outlined in Algorithm~\ref{alg:pipeline}. It takes an input trie $T$ and a width parameter $p$ and produces a compressed, $p$-sortable automaton.
\begin{algorithm}[H]
\caption{$CompressTrie(T,p)$}
\label{alg:pipeline}
\begin{algorithmic}[1]
\Require Input trie $T$, width integer parameter $p$
\Ensure A compressed, $p$-sortable automaton $\mathcal{A}$
    \State $V_{sorted} \gets \text{PathSort}(T)$ 
    \State $N[1,\dots,|Q|] \gets \text{ComputeEquivalenceClasses}(T)$
    \State $G_{bipartite} \gets \text{ConstructMWPBMInstance}(V_{sorted}, N, p)$
    \State $M \gets \text{SolveMWPBM}(G_{bipartite})$ 
    \State $C[1,\dots,p] \gets \text{ExtractChainsFromMatching}(M, V_{sorted})$ 
    \State $\mathcal{A} \gets \text{CollapseChains}(C, N)$ 
    \State \Return $\mathcal{A}$
\end{algorithmic}
\end{algorithm}

The first step of the pipeline (line 2) is to establish a total order on the nodes of the trie. This is achieved by sorting the nodes co-lexicographically using the \textbf{Path Sort} algorithm, which we detail in \cref{sec:pathSort}. This sorting is fundamental, as it arranges the nodes in the order required for a Wheeler automaton.

Next, the algorithm identifies which nodes are candidates for merging (line 3). This is done by partitioning the nodes into equivalence classes based on the structure of the subtrees rooted at each node. Two nodes are in the same class if and only if their subtrees are isomorphic. This is equivalent to computing the Myhill-Nerode equivalence classes for the finite language accepted by the trie, a process we adapt from Revuz's algorithm for minimizing acyclic DFAs (\cref{sec:revuz}).

The core of the algorithm lies in lines 5-7, where the problem of optimally partitioning the sorted nodes into $p$ chains is solved. As detailed in \cref{cha:mwpm}, we reduce this problem to finding a Minimum Weight Perfect Bipartite Matching (MWPBM). A bipartite graph is constructed where the weights on the edges correspond to the "cost" of placing two nodes in the same chain. By finding a perfect matching with minimum weight using standard algorithms (\cref{sec:mwpbm_solutions}), we can reconstruct a set of $p$ chains that minimizes the total number of runs (\cref{def:run}), thereby maximizing compression.

Finally, the compressed automaton is constructed (line 8). The algorithm iterates through each of the $p$ chains and "collapses" any consecutive sequence of nodes belonging to the same equivalence class into a single state. This run-length compression, which we describe in \cref{sec:collapsing}, produces the final $p$-sortable automaton, which can then be indexed for efficient querying (\cref{chp:indexing}).

\begin{example} \label{ex:string_example}
    In our running example, we begin with the tree shown in \cref{fig:original_tree}. Each node in this tree is labeled with its corresponding Myhill-Nerode equivalence class, as determined in \cref{ex:ADFA_minimization}. By traversing the nodes in co-lexicographic order, we construct a string $S$ where each character represents the Myhill-Nerode equivalence class of a node:
    \[
        S = \text{ABCDDCBDDDD}
    \]
    This string $S$ then becomes the input to the \textsc{String Partitioning Problem} (see \cref{sec:string_partitioning}), where the objective is to partition it into $p$ subsequences while minimizing the total number of runs.

    % First figure - Original tree
    \begin{figure}[H]
        \centering
        \begin{tikzpicture}[
            level distance=1.5cm,
            sibling distance=3cm,
            state/.style={circle, draw, minimum size=7mm},
            accepting/.style={circle, draw, double, minimum size=7mm},
            edge from parent/.style={draw, -latex},
            level 1/.style={sibling distance=4cm},
            level 2/.style={sibling distance=2.5cm},
            level 3/.style={sibling distance=2cm}
            ]
        
        \node[state] (a) {A}
            child {node[state] (b) {B} 
            child {node[state] (d) {C}
                child {node[accepting] (h) {D}}
                child {node[accepting] (i) {D}}
            }
            child {node[accepting] (e) {D}}
            }
            child {node[state] (c) {B}
            child {node[state] (f) {C}
                child {node[accepting] (l) {D}}
                child {node[accepting] (m) {D}}
            }
            child {node[accepting] (g) {D}}
            };
        \end{tikzpicture}
        \caption{Tree ADFA of \cref{fig:example_ADFA}. Each node is labeled with its equivalence class.}
        \label{fig:original_tree}
    \end{figure}
\end{example}
\section{Reducing Chains-Division Problem to the Assignment Problem}
In this section, we will show how we can reduce the problem of finding the optimal partition of the nodes of a tree $T$ given their equivalence classes into $p$ with $p \leq |E| \leq t$ chains to the Minimum Weight Perfect Bipartite Matching problem, where $E$ is the set of equivalence classes of the nodes of $T$ and $t$ is the number of nodes of the tree. This reduction will allow us to solve the problem in polynomial time as shown in the previous chapter.

Then we will show how to optimize the reduction by introducing some constraints that will allow us to reduce the number of edges in the bipartite graph, and we will also show how to move from the Minimum Weight Perfect Bipartite Matching problem to the more studied Maximum Weight Perfect Bipartite Matching problem without losing generality.

\subsection{Chains-Division problem definition}
It is important to start by defining the problem we want to solve.

\begin{definition}[\textsc{CHAINS-DIVISION} problem] \label{def:problem_def}
    Given a tree $T$ with $t$ nodes, and given the equivalence classes $E$ coming from the Hopcroft algorithm applied to the tree $T$, the sorted order of the nodes in $T$ according to the upward path $\pi$, and the number of chain $2 \leq p \leq t$, we want to find the optimal partition of the nodes of $T$ into $p$ chains such that the run length encoding of each chain is minimized.
\end{definition}

Let's give a formal definition of run length encoding.
\begin{definition}[Run length encoding]
    Given a sequence $S = \{s_1, s_2, \dots, s_n\}$, the run length encoding of $S$ is the sequence $R = \{r_1, r_2, \dots, r_m\}$ where $r_i$ is the number of times the element $s_i$ is repeated in $S$. It allows us to represent the sequence $S$ in a more compact way.
\end{definition}

So, we aim to divide the nodes of the tree into $p$ chains such that the run length encoding of the chains is minimized meaning that we want to minimize the number of distinct equivalence classes in each chain. Follows the definition of chain.

\begin{definition}[Chains] \label{def:chains}
    Given a tree $T$ with $t$ nodes and $p$ chains, a chain $C$ is a sequence of nodes $C = \{c_1, c_2, \dots, c_m\}$ such that $c_i$ is a node of $T$ for $i = 1, 2, \dots, m$ and $m \leq t$. Also, Each node of $T$ is in exactly one chain and the nodes of the chain are ordered according to the upward path $\pi$ of the tree.
\end{definition}

\subsection{Bipartite graph construction}
Now we will show how to construct a bipartite graph that will allow us to solve the problem of finding the optimal partition of the nodes of a tree $T$ given their equivalence classes into $p$ with $p \leq |E| \leq t$ chains.

Let $T$ be a tree with $t$ nodes and $p$, which is the number of chains we want to partition the nodes into. Let $E$ be the set of equivalence classes of the nodes of $T$. We can construct a bipartite graph $G = (V, E)$ such that vertices are divided in two disjoint sets $V = V_1 \cup V_2$ in the following way:

\begin{definition} \label{def:bip_construction}
    The two sets $V_1$ and $V_2$ of the bipartite graph $G$ are constructed in the following way:
    \begin{itemize}
        \item $V_1$ contains $t + p$ nodes composed by $p$ source nodes $s_1, s_2, \dots, s_p$ and the $t$ elements of $E$ ordered.
        \item $V_2$ contains $t + p$ nodes composed by the $t$ elements of $E$ ordered and $p$ destination nodes $d_1, d_2, \dots, d_p$.
    \end{itemize}

    Then the edges of the graph $G$ will be constructed in the following way:
    \begin{enumerate}
        \item The sources nodes $s_i \in V_1$ for $i = 1, 2, \dots, p$ are connected to the first $p$ nodes with distinct equivalence class in $V_2$ with weight $1$.
        \item Each of the $t$ nodes of the tree in $V_1$ is connected to the first $p$ (at most) nodes with distinct class in $V_2$ (and without the same class of the considered node) coming after it in the ordering of the nodes  with weight $1$.
        \item Each of the $t$ nodes of the tree in $V_1$ is also connected to the first node with its same class in $V_2$ coming after it in the ordering of the nodes of $t$ with weight $0$ iff there is one, otherwise $p$ edges with weight $0$ are added to each of the destination nodes $d_i \in V_2$ for $i = 1, 2, \dots, p$.
    \end{enumerate}
\end{definition}

Notice that it is important to consider the order of the nodes of the two sets $V_1$ and $V_2$ as stated above, because we will need to connect the source nodes to the destination nodes in a way that will allow us to find the optimal partition of the nodes of the tree. In Figure \ref{fig:reduction_example} the nodes are ordered from top to bottom. An example of the node structure is shown in Figure \ref{fig:reduction_example_parts}.

Notice also that when we talk about the same $V_1$ node placed in $V_2$ we are referring to the corresponding node in $V_2$ that derives from the same node in the original tree $T$ since the nodes of the tree are placed ordered in both sets $V_1$ and $V_2$. In Figure \ref{fig:reduction_example_parts}, \ref{fig:reduction_example} and \ref{fig:reduction_small_examples} the node's correspondence is achieved by putting the two nodes at the same level.

\begin{figure}[H]
    \centering
    \includegraphics[width=0.3\textwidth]{Immagini/bipartite_keys_part.png}
    \caption[Bipartite nodes structure]{Example of a bipartite graph nodes constructed from a tree with the following equivalence classes $E = \{1,2,1,3,1,2,2\}$. The nodes are ordered from top to bottom.}
    \label{fig:reduction_example_parts}
\end{figure}

\begin{definition}[Bipartite graph properties]
    The resulting bipartite graph $G$ will have $2t + 2p$ nodes and $O(t (p + 1) + p^2 + tp)$ edges, where the $O(t (p + 1))$ edges come from the tree nodes, the $O(p^2)$ edges come from the sources since each source node is connected to $p$ nodes, and the $O(tp)$ edges come from the destination nodes since in the worst case we have $t$ distinct equivalence class and so all the nodes are connected to the destination nodes. The weight of the edges will be $0$ or $1$.
\end{definition}

Let's see a small example for each case, consider $p=2$. In Figure \ref{fig:reduction_small_examples}-(a) there is an example for the sources' edges, as stated before, for each source $p$ nodes with weight $1$ are created and connected to the first $p$ nodes with distinct equivalence class in $V_2$.

In Figure \ref{fig:reduction_small_examples}-(b) there is an example for the tree nodes' edges, for each node in the tree $T$ edges with weight $1$ are created and connected to the first $p$ nodes with distinct equivalence class in $V_2$ after the corresponding node in $V_2$ (coming after the node itself in the ordering), and edges with weight $0$ are created and connected to the first node with the same class in $V_2$ after the corresponding node in $V_2$. As we can see from the image, we consider the first node in $V_1$ labelled $1$ that is connected to the node $2$ with weight $1$ and to the node $3$ with weight $1$, and to the second node labelled $1$ in $V_2$ with weight $0$.

Lastly, in Figure \ref{fig:reduction_small_examples}-(c) there is an example for the destination nodes' edges, we start by considering the first node in $V_1$ that is labelled $1$, it is connected to the node $2$ with weight $1$, then since there is no node with the same class in $V_2$ we connect it to the destination nodes $d_1$ and $d_2$ with weight $0$. The same is done for the second node in $V_1$ that is labelled $2$ since no nodes are coming after it in the order it is connected to the destination nodes $d_1$ and $d_2$ with weight $0$.

\begin{figure}[H]
    \centering
    \begin{tabular}{ccc}
        \begin{tikzpicture}[node distance={10mm}, thick, auto=center, main/.style = {draw, circle}]
            \node[main] (1s) {$s_1$};
            \node[main] (2s) [below of=1s] {$s_2$};
            \node[main] (3s) [below of=2s] {$1$};
            \node[main] (4s) [below of=3s] {$1$};
            \node[main] (5s) [below of=4s] {$2$};

            \node[main] (1d) [right=3cm of 3s] {$1$};
            \node[main] (2d) [below of=1d] {$1$};
            \node[main] (3d) [below of=2d] {$2$};

            \draw[red, ->] (1s) -- (1d);
            \draw[red, ->] (1s) -- (3d);
            \draw[red, ->] (2s) -- (1d);
            \draw[red, ->] (2s) -- (3d);
        \end{tikzpicture} &
        \begin{tikzpicture}[node distance={10mm}, thick, auto=center, main/.style = {draw, circle}]
            \node[main] (3s) [below of=2s] {$1$};
            \node[main] (4s) [below of=3s] {$2$};
            \node[main] (5s) [below of=4s] {$1$};
            \node[main] (6s) [below of=5s] {$3$};
            \node[main] (1d) [right=3cm of 3s] {$1$};
            \node[main] (2d) [below of=1d] {$2$};
            \node[main] (3d) [below of=2d] {$1$};
            \node[main] (4d) [below of=3d] {$3$};

            \draw[red, ->] (3s) -- (2d);
            \draw[red, ->] (3s) -- (4d);
            \draw[green, ->] (3s) -- (3d);
        \end{tikzpicture} &
        \begin{tikzpicture}[node distance={10mm}, thick, auto=center, main/.style = {draw, circle}]
            \node[main] (7s) [below of=6s] {$1$};
            \node[main] (9s) [below of=7s] {$2$};
            \node[main] (5d) [below of=4d] {$1$};
            \node[main] (6d) [below of=5d] {$2$};
            \node[main] (8d) [below of=6d] {$d_1$};
            \node[main] (9d) [below of=8d] {$d_2$};

            \draw[red, ->] (7s) -- (6d);
            \draw[green, ->] (7s) -- (8d);
            \draw[green, ->] (7s) -- (9d);
            \draw[green, ->] (9s) -- (8d);
            \draw[green, ->] (9s) -- (9d);
        \end{tikzpicture} \\
    (a) & (b) & (c) \\
    \end{tabular}
    \caption[Reduction cases examples]{Considering $p=2$ these three examples show how to connect nodes in the bipartite graph in the case of sources (a), tree nodes (b), and destinations (c).}
    \label{fig:reduction_small_examples}
\end{figure}

\subsection{Proof of correctness}
Let's start by stating the following lemmas.

\begin{lemma} \label{lemma:all_destinations}
    Exactly $|E|$ nodes of the tree $T$ of the set $V_1$ are connected each to all the destination nodes $d_i \in V_2 \quad \forall i = 1, 2, \dots, p$ with weight $0$. Where $E$ is the set of equivalence classes of the nodes of $T$ coming from the Hopcroft algorithm applied to the tree $T$.
\end{lemma}

\begin{proof}
    Since the destination nodes $d_i \in V_2$ are connected to the nodes of the tree $T$ with weight $0$ only if there is no other node with the same class in $V_2$ coming after the node in the ordering, then for sure there are $|E|$ nodes of the tree $T$ in $V_1$ that have no other node with the same class in $V_2$ coming after the node in the ordering.
\end{proof}

\begin{lemma} \label{lemma:optimal_cost}
    The optimal solution of the \textit{CHAINS-DIVISION} problem for an instance $\mathcal I$ for a tree $T$ is always greater or equal to the number of equivalence classes coming from the Hopcroft algorithm applied to the tree $T$.
\end{lemma}

\begin{proof}
    Since we aim to minimize the run length encoding of the chains, and the minimum cost of a chain is $1$, then the optimal cost of the \textit{CHAINS-DIVISION} problem for the tree $T$ is always greater or equal to the number of equivalence classes since if we dispose them in $|E|$ chains we will have a cost of $|E|$ since each chain contains only nodes with the same class, and if we dispose them in $p < |E|$ chains we will have a cost greater or equal to $|E|$ since we will have to put at least two nodes with the same class into one chain or more.
\end{proof}

\begin{claim} \label{claim:p_less_than_E}
    The solutions for the \textit{CHAINS-DIVISION} problem for the instances where the number $p$ of chains is greater than $|E|$ are not better than the solutions for the instances where $p \leq |E|$.
\end{claim}

\begin{proof}
    The proof comes directly from lemma \ref{lemma:optimal_cost}.
\end{proof}

\begin{lemma} \label{lemma:greater_nodes}
    Given a bipartite graph $G$ constructed as stated in Definition \ref{def:bip_construction} from a tree $T$, for each node $u \in V_1$ coming from the nodes of $T$ it is impossible for $u$ to be connected to node $v \in V_2$ coming from the nodes of $T$ such as $v < u$ in the order of the nodes of the tree $T$.
\end{lemma}

\begin{proof}
    The proof comes from the construction of $G$ where the nodes of $V_1$ are always connected to the nodes of $V_2$ coming after them in the ordering of the nodes of the tree $T$.
\end{proof}

\begin{lemma} \label{lemma:matching_existence}
    For every possible instance of the \textit{CHAINS-DIVISION} problem, a perfect matching exists in the bipartite graph G constructed as specified in Definition \ref{def:bip_construction}.
\end{lemma}

\begin{proof}
    The proof comes from the construction of the bipartite graph $G$ and from theorem \ref{thm:halls_marriage_theorem}. We are going to proof that the bipartite graph $G$ constructed as stated in Definition \ref{def:bip_construction} satisfies the Hall's condition and so, since by construction $|V_1| = |V_2|$, a perfect matching for $G$ exists.

    To verify Hall's condition we need to prove that for any subset $W \subseteq V_1$ we have that $|W| \leq |N(W)|$, where $N(W)$ is the neighborhood of $W$ (Definition \ref{def:neighborhood}). We have the following cases (we refer to the set of source nodes $s_i \in V_1$ as $S$, the set of destination nodes $d_i \in V_2$ as $D$, and the set of tree nodes $u_i \in V_1$ as $U$):
    \begin{enumerate}
        \item \textbf{$W$ contains only source nodes ($W \subseteq S$)}: Let $W = \{s_{i_1}, \dots, s_{i_k}\}$ where $k = |W| \leq p$. By construction rule 1, each source $s_i$ is connected to the first $p$ distinct-class nodes $v_j$ in $V_2$. Let these target nodes be $T = \{v_{j_1}, \dots, v_{j_p}\}$. The neighborhood $N(W)$ is a subset of $T$. Since all sources in $W$ connect to nodes within $T$, $|N(W)| \leq |T| = p$. However, the construction connects each source $s_i$ to $p$ distinct nodes in $V_2$. A precise analysis of $N(W)$ is needed. In the definition, where all sources connect to the same first $p$ distinct class nodes, $N(W)$ would be exactly those $p$ nodes if $W$ is non-empty. So, $|N(W)| = p$. Since $|W| = k \leq p$, we have $|N(W)| \geq |W|$.
        \item \textbf{$W$ contains only tree nodes ($W \subseteq U$)}: Let $W = \{u_{j_1}, \dots, u_{j_k}\} \subseteq V_1$ where $k = |W|$. This is the most complex case. We need to consider the neighborhood $N(W) \subseteq V_2 = V \cup D$. By construction:
        \begin{itemize}
            \item Each $u_j \in W$ connects only to nodes $v_l$ in $V_2$ where $l > j$ (lemma \ref{lemma:greater_nodes})
            \item Each node connects to the next node of the same class with weight 0 (same-class connections)
            \item Each node connects to the next $p$ distinct-class nodes with weight 1 (distinct-class connections) 
            \item Nodes that are last of their class connect to all destination nodes $d_1,\dots,d_p$ (destination connections)
        \end{itemize}
        We distinguish the following cases:
        \begin{itemize}
            \item In $W$ there are only nodes which are not last of their class, by lemma \ref{lemma:greater_nodes}, $u_m$ connects to nodes $v_l$ or $d_l$ where $l > m$. For sure at least one neighbor is distinct from those of earlier nodes in $W$ meaning that $|N(W)| \geq |W|$.
            \item In $W$ there are only nodes which are last of their class, also in this case for sure we have $|N(W)| \geq |W|$ since the last node of $W$ in the ordering connects only to all destination nodes $d_1,\dots,d_p$ (with $p \geq 2$) but the previous ones in the ordering connect to destinations but also to the next node of distinct class with weight 1 allowing to connect not only the destination nodes but also to other distinct tree nodes in $V_2$.
            \item In $W$ there are both nodes which are last of their class and nodes which are not last of their class, in this last case we have $|N(W)| \geq |W|$ as a derivation from the previous cases.
        \end{itemize}
        \item \textbf{$W$ contains both source and tree nodes ($W=W_S \cup W_U, where W_S \subseteq S, W_U \subseteq U$)}: We have $|N(W)| \geq |W|$ as a derivation from the previous cases and lemma \ref{lemma:greater_nodes}.
    \end{enumerate}
\end{proof}

\begin{comment}
    \begin{proof}
        The proof comes from the construction of the bipartite graph $G$ and from theorem \ref{thm:perfect_matching_existence}. We are going to proof that the bipartite graph $G$ constructed as stated in Definition \ref{def:bip_construction} has a Tutte matrix (definition \ref{def:tutte_matrix}) with determinant different from $0$ and so a perfect matching for $G$ exists.

        We know that a $n \times n$ matrix $M$ has $Det(M) \neq 0$ if and only if it has full rank ($rank(M) = n$), or equivalently if it has $n$ linearly independent rows or columns. We can see that the bipartite graph $G$ has $2t + 2p$ nodes and $O(t (p + 1) + p^2 + tp)$ edges, and so the Tutte matrix of $G$ will have $2t + 2p$ rows and $2t + 2p$ columns. The columns of $M$ are all independent since each node of $G$ in $V_1$ is connected only to nodes in $V_2$ that are greater than $u$ in the ordering. Also each node is connected to at least one node in $V_2$ and at most $p + 1$ distinct nodes with distinct class in $V_2$. Those conditions on the edges are sufficient to get a full rank matrix and so a perfect matching for $G$ exists.
    \end{proof}

    In figure \ref{fig:tutte_matrix_ex} the Tutte matrix for the bipartite graph in Figure \ref{fig:reduction_example}-(a) is shown.

    \begin{figure}
        \centering
        \[
        \begin{array}{c|ccccccccc}
                & \text{1} & \text{2} & \text{1} & \text{3} & \text{1} & \text{2} & \text{2} & \text{$d_1$} & \text{$d_2$} \\
            \hline
            \text{$s_1$} & 1 & 1 & 0 & 0 & 0 & 0 & 0 & 0 & 0 \\
            \text{$s_2$} & 1 & 1 & 0 & 0 & 0 & 0 & 0 & 0 & 0 \\
            \text{1}   & 0 & 1 & 1 & 1 & 0 & 0 & 0 & 0 & 0 \\
            \text{2}   & 0 & 0 & 1 & 1 & 0 & 1 & 0 & 0 & 0 \\
            \text{1}   & 0 & 0 & 0 & 1 & 1 & 1 & 0 & 0 & 0 \\
            \text{3}   & 0 & 0 & 0 & 0 & 1 & 1 & 0 & 1 & 1 \\
            \text{1}   & 0 & 0 & 0 & 0 & 0 & 1 & 0 & 1 & 1 \\
            \text{2}   & 0 & 0 & 0 & 0 & 0 & 0 & 1 & 0 & 0 \\
            \text{2}   & 0 & 0 & 0 & 0 & 0 & 0 & 0 & 1 & 1 \\
        \end{array}
        \]
        \caption[Tutte matrix example]{Example of a Tutte matrix for a bipartite graph in figure \ref{fig:reduction_example}-(a). As we can see the matrix has full rank and so a perfect matching exists.}
        \label{fig:tutte_matrix_ex}
    \end{figure}
\end{comment}

Now we can prove the correctness of the reduction.

\begin{theorem}
    An optimal solution of an instance $\mathcal I$ with $p \leq |E|$ of the problem defined in Definition \ref{def:problem_def} is equivalent to an optimal solution of the Minimum Weight Perfect Bipartite Matching for the instance $r(\mathcal I)$ where $r: \mathcal{I}_{CHAINS-DIVISION} \rightarrow \mathcal{I}_{MWPBM}$ is the reduction function that maps an instance of the problem defined in Definition \ref{def:problem_def} to an instance of the Minimum Weight Perfect Bipartite Matching problem defined in Definition \ref{def:mwpbm} constructed as stated in Definition \ref{def:bip_construction}.
\end{theorem}

\begin{proof}
    Let $\mathcal{I}$ be an instance of the \textsc{CHAINS-DIVISION} problem with a tree $T$ (having $t$ nodes), its equivalence classes $E$, and a target number of chains $p$. From claim \ref{claim:p_less_than_E}, we only consider $p \leq |E|$. Let $G = (V_1 \cup V_2, E')$ be the bipartite graph constructed according to Definition \ref{def:bip_construction}, which is the instance $r(\mathcal{I})$ for the \textsc{MWPBM} problem. We need to show that an optimal solution to \textsc{CHAINS-DIVISION} for $\mathcal{I}$ corresponds directly to an optimal solution (a minimum weight perfect matching) for \textsc{MWPBM} on $G$, and vice versa.

    By construction, $|V_1| = |V_2| = t+p$. Lemma \ref{lemma:matching_existence} guarantees that a perfect matching always exists in $G$.

    Consider a perfect matching $M$ in $G$. Since $|V_1|=|V_2|$ and $M$ is perfect, every node in $V_1$ is matched to exactly one node in $V_2$, and vice versa. The matching $M$ consists of $t+p$ edges. Due to the construction and Lemma \ref{lemma:greater_nodes} (nodes $u \in V_1 \cap T$ only connect to $v \in V_2 \cap T$ with $v > u$ or to destination nodes $d_j$), the matching $M$ naturally decomposes into $p$ paths starting from the source nodes $s_1, \dots, s_p$ and ending at the destination nodes $d_1, \dots, d_p$. Each path traverses a sequence of nodes corresponding to the nodes of the tree $T$.
    Specifically, a path starting at $s_i$ will match $s_i$ to some tree node $u_{j_1} \in V_2$. Then, the corresponding node $u_{j_1} \in V_1$ must be matched to some $u_{j_2} \in V_2$ (where $j_2 > j_1$) or a destination node $d_k$. This continues until a tree node $u_{j_m} \in V_1$ is matched to a destination node $d_k$. This forms a sequence $s_i \rightarrow u_{j_1} \rightarrow u_{j_2} \rightarrow \dots \rightarrow u_{j_m} \rightarrow d_k$.

    Let this set of $p$ paths define a partition of the tree nodes into $p$ chains $C_1, \dots, C_p$, where each chain $C_k$ contains the sequence of tree nodes visited by the $k$-th path. The ordering within each chain respects the original tree ordering $\pi$ due to Lemma \ref{lemma:greater_nodes}.

    Now, let's analyze the weight of the matching $M$. The total weight $W(M)$ is the sum of the weights of its edges. According to Definition \ref{def:bip_construction}:
    \begin{itemize}
        \item Edges from source nodes $s_i$ have weight 1. There are $p$ such edges in $M$. These correspond to the start of each chain, contributing a base cost of 1 per chain.
        \item Edges $(u_j, u_k)$ between tree nodes in $V_1$ and $V_2$ have weight 0 if $u_j$ and $u_k$ belong to the same equivalence class ($class(u_j) = class(u_k)$). These edges continue a run of the same class within a chain.
        \item Edges $(u_j, u_k)$ between tree nodes have weight 1 if $class(u_j) \neq class(u_k)$. These edges signify a change in the equivalence class within a chain, thus starting a new run in the RLE sense.
        \item Edges $(u_j, d_k)$ from a tree node to a destination node have weight 0. These edges terminate a chain and do not contribute to the RLE cost.
    \end{itemize}
    The total weight of the matching $M$ is therefore 
    $$W(M) = p + (\text{number of weight-1 edges between tree nodes})$$
    The cost of the chain partition, defined as the sum of the lengths of the run-length encodings of the chains, is precisely $p + (\text{number of class changes within chains})$. A class change occurs exactly when a path in the matching uses a weight-1 edge between tree nodes.
    Therefore, $W(M)$ is exactly equal to the RLE cost of the partition defined by the matching $M$.

    Conversely, any valid partition of $T$ into $p$ ordered chains $C_1, \dots, C_p$ can be mapped to a perfect matching $M$ in $G$. For each chain $C_k = [u_{k,1}, \dots, u_{k,m_k}]$, we construct a path in $G$: match $s_k$ to $u_{k,1} \in V_2$. Then match $u_{k,i} \in V_1$ to $u_{k,i+1} \in V_2$ for $i=1, \dots, m_k-1$. Finally, match $u_{k,m_k} \in V_1$ to one of the available destination nodes $d_j$. Since we have $p$ chains and $p$ source/destination nodes, and every tree node is in exactly one chain, this process uses all $t+p$ nodes in $V_1$ and $V_2$, forming a perfect matching. The weight of this matching will correspond to the RLE cost of the partition, as argued above.

    Since there is a one-to-one correspondence between valid chain partitions and perfect matchings in $G$, and the cost of a partition (total RLE length) is equal to the weight of the corresponding matching, minimizing the matching weight is equivalent to minimizing the RLE cost. Thus, an optimal solution to \textsc{MWPBM} on $G$ yields an optimal solution to \textsc{CHAINS-DIVISION} on $T$.
\end{proof}

\subsection{Full example}
Consider The example in Figure \ref{fig:reduction_example} where we have a tree $T$ with $t=7$ nodes, $p = 2$ chains and the equivalence classes $E = \{1,2,1,3,1,2,2\}$ sorted accordingly to the upward path $\pi$ of each node of the tree. We can construct the two distinct sets $V_1$ and $V_2$ of the bipartite graph $G$ as follows: $V_1 = \{s_1, s_2, 1,2,1,3,1,2,2\}$ and $V_2 = \{1,2,1,3,1,2,2, d_1, d_2\}$. The edges of the graph $G$ will be constructed as stated in Definition \ref{def:bip_construction}. In Figure \ref{fig:reduction_example}-(a) we have the resulting bipartite graph, and in Figure \ref{fig:reduction_example}-(b) we have one of the possible minimum perfect matchings for the graph in (a) having weight $4$. A the end we can see that the optimal partition of the nodes of the tree $T$ is $C_1 = \{1,1,1,2,2\}$ and $C_2 = \{2,3\}$ with a total cost of $4$, this can be obtained starting from the sources and by following the edges of the nodes, jumping to the corresponding node in $V_1$ and following the edges again until we reach the destination nodes.

\begin{figure}[H]
    \centering
    \begin{tabular}{cc}
        \begin{tikzpicture}[node distance={10mm}, thick, auto=center, main/.style = {draw, circle}]
            \node[main] (1s) {$s_1$};
            \node[main] (2s) [below of=1s] {$s_2$};
            \node[main] (3s) [below of=2s] {$1$};
            \node[main] (4s) [below of=3s] {$2$};
            \node[main] (5s) [below of=4s] {$1$};
            \node[main] (6s) [below of=5s] {$3$};
            \node[main] (7s) [below of=6s] {$1$};
            \node[main] (8s) [below of=7s] {$2$};
            \node[main] (9s) [below of=8s] {$2$};
            \node[main] (1d) [right=3cm of 3s] {$1$};
            \node[main] (2d) [below of=1d] {$2$};
            \node[main] (3d) [below of=2d] {$1$};
            \node[main] (4d) [below of=3d] {$3$};
            \node[main] (5d) [below of=4d] {$1$};
            \node[main] (6d) [below of=5d] {$2$};
            \node[main] (7d) [below of=6d] {$2$};
            \node[main] (8d) [below of=7d] {$d_1$};
            \node[main] (9d) [below of=8d] {$d_2$};

            \draw[red, ->] (1s) -- (1d);
            \draw[red, ->] (1s) -- (2d);
            \draw[red, ->] (2s) -- (1d);
            \draw[red, ->] (2s) -- (2d);
            \draw[red, ->] (3s) -- (2d);
            \draw[red, ->] (3s) -- (4d);
            \draw[green, ->] (3s) -- (3d);
            \draw[red, ->] (4s) -- (3d);
            \draw[red, ->] (4s) -- (4d);
            \draw[green, ->] (4s) -- (6d);
            \draw[red, ->] (5s) -- (4d);
            \draw[green, ->] (5s) -- (5d);
            \draw[red, ->] (5s) -- (6d);
            \draw[red, ->] (6s) -- (5d);
            \draw[red, ->] (6s) -- (6d);
            \draw[green, ->] (6s) -- (8d);
            \draw[green, ->] (6s) -- (9d);
            \draw[red, ->] (7s) -- (6d);
            \draw[green, ->] (7s) -- (8d);
            \draw[green, ->] (7s) -- (9d);
            \draw[green, ->] (8s) -- (7d);
            \draw[green, ->] (9s) -- (8d);
            \draw[green, ->] (9s) -- (9d);
        \end{tikzpicture} &
        \begin{tikzpicture}[node distance={10mm}, thick, auto=center, main/.style = {draw, circle}]
            \node[main] (1s) {$s_1$};
            \node[main] (2s) [below of=1s] {$s_2$};
            \node[main] (3s) [below of=2s] {$1$};
            \node[main] (4s) [below of=3s] {$2$};
            \node[main] (5s) [below of=4s] {$1$};
            \node[main] (6s) [below of=5s] {$3$};
            \node[main] (7s) [below of=6s] {$1$};
            \node[main] (8s) [below of=7s] {$2$};
            \node[main] (9s) [below of=8s] {$2$};
            \node[main] (1d) [right=3cm of 3s] {$1$};
            \node[main] (2d) [below of=1d] {$2$};
            \node[main] (3d) [below of=2d] {$1$};
            \node[main] (4d) [below of=3d] {$3$};
            \node[main] (5d) [below of=4d] {$1$};
            \node[main] (6d) [below of=5d] {$2$};
            \node[main] (7d) [below of=6d] {$2$};
            \node[main] (8d) [below of=7d] {$d_1$};
            \node[main] (9d) [below of=8d] {$d_2$};

            \draw[red, ->] (1s) -- (1d);
            \draw[red, ->] (2s) -- (2d);
            \draw[green, ->] (3s) -- (3d);
            \draw[red, ->] (4s) -- (4d);
            \draw[green, ->] (5s) -- (5d);
            \draw[green, ->] (6s) -- (8d);
            \draw[red, ->] (7s) -- (6d);
            \draw[green, ->] (8s) -- (7d);
            \draw[green, ->] (9s) -- (9d);
        \end{tikzpicture} \\
    (a) & (b) \\
    \end{tabular}
    \caption[Reduction full example]{Example of a reduction for the sorted nodes' equivalency classes $E = \{1,2,1,3,1,2,2\}$. In (a), we have the resulting bipartite graph constructed from $E$. In (b), we have the resulting perfect matching for the graph in (a) having weight $4$. Green edges weigh $0$, while red edges weigh $1$.}
    \label{fig:reduction_example}
\end{figure}

\subsection{Heuristics and Improvements}
Some changes can be made to the reduction in order to optimize it and to reduce the number of edges in the bipartite graph. Here are some of the improvements that can be made.

\begin{itemize}
    \item \textbf{Sources' edges optimization}:
    \begin{lemma} \label{lemma:sources_optimization}
        The sources' edges can be optimized by connecting each source only to the smaller node (considering the order of the nodes) in $V_2$ coming from the tree $T$ that is not connected to any other source.
    \end{lemma}

    \begin{proof}
        Since the source nodes are needed to distinguish the chains as starting points, we need that each source is connected to at least one node in $V_2$ coming from the tree $T$. Having the sources connected to the first $p$ nodes with distinct equivalence class in $V_2$ is not necessary since allows us just to invert the chains starting from each source and so it is redundant. We can connect each source to the smaller node in $V_2$ coming from the tree $T$ that is not connected to any other source since we need to connect each source to at least one node in $V_2$ coming from the tree $T$ and this will allow us to distinguish the chains.
    \end{proof}

    This will reduce the number of edges coming from the sources from $O(p^2)$ to $O(p)$. In Figure \ref{fig:heuristics_example}-(a) the removed edges are shown in green.
    \item \textbf{Tree nodes' edges optimization 1}:
    \begin{lemma} \label{lemma:tree_optimization_1}
        The tree nodes' edges can be optimized by removing the edges of tree nodes that are connected to nodes in $V_2$ already linked to a source node in $V_1$.
    \end{lemma}

    \begin{proof}
        From definition \ref{def:matching} we know that a matching $M \in E$ is a collection of edges such that every vertex of $V$ is incident to at most one edge of $M$. In other words, a matching is a set of edges such that no two edges share a common vertex. Given that, in all the solutions to the problem all sources will be connected to exactly one node in $V_2$ coming from the tree $T$ and so we can remove the edges of the tree nodes that are connected to nodes in $V_2$ already linked to a source node in $V_1$ since they will not be part of the final matching.
    \end{proof}

    This will reduce the number of edges by a factor of $O(p - 1)$. In Figure \ref{fig:heuristics_example}-(a) the removed edges are shown in blue.

    \item \textbf{Tree nodes' edges optimization 2}:
    \begin{lemma} \label{lemma:tree_optimization_2}
        The tree nodes' edges can be optimized by removing the edges with weight $1$ starting from a node $u \in V_1$ to a node $v \in V_2$ if the node $u$ has another edge with weight $0$ connected to a node $z \in V_2$ such that $z < v$ in the ordering of the nodes.
    \end{lemma}

    \begin{proof}
        Since every time we have an edge with weight $0$ between two nodes of $V_1$ and $V_2$ it means that those two nodes have the same equivalence class and so there is no need to add additional cost trying to connect that node to nodes with different classes coming after in the ordering, as that would only increase the cost without providing any benefit to the solution. Also, we know that connecting nodes with the same class is always the best choice for optimizing the run length encoding of each chain.
    \end{proof}

    In Figure \ref{fig:heuristics_example}-(a) the removed edges are shown in red.
\end{itemize}

In figure \ref{fig:heuristics_example}-(b) we can see the resulting bipartite graph for the example shown in previous section after the optimizations.

\begin{figure}[H]
    \centering
    \begin{tabular}{cc}
        \begin{tikzpicture}[node distance={10mm}, thick, auto=center, main/.style = {draw, circle}]
            \node[main] (1s) {$s_1$};
            \node[main] (2s) [below of=1s] {$s_2$};
            \node[main] (3s) [below of=2s] {$1$};
            \node[main] (4s) [below of=3s] {$2$};
            \node[main] (5s) [below of=4s] {$1$};
            \node[main] (6s) [below of=5s] {$3$};
            \node[main] (7s) [below of=6s] {$1$};
            \node[main] (8s) [below of=7s] {$2$};
            \node[main] (9s) [below of=8s] {$2$};
            \node[main] (1d) [right=3cm of 3s] {$1$};
            \node[main] (2d) [below of=1d] {$2$};
            \node[main] (3d) [below of=2d] {$1$};
            \node[main] (4d) [below of=3d] {$3$};
            \node[main] (5d) [below of=4d] {$1$};
            \node[main] (6d) [below of=5d] {$2$};
            \node[main] (7d) [below of=6d] {$2$};
            \node[main] (8d) [below of=7d] {$d_1$};
            \node[main] (9d) [below of=8d] {$d_2$};

            \draw[black, dashed, ->] (1s) -- (1d);
            \draw[green, dashed, ->] (1s) -- (2d);
            \draw[green, dashed, ->] (2s) -- (1d);
            \draw[black, dashed, ->] (2s) -- (2d);
            \draw[blue, dashed, ->] (3s) -- (2d);
            \draw[red, dashed, ->] (3s) -- (4d);
            \draw[black, ->] (3s) -- (3d);
            \draw[black, dashed, ->] (4s) -- (3d);
            \draw[black, dashed, ->] (4s) -- (4d);
            \draw[black, ->] (4s) -- (6d);
            \draw[black, dashed, ->] (5s) -- (4d);
            \draw[black, ->] (5s) -- (5d);
            \draw[red, dashed, ->] (5s) -- (6d);
            \draw[black, dashed, ->] (6s) -- (5d);
            \draw[black, dashed, ->] (6s) -- (6d);
            \draw[black, ->] (6s) -- (8d);
            \draw[black, ->] (6s) -- (9d);
            \draw[black, dashed, ->] (7s) -- (6d);
            \draw[black, ->] (7s) -- (8d);
            \draw[black, ->] (7s) -- (9d);
            \draw[black, ->] (8s) -- (7d);
            \draw[black, ->] (9s) -- (8d);
            \draw[black, ->] (9s) -- (9d);
        \end{tikzpicture} &
        \begin{tikzpicture}[node distance={10mm}, thick, auto=center, main/.style = {draw, circle}]
            \node[main] (1s) {$s_1$};
            \node[main] (2s) [below of=1s] {$s_2$};
            \node[main] (3s) [below of=2s] {$1$};
            \node[main] (4s) [below of=3s] {$2$};
            \node[main] (5s) [below of=4s] {$1$};
            \node[main] (6s) [below of=5s] {$3$};
            \node[main] (7s) [below of=6s] {$1$};
            \node[main] (8s) [below of=7s] {$2$};
            \node[main] (9s) [below of=8s] {$2$};
            \node[main] (1d) [right=3cm of 3s] {$1$};
            \node[main] (2d) [below of=1d] {$2$};
            \node[main] (3d) [below of=2d] {$1$};
            \node[main] (4d) [below of=3d] {$3$};
            \node[main] (5d) [below of=4d] {$1$};
            \node[main] (6d) [below of=5d] {$2$};
            \node[main] (7d) [below of=6d] {$2$};
            \node[main] (8d) [below of=7d] {$d_1$};
            \node[main] (9d) [below of=8d] {$d_2$};

            \draw[black, dashed, ->] (1s) -- (1d);
            \draw[black, dashed, ->] (2s) -- (2d);
            \draw[black, ->] (3s) -- (3d);
            \draw[black, dashed, ->] (4s) -- (3d);
            \draw[black, dashed, ->] (4s) -- (4d);
            \draw[black, ->] (4s) -- (6d);
            \draw[black, dashed, ->] (5s) -- (4d);
            \draw[black, ->] (5s) -- (5d);
            \draw[black, dashed, ->] (6s) -- (5d);
            \draw[black, dashed, ->] (6s) -- (6d);
            \draw[black, ->] (6s) -- (8d);
            \draw[black, ->] (6s) -- (9d);
            \draw[black, dashed, ->] (7s) -- (6d);
            \draw[black, ->] (7s) -- (8d);
            \draw[black, ->] (7s) -- (9d);
            \draw[black, ->] (8s) -- (7d);
            \draw[black, ->] (9s) -- (8d);
            \draw[black, ->] (9s) -- (9d);
        \end{tikzpicture} \\
    (a) & (b) \\
    \end{tabular}
    \caption[Reduction heuristics example]{Example of a reduction for the sorted nodes' equivalency classes $E = \{1,2,1,3,1,2,2\}$ applying also the heuristics showed. In (a), the edges removed are shown in green for lemma \ref{lemma:sources_optimization}, blue for lemma \ref{lemma:tree_optimization_1}, and red for lemma \ref{lemma:tree_optimization_2}. In (b), we have the resulting bipartite graph after the heuristics applied. Dashed edges weigh $1$, while solid edges weigh $0$.}
    \label{fig:heuristics_example}
\end{figure}

\subsection{Moving to Maximum weight perfect bipartite matching}
In this section, we will discuss how to slightly modify the reduction process to move from a minimum weight perfect bipartite matching problem to a maximum weight perfect bipartite matching problem. This will be helpful in solving the problem more efficiently by using some known algorithms to solve the maximum weight perfect bipartite matching problem.

\begin{theorem}
    An optimal solution of an instance $\mathcal I$ with $p \leq |E|$ of the \textsc{CHAINS-DIVISION} problem is equivalent to an optimal solution of the Maximum Weight Perfect Bipartite Matching for the instance $r(\mathcal I)$ where $r: \mathcal{I}_{CHAINS-DIVISION} \rightarrow \mathcal{I}_{MWPBM}$ is the reduction function that maps an instance of the \textsc{CHAINS-DIVISION} problem to an instance of the Maximum Weight Perfect Bipartite Matching problem constructed as stated in Definition \ref{def:bip_construction} but with inverted weights (weight $0$ becomes $1$ and weight $1$ becomes $0$).
\end{theorem}

\begin{proof}
    Let $M$ be a perfect matching in the bipartite graph $G$ constructed as stated in Definition \ref{def:bip_construction}. Let $w(M)$ be the sum of the weights of the edges in the matching $M$. From the previous theorem, we know that the optimal solution of the \textsc{CHAINS-DIVISION} problem is equivalent to finding a perfect matching $M$ in $G$ that minimizes $w(M)$.

    Let $G'$ be a bipartite graph constructed as $G$ but with inverted weights (weight $0$ becomes $1$ and weight $1$ becomes $0$). Let $M'$ be a perfect matching in $G'$ and let $w'(M')$ be the sum of the weights of the edges in the matching $M'$. Let $k$ be the number of edges in the matching.

    We can see that for any matching $M$ in $G$:
    \[ w'(M) = k - w(M) \]

    This means that maximizing $w'(M)$ is equivalent to minimizing $w(M)$. Therefore, finding the maximum weight perfect matching in $G'$ is equivalent to finding the minimum weight perfect matching in $G$, which in turn is equivalent to finding the optimal solution of the \textsc{CHAINS-DIVISION} problem.
\end{proof}

\section{Collapsing Nodes in Chains} \label{sec:collapsing}
The next crucial step of our compression scheme is to reduce the space required by each chain by collapsing equivalent nodes. Specifically, any sequence of consecutive nodes within the same chain that belong to the same equivalence class is merged into a single representative node. This new node preserves the connectivity of the original structure by inheriting all distinct outgoing and ingoing edges from the nodes it replaces. In the following subsections we will introduce the concept of non-deterministic finite automaton and define how nodes are collapsed in a chain. Then, we prove that this transformation is language-preserving (see \cref{lemma:collapsing_equivalence}).

\subsection{How to Collapse Nodes}
Now, we define the concept of collapsing consecutive equivalent nodes in a chain.
\begin{definition}[Collapsing consecutive equivalent nodes]
    \label{def:collapsing}
    Let $V$ be the set of nodes of the tree, let $\Sigma$ be the alphabet, and let $E \subseteq V \times \Sigma \times V$ be the set of labeled edges.
    Let $\mathcal{P} = \{C_1, C_2, \ldots, C_m\}$ be the set of all chains partitioning $V$.
    
    For each chain $C_i = (u_1^{(i)}, u_2^{(i)}, \ldots, u_{n_i}^{(i)}) \in \mathcal{P}$, partition it into maximal consecutive blocks $B_1^{(i)}, \ldots, B_{k_i}^{(i)}$, where each block $B_t^{(i)} = (u_j^{(i)}, \ldots, u_\ell^{(i)})$ satisfies $\equivsetfunc{u_r^{(i)}} = \equivsetfunc{u_s^{(i)}}$ for all $r,s \in \{j,\ldots,\ell\}$, and the block is maximal (cannot be extended).

    The collapsed chain is $C_i' = (v_1^{(i)}, \ldots, v_{k_i}^{(i)})$, where each block $B_t^{(i)}$ is replaced by a single node $v_t^{(i)}$.
    
    Define the global collapse map $\Phi: V \to V'$ where $V' = \bigcup_{i=1}^m C_i'$, such that $\Phi(u) = v_t^{(i)}$ if $u \in B_t^{(i)}$ for some chain $C_i$ and block $B_t^{(i)}$.
    
    Then:
    \begin{itemize}[leftmargin=25pt]
        \item Two consecutive nodes $u_j^{(i)}, u_{j+1}^{(i)} \in C_i$ are collapsed into the same node if and only if $\equivsetfunc{u_j^{(i)}} = \equivsetfunc{u_{j+1}^{(i)}}$.
        \item The edge set after collapsing all chains is
        \[
            E' \;=\; \{\, (\Phi(x), a, \Phi(y)) \;:\; (x,a,y) \in E \,\},
        \]
        where parallel duplicates are removed (i.e., $E'$ is treated as a set).
    \end{itemize}
\end{definition}

\begin{example}
    Consider \cref{ex:reduction_ex} where we obtained the chains $C_1 = \{A,C,C,B\}$ and $C_2 = \{B,D,D,D,D,D,D\}$ for the tree ADFA in \cref{fig:example_ADFA} by applying the reduction from \textsc{Chain-Division} to \textsc{MWPBM}. The nodes inside each chain are the following:
    \begin{itemize}
        \item $C_1 = \{a,d,f,c\}$
        \item $C_2 = \{b,h,l,e,i,m,g\}$
    \end{itemize}
    
    Applying the collapsing operation from \cref{def:collapsing}:
    \begin{itemize}
        \item For $C_1 = (a,d,f,c)$ with classes $(A,C,C,B)$: 
        \begin{itemize}
            \item Block $B_1 = \{a\}$ (class $A$) $\rightarrow$ collapsed node $v_1$. The node $a$ is the initial state. It has two outgoing edges: $a \xrightarrow{0} b$ and $a \xrightarrow{1} c$. Since $b$ and $c$ collapse to $w_1$ and $v_3$ respectively, we obtain $v_1 \xrightarrow{0} w_1$ and $v_1 \xrightarrow{1} v_3$.
            \item Block $B_2 = \{d,f\}$ (class $C$) $\rightarrow$ collapsed node $v_2$. The outgoing edges of $d$ and $f$ are:
            \[
                d \xrightarrow{0} h,\; d \xrightarrow{1} i,\qquad
                f \xrightarrow{0} l,\; f \xrightarrow{1} m.
            \]
            After collapsing, we obtain:
            \[
                v_2 \xrightarrow{0} w_2, v_2 \xrightarrow{1} w_2,\qquad v_2 \xrightarrow{0} w_2, v_2 \xrightarrow{1} w_2.
            \]
            Since we have two identical edges we can keep only one of each.
            \item Block $B_3 = \{c\}$ (class $B$) $\rightarrow$ collapsed node $v_3$. It has two outgoing edges: $c \xrightarrow{0} f$ and $c \xrightarrow{1} g$. Since $f$ and $g$ collapse to $v_2$ and $w_2$ respectively, we obtain $v_3 \xrightarrow{0} v_2$ and $v_3 \xrightarrow{1} w_2$.
        \end{itemize}
        Result: $C_1' = (v_1, v_2, v_3)$ with classes $(A, C, B)$. Here, $v_1$ is the initial state.
        
        \item For $C_2 = (b,h,l,e,i,m,g)$ with classes $(B,D,D,D,D,D,D)$:
        \begin{itemize}
            \item Block $B_1 = \{b\}$ (class $B$) $\rightarrow$ collapsed node $w_1$. It has two outgoing edges: $b \xrightarrow{0} d$ and $b \xrightarrow{1} e$. As $d,e$ collapse to $v_2$ and $w_2$ respectively, we obtain $w_1 \xrightarrow{0} v_2$ and $w_1 \xrightarrow{1} w_2$.
            \item Block $B_2 = \{h,l,e,i,m,g\}$ (all class $D$) $\rightarrow$ collapsed node $w_2$. The node $w_2$ collects all incoming edges formerly targeting any of $b,h,l,e,i,m,g$, and it is accepting.
        \end{itemize}
        Result: $C_2' = (w_1, w_2)$ with classes $(B,D)$, and $w_2$ is the unique accepting state for this example.
    \end{itemize}
    
    The collapsed chains preserve all distinct outgoing and incoming edges through the collapse map $\Phi$, significantly reducing the space complexity from 11 nodes to 5 nodes total. The resulting NFA is shown in \cref{fig:minimized_chains}.

    \begin{figure}[H]
        \centering
        \begin{tikzpicture}[->, >=stealth, node distance=3cm, on grid, auto]
            \node[state, initial, initial text=] (v1) {$v_1$};
            \node[state] (v2) [right=of v1] {$v_2$};
            \node[state] (v3) [right=of v2] {$v_3$};
            \node[state] (w1) [right=of v3] {$w_1$};
            \node[state, accepting] (w2) [right=of w1] {$w_2$};
        
            \path (v1) edge [bend left] node {0} (w1)
                    edge [bend right] node[below] {1} (v3)
                % v2 -> w1 (due archi su ancoraggi diversi)
                (v2) edge [out=40, in=140, looseness=1, shorten >=1pt] node {0} (w2.100)
                    edge [out=-40, in=250, looseness=1, shorten >=1pt] node[below] {1} (w2.260)
                % v3 -> v2 / v4 (rimangono)
                (v3) edge [bend right] node[above] {0} (v2)
                % v3 -> w1 (un arco con angolo dedicato)
                    edge [out=-30, in=220, looseness=1.3, shorten >=1pt] node[below] {1} (w2.230)
                % v4 -> w1 (due archi su ancoraggi diversi)
                (w1) edge [bend left] node {0} (v2)
                    edge [bend right] node[below] {1} (w2);
        \end{tikzpicture}
        \caption{NFA obtained after collapsing equivalent nodes in chains $C_1$ and $C_2$.}
        \label{fig:minimized_chains}
    \end{figure} 
\end{example}

\subsection{Language Equivalence}
Now, we need to prove that the language recognized by the NFA obtained after collapsing equivalent nodes in chains following \cref{def:collapsing} is equivalent to the language of the original tree ADFA. We define the language of an NFA or ADFA $F$ as $L(F)$.

\begin{lemma}[Root forms a singleton class under Revuz]
    \label{lemma:root_singleton}
    In a tree ADFA minimized by Revuz’s algorithm (\cref{sec:revuz}), the root forms a singleton equivalence class.
\end{lemma}
\begin{proof}
    Let $T$ be a tree ADFA, let $r$ be its root, and let $D = h(r)$ be its height as in \cref{def:height}. For every other node $u \neq r$ at depth $k \ge 1$, we have $h(u) \le D - k < D$. Hence $r$ is the unique state in the level $\Pi_D$. Since Revuz’s algorithm partitions by height and refines within each level, $r$ cannot be merged with any other state and thus forms its own equivalence class.
\end{proof}

\begin{lemma}[All leaves fall into the same class under Revuz]
    \label{lemma:all_leaves_same_class}
    In a tree ADFA minimized by Revuz’s algorithm, all leaves belong to the same equivalence class.
\end{lemma}
\begin{proof}
    Every leaf $\ell$ is final and has no outgoing transitions, so $h(\ell) = 0$. In the base step on level $\Pi_0$, the label of a state depends only on its finality and on transitions to already distinguished classes. All leaves share the same label (final, no outgoing transitions), hence they are merged into a single equivalence class in $\Pi_0$.
\end{proof}

\begin{lemma}
    \label{lemma:collapsing_equivalence}
    Let $T$ be a tree ADFA with alphabet $\Sigma$. Let $C_1, C_2, \ldots, C_m$ be the chains partitioning $T$ as defined in \cref{def:problem_def}. Let $N$ be the NFA obtained after collapsing equivalent nodes in chains $C_1, C_2, \ldots, C_m$ as defined in \cref{def:collapsing}. Then, $L(N) = L(T)$.
\end{lemma}

\begin{proof}
    We prove the equality $L(N) = L(T)$ by showing both inclusions $L(T) \subseteq L(N)$ and $L(N) \subseteq L(T)$.

    \textbf{($L(T) \subseteq L(N)$):} Let $w \in L(T)$. Then there exists an accepting path in $T$ from the root to some accepting state:
    \[
        q_0 \xrightarrow{a_1} q_1 \xrightarrow{a_2} q_2 \xrightarrow{a_3} \cdots \xrightarrow{a_n} q_n
    \]
    where $q_0$ is the root, $q_n$ is accepting, and $w = a_1a_2\ldots a_n$.

    By \cref{def:collapsing}, each node $q_i$ in the original tree is mapped to a collapsed node $\Phi(q_i)$ in $N$. Since the collapse map preserves all edges (by the definition of $E'$), there exists a corresponding path in $N$:
    \[
        \Phi(q_0) \xrightarrow{a_1} \Phi(q_1) \xrightarrow{a_2} \Phi(q_2) \xrightarrow{a_3} \cdots \xrightarrow{a_n} \Phi(q_n)
    \]
    Since $q_0$ is the root of $T$, by \cref{lemma:root_singleton} its image $\Phi(q_0)$ is the unique initial state of $N$. Since $q_n$ is accepting in $T$, by \cref{lemma:all_leaves_same_class} its image $\Phi(q_n)$ lies in the unique leaf class, which is marked accepting in $N$. Hence, $w \in L(N)$.

    \textbf{($L(N) \subseteq L(T)$):} Let $w \in L(N)$. Then there exists an accepting path in $N$:
    \[
        v_0 \xrightarrow{a_1} v_1 \xrightarrow{a_2} v_2 \xrightarrow{a_3} \cdots \xrightarrow{a_n} v_n
    \]
    where $v_0$ is the initial state, $v_n$ is accepting, and $w = a_1a_2\ldots a_n$.

    Each collapsed node $v_i$ corresponds to some block $B_t^{(j)}$ in the original tree. By \cref{def:collapsing}, every edge $(v_{i-1}, a_i, v_i)$ in $N$ corresponds to at least one edge $(u, a_i, u')$ in the original tree $T$, where $\Phi(u) = v_{i-1}$ and $\Phi(u') = v_i$.

    Since the chains preserve the ordering from the original tree structure, we can construct a valid path in $T$ by selecting appropriate representatives from each collapsed block. Specifically, we can choose nodes $u_0, u_1, \ldots, u_n$ in $T$ such that $\Phi(u_i) = v_i$ and $(u_{i-1}, a_i, u_i) \in E$ for all $i = 1, \ldots, n$.

    Since $v_0$ corresponds to the root block and $v_n$ is accepting, for \cref{lemma:root_singleton,lemma:all_leaves_same_class} we have $u_0$ as the root and $u_n$ as an accepting state in $T$ respectively. Therefore, $w \in L(T)$.

    Thus, $L(N) = L(T)$.
\end{proof}

Collapsing the nodes as in \cref{def:collapsing} preserves the language of the original tree ADFA and the resulting chains inherit a total order. This enables the application of the NFA indexing scheme of Cotumaccio et al.~\cite{cotumaccio2023co}, which we present in the next chapter.

\begin{theorem}[Myhill--Nerode]
Let $L \subseteq \Sigma^*$ and define the indistinguishability relation
\[
x \sim_L y \;\Longleftrightarrow\; \forall z \in \Sigma^*,\; xz \in L \iff yz \in L .
\]
Then:
\begin{enumerate}
    \item $\sim_L$ is an equivalence relation on $\Sigma^*$.
    \item $L$ is regular iff $\sim_L$ has finitely many equivalence classes.
    \item The states of the minimal DFA for $L$ are in one-to-one correspondence with the equivalence classes of $\sim_L$.
    \item $L$ is exactly the union of those equivalence classes that intersect $L$:
    \[
        L \;=\; \bigcup \{ [x]_{\sim_L} \mid x \in L \}.
    \]
\end{enumerate}
\end{theorem}

\begin{lemma}
Let $M=(Q,\Sigma,\delta,q_0,F)$ be a DFA recognizing $L \subseteq \Sigma^*$.
If two states $p,q \in Q$ correspond to the same Myhill--Nerode class for $L$ (i.e., for all $w\in\Sigma^*$ we have $\delta(p,w)\in F \iff \delta(q,w)\in F$),
then merging $p$ and $q$ into a single state yields an automaton (possibly nondeterministic) that still recognizes exactly $L$.
\end{lemma}

\begin{proof}
By the Myhill--Nerode theorem, every state of $M$ corresponds to a unique equivalence class of $\sim_L$, and $L$ is exactly the union of those classes that intersect $L$.
If $p$ and $q$ belong to the same equivalence class, then for every continuation $z \in \Sigma^*$ we have
\[
\delta(p,z) \in F \iff \delta(q,z) \in F.
\]
Thus replacing $p$ with $q$ (or vice versa) in any path does not affect whether the run ends in an accepting state. Therefore merging $p$ and $q$ does not alter the set of accepted strings, i.e.\ the recognized language remains $L$.
\end{proof} 
\newpage
\
\newpage
\input{Capitoli/Experiments.tex}
\newpage
\
% ----------------------
% ---- BIBLIOGRAPHY ----
% ----------------------
\backmatter
\sloppy % serve a non fare andare i link oltre ai margini
\printbibliography[heading=bibintoc, title=Bibliography]
\printbibliography[type=online, heading=bibintoc, title=Web bibliography]
% ----------------------
% ---- DOCUMENT END ----
% ----------------------
\end{document} % Fine documento
