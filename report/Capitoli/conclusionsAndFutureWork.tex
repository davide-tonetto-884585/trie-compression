\chapter{Conclusions and Future Works}
This thesis has presented a preliminary study of a novel pipeline designed for trie compression. The initial results are promising and suggest a strong potential for this approach. Specifically, our findings demonstrate that a minor increase in the co-lexicographical width of the input trie can achieve significant compression, particularly when applied to tries with a high degree of internal repetitiveness.

A central contribution of this research is the development of a methodology to balance trie compression with indexability. We framed this challenge as a String Partitioning problem, aiming to identify the optimal p-sortable compressed automaton for a given input trie. The core of our approach lies in a strategic partitioning of the trie's nodes. This partitioning is designed to maximize compression while ensuring the resulting automaton is $p$-sortable, thus controlling balance between a compact representation and efficient indexing. Furthermore, we demonstrated that this optimization problem can be reduced to finding a minimum-weight perfect matching in a bipartite graph, which allows for the identification of optimal node pairings to achieve the highest degree of compression.

The reason we specifically aim for a $p$-sortable automaton is that it allows to find a balance between storage efficiency and usability. While maximum compression could be achieved by converting the trie into a minimal directed acyclic graph (DAG), such a structure is generally difficult to index, rendering it inefficient for search operations. A p-sortable automaton, however, allows for the creation of an effective index. This index enables fast querying and data retrieval directly on the compressed form, thereby avoiding the performance bottleneck of decompressing the data before use. This method produces a p-sortable automaton that can be indexed effectively, allowing for efficient subsequent querying and data retrieval operations without a notable sacrifice in performance. This approach offers a valuable compromise in the trade-off between the extremes of full, un-indexable compression and uncompressed, fully-indexable structures.

\section{Future Works}
The research presented in this thesis is preliminary, and several avenues for future work have been identified. The reduction we have proposed for the String Partitioning problem, while effective, exhibits a quadratic time complexity with respect to the number of nodes in the trie. This computational cost renders it inefficient for very large-scale applications. A critical next step is to investigate improvements to this reduction to develop more scalable and efficient solutions (see \cref{sec:future_improvements} for further details). Subsequently, these optimized algorithms should be rigorously evaluated on real-world datasets to validate their performance and practical applicability.

Another significant direction for future research is the exploration of methods to produce a p-sortable deterministic finite automaton directly from our pipeline, rather than the current p-sortable non-deterministic finite automaton. A deterministic representation would offer further advantages in terms of space efficiency. One potential approach to achieve this is to develop a pruning strategy for the output NFA. Such a method would need to carefully remove states and transitions in a way that transforms the NFA into an equivalent DFA, ensuring that the recognized language remains unchanged.

The research and development of these future works will be continued during the author's PhD program.