\chapter{Introduction} \label{chp:introduction}
The problem of compressing large sets of strings, or finite languages, is a fundamental challenge in computer science with applications in areas like bioinformatics, natural language processing, and data indexing. A finite language can be naturally represented by a trie. In this representation, each string in the language corresponds to a unique path from the root to a final state. Compressing the language is therefore equivalent to compressing its corresponding trie structure.

Tries are exceptionally efficient for solving problems related to string collections and prefix-based queries. Their structure makes them ideal for a variety of applications:
\begin{enumerate}
    \item \textbf{Autocomplete and Predictive Text:} Tries are widely used in search engines and text editors to suggest completions for a given prefix. By traversing the trie, all words sharing the prefix can be quickly retrieved.
    \item \textbf{Spell Checkers:} A trie can store a dictionary of valid words. To check a word, one simply traverses the trie. If the path corresponding to the word does not end in a terminal node, the word is either misspelled or not in the dictionary.
    \item \textbf{IP Routing:} In networking, routers use tries to store routing tables. This allows for efficient longest prefix matching to determine the optimal route for an IP packet.
    \item \textbf{Bioinformatics:} Tries are used to store and search large collections of DNA sequences or other biological data, enabling efficient pattern matching and analysis of shared subsequences.
\end{enumerate}

Traditional compression algorithms often fail to exploit the inherent structural properties of tries. To address this, specialized techniques have been developed. Among the most prominent is the \textit{Extended Burrows-Wheeler Transform (XBWT)}~\cite{ferragina2009compressing}, which extends the classical Burrows-Wheeler Transform~\cite{burrows1994block} to labeled trees and can be applied to tries to achieve significant compression by capturing their structural regularities.

However, existing techniques may not be optimal when dealing with tries that exhibit a high degree of repetitiveness. Such is the case for languages containing many strings with shared substrings, leading to tries with large, identical subtrees. This thesis introduces and analyzes a novel compression technique specifically designed to exploit these repetitions.

\section{Challenges and Contributions}
The primary goal of this thesis is to develop a data structure that both compresses a given finite language and efficiently supports indexing queries, such as navigational and subpath queries (see \cref{def:tree_operations}). This challenge involves navigating a fundamental trade-off between compression and indexability, which we explore in detail in \cref{sec:wheeler_and_psortable_graphs}. Two straightforward approaches highlight the extremes of this spectrum:

\begin{enumerate}
    \item[A] \textbf{Full Compression, Difficult Indexing:} One could minimize the input trie into the smallest possible equivalent DFA using an algorithm like Revuz's~\cite{revuz1992minimisation}. While this yields optimal compression through DAG compression (see \cref{sed:fsa}), indexing the resulting general DFA is a notoriously difficult problem. As shown by Equi et al.~\cite{equiGraphsCannotBe2023}, polynomial-time indexing schemes for general DFAs cannot support sub-quadratic query times unless the Strong Exponential Time Hypothesis (SETH) fails, making this approach unsuitable for most indexing purposes.
    
    \item[B] \textbf{Full Indexability, No Compression:} At the other extreme, the input trie itself can be used as an index. Tries are Wheeler graphs~\cite{gagie2017wheeler}, specifically $1$--sortable automata (\cref{def:wheeler_automaton}), a property that makes them highly amenable to efficient indexing~\cite{cotumaccio2023co}. While this provides excellent query performance through the co--lexicographic ordering of states (see \cref{def:colex_order_on_automaton}), it offers no compression, as even highly repetitive subtrees are stored explicitly.
\end{enumerate}

This thesis proposes a novel algorithm that finds a sweet spot in this trade-off, which we develop throughout \cref{chp:project_overview}. The central idea is to partially minimize the input trie while ensuring that the resulting automaton remains efficiently indexable. We achieve this by leveraging the theory of $p$--sortable graphs (see \cref{sec:wheeler_and_psortable_graphs}), developing a method that strategically increases the sortability parameter $p$ (a parameter interpolating between trade-offs A and B above described) just enough to enable significant compression. The motivation for this approach is rooted in the observation that a small increase in $p$ can lead to substantial improved compression: as noted by Policriti et al.~\cite{manziniRationalConstructionWheeler2024}, there are cases where increasing $p$ from 1 to 2 allows for an exponential reduction in the automaton's size, a phenomenon we explore in detail in \cref{sec:wheeler_and_psortable_graphs}.

Our compression scheme, presented in \cref{chp:project_overview}, works by first sorting the trie's nodes by the co--lexicographic order of the strings connecting them to the root. Then, the scheme partitions this sorted sequence of nodes into $p$ disjoint subsequences called \emph{chains}, where each chain preserves the co--lexicographic order of its nodes. Our method optimizes this organization to merge the maximum number of adjacent Myhill--Nerode--equivalent states while ensuring $p$--sortability. 
We frame this problem as a string partitioning problem. Consider the sequence of nodes in the trie, when read in co--lexicographic order (see \cref{def:colex_order_on_automaton}), as a single long string. The ``character'' corresponding to each node is its Myhill--Nerode equivalence class (see \cref{def:myhill-nerode}), which determines if it can be merged with other nodes. The task is to partition this string of nodes into $p$ subsequences such that the number of equal-letter runs (\cref{def:run}) is minimized. Minimizing the number of runs directly corresponds to maximizing the number of merged states, yielding a compact $p$--sortable automaton. We call this problem the \emph{String Partitioning Problem} (\cref{def:string_partitioning_problem}).

\begin{restatable}[Run]{definition}{rundef}
    \label{def:run}
    Let $S = s_1s_2\ldots s_n$ be a string over an alphabet $\Sigma$. A substring $S[i \dots j] = s_i s_{i+1} \ldots s_j$ (where $1 \le i \le j \le n$) is a \textit{run} if it satisfies the following conditions:
    \begin{enumerate}
        \item \textbf{Homogeneity:} All characters in the substring are identical, i.e., $s_k = s_i$ for all $k$ such that $i \le k \le j$.
        \item \textbf{Maximality:} The substring cannot be extended to the left or right without violating homogeneity. That is:
        \begin{itemize}
            \item If $i > 1$, then $s_{i-1} \neq s_i$.
            \item If $j < n$, then $s_{j+1} \neq s_j$.
        \end{itemize}
    \end{enumerate}
\end{restatable}

\begin{restatable}{definition}{setsubseqdef}
    For a string $S = s_1s_2\ldots s_n$ over an alphabet $\Sigma$, and a set $I=\{i_1, i_2, \ldots, i_k\} \in [n]$ with $i_1 < i_2 < \cdots < i_k$, we define $S[I] := s_{i_1} s_{i_2} \ldots s_{i_k}$ as the subsequence indexed by $I$.
\end{restatable}

\begin{restatable}{definition}{runsdef}
    Let $\tau(S)$ be the number of runs in a string $S$, i.e., $\tau(S) = 1 + |\{i \in [n-1]:S[i] \neq S[i+1]\}|$.
\end{restatable}

\begin{restatable}[String Partitioning Problem]{definition}{stringpartitioningdef} \label{def:string_partitioning_problem}
    Let $\mathcal{I} = (S, p)$ be an instance of the String Partitioning Problem where $S$ is a string of length $n$ over an alphabet $\Sigma$, and $p$ is a positive integer. The output of the problem is a partition $\mathcal{P} = I_1,\dots,I_p$ of $[n]$ such that $\gamma(\mathcal{P}) = \sum_{i=1}^p \tau(S[I_i])$ is minimized.
\end{restatable}

\begin{example}[String Partitioning]
\label{ex:string-partitioning}
    Let us consider the string $S = \texttt{2213122152}$ of length 10. The number of runs in $S$ is 8, given by the decomposition $\texttt{(22)(1)(3)(1)(22)(1)(5)(2)}$.
    We want to partition the set of indices $\{1, \dots, 10\}$ into two sets, $I_1$ and $I_2$, to minimize the total number of runs in the corresponding subsequences.

    A possible partition is:
    \begin{itemize}
        \item $I_1 = \{3, 5, 8, 9\}$
        \item $I_2 = \{1, 2, 4, 6, 7, 10\}$
    \end{itemize}

    This partition yields the following subsequences:
    \begin{itemize}
        \item The subsequence corresponding to $I_1$ is $S[I_1] = \texttt{1115}$. This subsequence has 2 runs: $(\texttt{111})$ and $(\texttt{5})$.
        \item The subsequence corresponding to $I_2$ is $S[I_2] = \texttt{223222}$. This subsequence has 3 runs: $(\texttt{22})$, $(\texttt{3})$, and $(\texttt{222})$.
    \end{itemize}

    The total number of runs for this partition is $2 + 3 = 5$. This is a reduction from the original 8 runs in $S$. This example illustrates how partitioning a string's indices can reduce the total number of runs in the resulting subsequences.
\end{example}

In Section \ref{sec:reduction_to_mwpbm}, we show that the String Partitioning Problem can be reduced to the problem of finding a minimum weight perfect matching on a bipartite graph (MWPBM), allowing us to use efficient, well-studied algorithms to find the optimal solution. 

To sum up, the main contribution of the thesis is to show how to convert a trie (recognizing a finite regular language) to an equivalent small compressed automaton that is $p$--sortable by construction (for any parameter $p$ specified by the user) and thus supports efficient queries using the data structure developed by Cotumaccio et al.~\cite{cotumaccio2023co}. 
The larger $p$ is, the better compression we can achieve (but the slower queries will be).
As our experimental results in \cref{chp:experiments} will show, in both the repetitive and non-repetitive scenarios, our method allows for a twofold reduction of the number of states by just increasing $p$ from 1 (the original trie) to $2$. By further increasing $p$, the number of states quickly approaches that of the smallest DFA for the language. 

\section{Structure of the Thesis}
This thesis is organized into several chapters, each building upon the last to provide a comprehensive exploration of our proposed trie compression scheme.

In \cref{chp:preliminary}, we begin by introducing the fundamental concepts and notation that are essential for understanding the subsequent chapters. This includes theoretical background on automata theory, labeled trees and tries, the Extended Burrows-Wheeler Transform (XBWT), Wheeler and $p$--sortable graphs, and the min-weight perfect bipartite matching (MWPBM) problem, which form the theoretical foundations of our work.

\Cref{chp:project_overview} then transitions from theory to practice, presenting the core ideas behind our novel trie compression algorithm. We introduce the concept of collapsing nodes and detail the reduction to a MWPBM problem, which is the central mechanism of our approach.

Following this, \cref{chp:experiments} is dedicated to the empirical evaluation of our method. We describe the implementation of our algorithm and present a series of experiments designed to assess its performance. This includes an analysis of the execution time and compression ratios achieved.

Finally, in \cref{chp:conclusions}, we summarize our findings and discuss the implications of our results. We also outline potential avenues for future research, including possible optimizations and extensions of the proposed scheme.