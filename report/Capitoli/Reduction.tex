\chapter{Reduction to the assignment problem}

\section{Introduction}
In this chapter we will show how we can reduce the problem of finding the optimal partition of the nodes of a tree given their equivalence classes into $p$ chains to the Minimum Weight Perfect Bipartite Matching problem. This reduction will allow us to solve the problem in polynomial time as showed in the previous chapter.

Then we will show how to optimize the reduction by introducing some constraints that will allow us to reduce the number of edges in the bipartite graph, and also we will show how to move from the Minimum Weight Perfect Bipartite Matching problem to the more studied Maximum Weight Perfect Bipartite Matching problem without losing generality.

\section{The reduction}
\subsection{Bipartite graph construction}
Let $T$ be a tree with $t$ nodes and $p$ the number of chains we want to partition the nodes into. Let $E$ be the set of equivalence classes of the nodes of $T$. We can construct a bipartite graph $G = (V, E)$ such that vertices are divided in two disjoint sets $V = V_1 \cup V_2$ in the following way:

Consider The example in Figure \ref{fig:reduction_example} where we have a tree $T$ with $t=7$ nodes, $p = 2$ chains and the equivalence classes $E = \{1,2,1,3,1,2,2\}$ sorted accordingly to the upward path $\pi$ of each node of the tree. We can construct the two distinct sets $V_1$ and $V_2$ of the bipartite graph $G$ as follows:
\begin{itemize}
    \item $V_1$ contains $t + p$ nodes composed by $p$ source nodes $s_1, s_2, \dots, s_p$ and the $t$ elements of $E$ ordered.
    \item $V_2$ contains $t + p$ nodes composed by the $t$ elements of $E$ ordered and $p$ destination nodes $d_1, d_2, \dots, d_p$.
\end{itemize}

Notice that it is important to consider the order of the nodes of the two sets $V_1$ and $V_2$ as stated above, because we will need to connect the source nodes to the destination nodes in a way that will allow us to find the optimal partition of the nodes of the tree. In Figure \ref{fig:reduction_example} the nodes are ordered from top to bottom.

Then the edges of the graph $G$ will be constructed in the following way:
\begin{enumerate}
    \item The sources nodes $s_i \in V_1$ for $i = 1, 2, \dots, p$ are connected to the first $p$ nodes with distinct equivalence class in $V_2$ with weight $1$.
    \item Each of the $t$ nodes of the tree in $V_1$ are connected to the first $p$ nodes with distinct class \textbf{after} the corresponding node in $T$ placed in $V_2$ with weight $1$.
    \item Each of the $t$ nodes of the tree in $V_1$ are also connected to the first node with the same class in $V_2$ \textbf{after} the corresponding node in $T$ placed in $V_2$ with weight $0$. If there is no node with the same class in $V_2$ then $p$ edges with weight $0$ are added to each of the destination nodes $d_i \in V_2$ for $i = 1, 2, \dots, p$.
\end{enumerate}

Notice that when we talk about the same $V_1$ node placed in $V_2$ we are referring to the corresponding node in $V_2$ that derive from the same node in the original tree $T$ since the nodes of the tree are placed ordered in both sets $V_1$ and $V_2$. In Figure \ref{fig:reduction_example} and \ref{fig:reduction_small_examples} the nodes correspondness is achieved by putting the two nodes at the same level.

At the end the resulting bipartite graph $G$ will have $2t + 2p$ nodes and $t \cdot p$ edges. The weight of the edges will be $0$ or $1$.

Let's see a small example for each case, consider $p=2$. In Figure \ref{fig:reduction_small_examples}-(a) there is an example for the sources' edges, as stated before, for each source $p$ nodes with weight $1$ are created and connected to the first $p$ nodes with distinct equivalence class in $v_2$.

\begin{figure}[H]
    \centering
    \begin{tabular}{ccc}
        \begin{tikzpicture}[node distance={10mm}, thick, auto=center, main/.style = {draw, circle}] 
            \node[main] (1s) {$s_1$};
            \node[main] (2s) [below of=1s] {$s_2$}; 
            \node[main] (3s) [below of=2s] {$1$};
            \node[main] (4s) [below of=3s] {$1$};
            \node[main] (5s) [below of=4s] {$2$};
            
            \node[main] (1d) [right=3cm of 3s] {$1$};
            \node[main] (2d) [below of=1d] {$1$};
            \node[main] (3d) [below of=2d] {$2$};
            
            \draw[red, ->] (1s) -- (1d);
            \draw[red, ->] (1s) -- (3d);
            \draw[red, ->] (2s) -- (1d);
            \draw[red, ->] (2s) -- (3d);
        \end{tikzpicture} &
        \begin{tikzpicture}[node distance={10mm}, thick, auto=center, main/.style = {draw, circle}] 
            \node[main] (3s) [below of=2s] {$1$};
            \node[main] (4s) [below of=3s] {$2$};
            \node[main] (5s) [below of=4s] {$1$};
            \node[main] (6s) [below of=5s] {$3$};
            \node[main] (1d) [right=3cm of 3s] {$1$};
            \node[main] (2d) [below of=1d] {$2$};
            \node[main] (3d) [below of=2d] {$1$};
            \node[main] (4d) [below of=3d] {$3$};
            
            \draw[red, ->] (3s) -- (2d);
            \draw[red, ->] (3s) -- (4d);
            \draw[green, ->] (3s) -- (3d);
        \end{tikzpicture} &
        \begin{tikzpicture}[node distance={10mm}, thick, auto=center, main/.style = {draw, circle}] 
            \node[main] (7s) [below of=6s] {$1$};
            \node[main] (9s) [below of=7s] {$2$};
            \node[main] (5d) [below of=4d] {$1$};
            \node[main] (6d) [below of=5d] {$2$};
            \node[main] (8d) [below of=6d] {$d_1$};
            \node[main] (9d) [below of=8d] {$d_2$};

            \draw[red, ->] (7s) -- (6d);
            \draw[green, ->] (7s) -- (8d);
            \draw[green, ->] (7s) -- (9d);
            \draw[green, ->] (9s) -- (8d);
            \draw[green, ->] (9s) -- (9d);
        \end{tikzpicture} \\
        (a) & (b) & (c) \\
    \end{tabular} 
    \caption[Reduction cases examples]{Considering $p=2$ these three examples shows how to connect nodes in the bipartite graph in the case of sources (a), tree nodes (b), and destinations (c).}
    \label{fig:reduction_small_examples}
\end{figure}

\subsection{Full example}

\begin{figure}[H]
    \centering
    \begin{tabular}{cc}
        \begin{tikzpicture}[node distance={10mm}, thick, auto=center, main/.style = {draw, circle}] 
            \node[main] (1s) {$s_1$};
            \node[main] (2s) [below of=1s] {$s_2$}; 
            \node[main] (3s) [below of=2s] {$1$};
            \node[main] (4s) [below of=3s] {$2$};
            \node[main] (5s) [below of=4s] {$1$};
            \node[main] (6s) [below of=5s] {$3$};
            \node[main] (7s) [below of=6s] {$1$};
            \node[main] (8s) [below of=7s] {$2$};
            \node[main] (9s) [below of=8s] {$2$};
            \node[main] (1d) [right=3cm of 3s] {$1$};
            \node[main] (2d) [below of=1d] {$2$};
            \node[main] (3d) [below of=2d] {$1$};
            \node[main] (4d) [below of=3d] {$3$};
            \node[main] (5d) [below of=4d] {$1$};
            \node[main] (6d) [below of=5d] {$2$};
            \node[main] (7d) [below of=6d] {$2$};
            \node[main] (8d) [below of=7d] {$d_1$};
            \node[main] (9d) [below of=8d] {$d_2$};
            
            \draw[red, ->] (1s) -- (1d);
            \draw[red, ->] (1s) -- (2d);
            \draw[red, ->] (2s) -- (1d);
            \draw[red, ->] (2s) -- (2d);
            \draw[red, ->] (3s) -- (2d);
            \draw[red, ->] (3s) -- (4d);
            \draw[green, ->] (3s) -- (3d);
            \draw[red, ->] (4s) -- (3d);
            \draw[red, ->] (4s) -- (4d);
            \draw[green, ->] (4s) -- (6d);
            \draw[red, ->] (5s) -- (4d);
            \draw[green, ->] (5s) -- (5d);
            \draw[red, ->] (5s) -- (6d);
            \draw[red, ->] (6s) -- (5d);
            \draw[red, ->] (6s) -- (6d);
            \draw[green, ->] (6s) -- (8d);
            \draw[green, ->] (6s) -- (9d);
            \draw[red, ->] (7s) -- (6d);
            \draw[green, ->] (7s) -- (8d);
            \draw[green, ->] (7s) -- (9d);
            \draw[green, ->] (8s) -- (7d);
            \draw[green, ->] (9s) -- (8d);
            \draw[green, ->] (9s) -- (9d);
        \end{tikzpicture} &
        \begin{tikzpicture}[node distance={10mm}, thick, auto=center, main/.style = {draw, circle}] 
            \node[main] (1s) {$s_1$};
            \node[main] (2s) [below of=1s] {$s_2$}; 
            \node[main] (3s) [below of=2s] {$1$};
            \node[main] (4s) [below of=3s] {$2$};
            \node[main] (5s) [below of=4s] {$1$};
            \node[main] (6s) [below of=5s] {$3$};
            \node[main] (7s) [below of=6s] {$1$};
            \node[main] (8s) [below of=7s] {$2$};
            \node[main] (9s) [below of=8s] {$2$};
            \node[main] (1d) [right=3cm of 3s] {$1$};
            \node[main] (2d) [below of=1d] {$2$};
            \node[main] (3d) [below of=2d] {$1$};
            \node[main] (4d) [below of=3d] {$3$};
            \node[main] (5d) [below of=4d] {$1$};
            \node[main] (6d) [below of=5d] {$2$};
            \node[main] (7d) [below of=6d] {$2$};
            \node[main] (8d) [below of=7d] {$d_1$};
            \node[main] (9d) [below of=8d] {$d_2$};
            
            \draw[red, ->] (1s) -- (1d);
            \draw[red, ->] (2s) -- (2d);
            \draw[green, ->] (3s) -- (3d);
            \draw[red, ->] (4s) -- (4d);
            \draw[green, ->] (5s) -- (5d);
            \draw[green, ->] (6s) -- (8d);
            \draw[red, ->] (7s) -- (6d);
            \draw[green, ->] (8s) -- (7d);
            \draw[green, ->] (9s) -- (9d);
        \end{tikzpicture} \\
        (a) & (b) \\
    \end{tabular}
    \caption[Reduction full example]{Example of a reduction for the sorted nodes' equivalency classes $E = \{1,2,1,3,1,2,2\}$. In (a), we have the resulting bipartite graph constructed from $S$. In (b), we have the resulting perfect matching for the graph in (a) having weight $4$. Green edges weigh $0$, while red edges weigh $1$.}
    \label{fig:reduction_example}
\end{figure}

\subsection{Proof of correctness}

\section{Heuristics and Improvements}
